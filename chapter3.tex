\chapter{Some non-UFDs of algebraic integers}\label{notUFD}

%--------------------------------------------------------
% FIRST SECTION.
% We show tht many complex domains of the form h(sqrt(m))
% and Z[sqrt(m)] are not UFD
%--------------------------------------------------------
\section{Some normed complex domains that are not an UFD}

In this section, we'll give a quite general condition on $m > 0$
which ensure that $\hhm{m}$ or $\ZZM{m}$ is not an UFD.

\bigskip
Let's start with a proposition about $\ZZM{m}$:
%-------------------------------------------------
% m = 1 or 2 (mod 4) positive non-square composite
% implies Z[sqrt(-m)] not UFD
%-------------------------------------------------
\begin{thm}\label{m_neg_composite_implies_not_uniqueness_1}
If $m \in \Z$ a composite number such that $m > 1$, then
$\ZZM{m}$ is not an UFD.
\end{thm}

\begin{proof}
%
Let \,$p$\, the smallest prime number with\, $p \divides m$,
\,and let\, $m = ph$,\, with $h \in \Nzero$.

Since $m$ is composite, of course\, $p < m \then h > 1
\then m = ph \geq p^2$ (as follows easily from the
minimality of $p$).

Now, in $\ZZM{m}$ we clarly have:
$$\left(\sqrt{-m}\right)^2 = -m = p\cdot(-h)$$
so that in $\ZZM{m}$ is\, $p \divides \left(\sqrt{-m}\right)^2$
\,but\, $p \notdivides \sqrt{-m}$\,
(as one can immediately check observing that the equality
\,$\sqrt{-m} = p(a + b \sqrt{-m})$,\, $a, b \in \Z$\: would
imply\, $pb = 1 \then p \divides 1$,\, a contradiction).

So if we prove that $p$ is irreducible in $\ZZM{m}$
we also immediately get that $\ZZM{m}$ is not an UFD.

Argue by contradiction, assuming that\, $p = \alpha\beta$\,
for\, $\alpha, \beta \in \ZZM{m}$ not units (\ie such
that\, $\Norm{\alpha} > 1$\, and\, $\Norm{\beta} > 1$).
Then, writing\, $\alpha = \left(a + b\sqrt{-m}\right)$\,
and \,$\beta = \left(c + d\sqrt{-m}\right)$ \,for\,
$a, b, c, d \in \Z$,\, we obtain:
$$
m \ge p^2 = \Norm{p} = \Norm{\alpha}\Norm{\beta} >
\Norm{\alpha} = a^2 + b^2 m \geq b^2 m
$$
so that:
$$ b^2 m < m \then b = 0. $$

It can also be proved in an identical manner that\,
$d = 0$,\, so that\, $p = ac$,\, with \,$a,c \in \Z$,\,
which implies\, $ a = \pm{1}$ or $c = \pm{1}$\, since
$p$ is prime in $\Z$.

But then also $\alpha = \pm 1$ or $\beta = \pm 1$, a contradiction
as $\alpha$ and $\beta$ are not units.
%
\end{proof}

Since $\hhm{m} = \ZZM{m}$ if $m \congruent 1$ or $2$ (mod $4$),
we immediately have the following result:

%-------------------------------------------------
% m = 1 or 2 (mod 4) positive squarefree composite
% implies h(sqrt(-m)) not UFD
%-------------------------------------------------
\begin{cor}\label{m_neg_composite_implies_not_uniqueness_2}
If\, $m > 1$\, is a composite squarefree integer with\,
\mbox{$m \congruent 1\; \textrm{or}\ 2\;\mathrm{(mod\;4)}$,}
then $\hhm{m}$ is not an UFD.
\end{cor}

%--------------------------------------------
% m = 3 (mod 4) positive squarefree composite
% implies h(sqrt(-m)) not UFD
%--------------------------------------------
A similar result holds also for\, $m \congruent 3$ (mod $4$):
\begin{thm}\label{m_neg_composite_implies_not_uniqueness_3}
If\, $m > 1$\, is a composite squarefree integer with\,
\mbox{$m \congruent 3\;\mathrm{(mod\;4)}$,} then $\hhm{m}$ is
not an UFD.
\end{thm}

\begin{proof}
%
As in the proof of previous theorem, let\, $p$\, the
smallest prime number such that\, $p \divides m$,\, and
write\, $m = pH$\, for\, $H \in \Nzero$\,;\, then we easily
get\, $p^2 \geq m$.

Again, in $\hhm{m}$ we have:
$$
\left(\sqrt{-m}\right)^2 = -m = p\cdot(-H)
$$
so that in\, $\hhm{m}$\, is\, $p \divides \left(\sqrt{-m}\right)^2$.
But in\, $\hhm{m}$\, is also $p \notdivides \sqrt{-m}$,\, since the
assumption\, $p \divides \sqrt{-m}$\, would imply that for
suitable $a, b \in \Z$ is\footnote{%
in fact, it must be also $a \congruent b \pmod 2$, but this is
of no importance here.}:
$$
\sqrt{-m} = p\left(\frac{a + b\sqrt{-m}}{2}\right)
\then 2 = pb \then b = 1 \textrm{ and } p = 2
$$
so that, since\, $p \divides\! m$,\, $m$ must be even, which
contradicts the hypothesis that $m \congruent 3 \pmod 4$.

So our aim is now to show that $p$ is irreducible in $\hhm{m}$.

Let
\begin{equation}\label{p_factorization}
p = \left(\frac{a + b\sqrt{-m}}{2}\right)
    \left(\frac{c + d\sqrt{-m}}{2}\right)
\end{equation}

\,for\, $a, b, c, d \in \Z$ with
$a \congruent b$ (mod 2) and $c \congruent d$ (mod 2).
We want first to show that\, $b = 0$\, or\, $d = 0$.
Argue by contradiction assuming\, $ b \ne 0 \ne d $.\,
If we write $A := \abs{bd}$,\, it results then
$A \geq 1$, and:
\begin{equation}\label{A_limits}
A^2 m^2 =
16m^2 \cdot \frac{b^2}{4} \cdot \frac{d^2}{4}
\leq 16 \left(\frac{a^2 + b^2 m}{4}\right)
        \left(\frac{c^2 + d^2 m}{4}\right)
= 16\Norm{p} = 16p^2
\end{equation}
from which, being\, $p^4 \leq A^2 p^4 \leq A^2 m^2$,\,
derives immediately:
$$
p^4 \leq 16p^2 \then p^2 \leq 16 \then p \leq 4 \then
p = 2 \textrm{\ or\ } 3.
$$
Finally, since $p \divides m$ and $m$ odd, we have $p=3$.

From inequality~(\ref{A_limits}), recalling that
$m = pH = 3H$, we have now:
$$
16\cdot3^2 = 16p^2 \geq A^2 m^2 = A^2 \cdot 3^2 H^2 \then
(AH)^2 \leq 16 \then H \leq AH \leq 4
$$
from which the possibilities:\, $H = 1,\, 2,\, 3$ or $4$.

But since $m$ is odd it is\, $2 \ne H \ne 4$,\, since $m$
is squarefree it is\, $H \ne p = 3$\, and since $m$ is
composite it is\, $H \ne 1$,\, so that we get the desired
contradiction which lead us to deduce that $b=0$ or $d=0$.

At this point it's clear that we can suppose without loss
of generality that $b = 0$; thus, since\, $a \congruent b$
(mod 2),\, it results $a = 2r,\: r \in \Zzero$.

The equation~(\ref{p_factorization}) becomes now:
$$
2p = cr + dr\sqrt{-m}
$$
from which, since\, $r \ne 0$,\, immediatly derives that\,
$d = 0$ and so, being\, $c \congruent d$ (mod 2),\,
that $c = 2s$ for an appropriate $s \in \Z$

Thus we finally have:
$$
2p = cr = 2sr \then p = sr \then r = \pm{1}\ \textrm{or}
\ s = \pm{1}
$$
and then:
$$
\frac{a + b\sqrt{-m}}{2} = r = \pm{1} \textrm{\quad or \quad}
\frac{c + d\sqrt{-m}}{2} = s = \pm{1}
$$
which proves that $p$ is irreducible in\, $\hhm{m}$, so that we
can conclude that $\hhm{m}$ is not an UFD.
%
\end{proof}

%----------------------------------------------------------
% A FIRST GENERAL RESULT:
% m > 0 composite  ==>  h(sqrt(-m)) and Z[sqrt(-m)] not UFD
%----------------------------------------------------------
Summarizing the two \mbox{%
propositions
(\ref{m_neg_composite_implies_not_uniqueness_2})
and
(\ref{m_neg_composite_implies_not_uniqueness_3})%
} we get the following:

\begin{thm}\label{m_neg_composite_implies_not_uniqueness}
If $m > 1$ is a composite squarefree integer then $\hhm{m}$
is not an UFD.
\end{thm}

\medskip
For the domains $\ZZM{m}$ there is another severe limitation
to the possibility of being UFD:

\begin{thm}\label{m_neg_odd_implies_not_uniqueness_1}
If $m > 1$ is an  odd integer, then $\ZZM{m}$ is not an UFD.
\end{thm}

\begin{proof}
%
Let us first show that 2 is irreducible in $\ZZM{m}$.
Argue by contradiction, assuming that:\,
$2 = \alpha\beta = \left(a + b\sqrt{-m}\right)
\left(c + d\sqrt{-m}\right)$\, with\, $\alpha,\beta$\, not
units in $\Z{\sqrt{-m}}$\, (\,\ie such that\,
$\Norm{\alpha}>1$ and $\Norm{\beta}>1$\,).
We get immediately:
$$
4 = \Norm{\alpha}\Norm{\beta} \then
\Norm{\alpha} = \Norm{\beta} = 2 \then
a^2 + mb^2 = 2
$$
If would be $b \ne 0$, then we should have\, $b^2 \geq 1$\,
and so\, $2 = a^2 + mb^2 \geq mb^2 \geq m \geq 3$,\,
contradiction. So it is\, $b = 0 \then a^2 = 2$,\, again
a contradiction.
Thus we can conclude that 2 is irreducible in $\ZZM{m}$.

Since $m$ is odd, we have now that in $\ZZM{m}$ is:

~~$\bullet\ \; 2 \notdivides \left(1 + \sqrt{-m}\right)$

~~$\bullet\ \; 2 \notdivides \left(1 - \sqrt{-m}\right)$

~~$\bullet\ \; 2 \divides (1 + m) = 
\left(1 + \sqrt{-m}\right)\cdot\left(1 - \sqrt{-m}\right)$

which, together with the irreducibility of 2, proves that
$\ZZM{m}$ is not an UFD.
%
\end{proof}

Again, since $\hhm{m} = \ZZM{m}$ for $m \congruent 1$ (mod $4$),
we immediately get:

\begin{cor}\label{m_neg_odd_implies_not_uniqueness_2}
If\/ $m > 1$\/ is a squarefree odd integer such that\/
$m \congruent 1\ (\mathrm{mod}\ 4)$, then $\hhm{m}$
is not an UFD.
\end{cor}

We can give now a ``definitive'' result on the complex
normed domains of the form $\ZZM{m}$:

\begin{thm}\label{definitive_result_on_uniqueness_in_Z[sqrt(-m)]}
If $m$ is a positive integer, then $\ZZM{m}$
is an UFD if and only if\/ $m = 1$ or $m = 2$.
\end{thm}

\begin{proof}
%
From theorem~(\ref{normed_UFDs_1}) of chapter~\ref{UFD} we know that
\,$\ZZM{1}$\, and\, $\ZZM{2}$\, are UFDs.

On the contrary, if\, $m > 2$,\, it can be:
\begin{itemize}
 \item $m$ even and $m > 2$, so that $m$ is composite and then
       $\ZZM{m}$ is not UFD by
       theorem~(\ref{m_neg_composite_implies_not_uniqueness_1}),\: or:
 \item $m > 1$ and $m$ odd, so that $\ZZM{m}$ not an UFD by
       theorem~(\ref{m_neg_odd_implies_not_uniqueness_2}),
\end{itemize}
and $\ZZM{m}$ can't be an UFD in either case.
\end{proof}

Since\, $\hhm{m} = \ZZM{m}$\, when\, $m \congruent 1$ or $2$ (mod $4$),
a trivial implication of the previous result is:

\begin{cor}
If \,$m > 2$\, is a squarefree integer such that \,$m \congruent 1$ or\,
$2 \pmod 4$, then $\hhm{m}$ is not an UFD.
\end{cor}

The following proposition is very easy to prove too,
using the previous corollary and the theorem
(\ref{m_neg_composite_implies_not_uniqueness}):

\begin{thm}\label{h(sqrt(-m))_UFD_constraint_m}
If\, $m > 1$\, is a squarefree integer such that $\hhm{m}$
is an UFD, then\, $m = 2$\, or \,$m$ is prime and
$m \congruent 3 \pmod 4$.
\end{thm}

The previous theorem can be greatly strengthened, leading to
the following result:

\begin{thm}\label{first_theorem_on_uniqueness_in_complex_h(sqrt(-p))}
Let $p > 3$ be an odd prime with\, $p \congruent 3 \pmod 4$, and
suppose that there exist a prime \,$q < \frac{p}{4}$\, such that $-p$
is a quadratic residue of $4q$; then $\hhm{p}$ is not an UFD.
\end{thm}

\begin{proof}
%
Let $z \in \Z$ such that: 
\begin{equation}\label{foo}
z^2 \congruent -p \pmod{4q}.
\end{equation}
Then it's clearly\, $z \congruent 1 \pmod 2$,\, so that we have\,
$\alpha := \left(\frac{1 + z\sqrt{-p}}{2}\right) \in \hhm{p}$.
From~(\ref{foo}) we have
$\Norm{\alpha} = \frac{1}{4}(z^2 + p) \in \Z$\,
and\, $q \divides \Norm{\alpha}$\, in\, $\Z$, and
thus, \textit{a fortiori},\, $q \divides \Norm{\alpha}
= \alpha\cg{\alpha}$\, in\, $\hhm{p}$. But since\,
$1 \notcongruent 0 \pmod{q}$, we immediately have\,
$q \notdivides \alpha$ and $q \notdivides \cg{\alpha}$.
Then, to prove that $\hhm{p}$ is not an UFD, we have only to
show that $q$ is irreducible in $\hhm{p}$.

Argue by contradiction, assuming that\, $q = \beta\gamma$
\,for\, $\beta, \gamma \in \hhm{p}$ not units,
\ie such that\, $\Norm{\beta} > 1$, $\Norm{\gamma} > 1$.
From this, since $q$ is prime, we deduce:
$$
q^2 = \Norm{q} = \Norm{\beta\gamma} =
\Norm{\beta} \Norm{\gamma} \then
\Norm{\beta} = \Norm{\gamma} = q
$$
Writing now\, $\beta = \frac{a + b\sqrt{-p}}{2}$\, for
suitable\, $a,b \in \Z$\, with\, $a \congruent b \pmod 2$,
it results, supposing\, $b \ne 0$:
$$
4q = 4\,\Norm{\beta} = a^2 + pb^2 \geq pb^2 > 4qb^2
\then b^2 < 1 \then b = 0
$$
so that we surely have\, $b = 0$.\, But then\, $a^2 = 4q$,
\,which, being\, $q$\, prime, is a contradiction.
%
\end{proof}

As consequence of previous theorem we have:
\begin{cor}\label{second_theorem_on_uniqueness_in_complex_h(sqrt(-p))}
Let\, $p > 3$\, be an odd prime such that
$p \congruent 3 \pmod 4$\, and\, such that $\hhm{p}$
is an UFD; then there exists a prime\, $q$\, such that\,
$p = 4q - 1$.
In particular, it must be\, $p = 7$\, (if\, $q = 2$) or\,
$p \congruent 3 \pmod 8$\, (if\, $q$ odd).
\end{cor}

\begin{proof}
%
Argue by contradiction, assuming that\, $\frac{1}{4}(p + 1)$
\,-- which is an integer since \mbox{$p \congruent 3 \pmod 4$}
and greater than $1$ since $p > 3$ -- is composite, and let $r$
be one of its prime divisors.
Then we can write\, $\frac{1}{4}(p + 1) = rA$\, for\,
$A \in \N$, $A > 1$, from which we deduce:

\begin{itemize}

\item 
$\frac{1}{4}(p + 1) = rA \geq 2r \then
r \leq \frac{1}{8}(p + 1) < \frac{1}{8} \cdot 2p
\then r < \frac{1}{4}p$

\item
$r \left|\, \frac{1}{4}(p + 1)\right.
\then 4r \divides (p + 1)$

\end{itemize}

We have then:
$$
-p \congruent 1^2\ (\mathrm{mod}\ 4r), \quad
1 < r < \frac{1}{4}p, \quad
r\,\ \textrm{prime}
$$
so that, by
theorem~(\ref{first_theorem_on_uniqueness_in_complex_h(sqrt(-p))}),
we have immediately that\, $\hhm{p}$ is not an UFD, a contradiction.
%
\end{proof}

Finally, we can see an immediate consequence of previous
result:

\begin{cor}\label{corollary_on_uniqueness_in_complex_h(sqrt(-p))}
Let\, $1 < p < 200$\, be a prime such that\, $\hhm{p}$\, is an
UFD;\, then\, $p$ must be one of the following numbers:
$$ 2,\;3,\;7,\;11,\;19,\;43,\;67,\;163 $$
\end{cor}

\begin{proof}
%
The only $p$ primes with\, $p < 200$\, such that\,
$p \leq 7$ \,or\, $p \congruent 3$ (mod 8)\, are:
$$
2,\; 3,\; 7,\; 11,\; 19,\; 43,\; 59,\; 67,\;
83,\; 107,\; 131,\; 139,\; 163,\; 179
$$
Moreover:
\begin{itemize}

\item $\frac{59 + 1}{4} = 15 = 3 \cdot 5$

\item $\frac{83 + 1}{4} = 21 = 3 \cdot 7$

\item $\frac{107 + 1}{4} = 27 = 3^3$

\item $\frac{131 + 1}{4} = 33 = 3 \cdot 11$

\item $\frac{139 + 1}{4} = 35 = 5 \cdot 7$

\item $\frac{179 + 1}{4} = 45 = 3^2 \cdot 5$

\end{itemize}

Thus our claim follows immediately from
corollary~(\ref{second_theorem_on_uniqueness_in_complex_h(sqrt(-p))}).
%
\end{proof}

%------------------------------------------------------
% SECOND SECTION.
% We show that many real domains of the form h(sqrt(m))
% and Z[sqrt(m)] are not UFD
%------------------------------------------------------
\section{Some normed real domains that are not UFD}

In this section, we'll give a quite general condition on $m > 0$
ensuring that $\hh{m}$ or $\ZZ{m}$ is not an UFD.

The results that we'll can obtain about real domains will be
weaker then those found about complex domains, and our proofs
will be obiouvsly more elaborated.

\bigskip

Let's start with a proposition about $\ZZ{m}$:
\begin{thm}\label{basic_negative_result_for_uniqueness_1}
Let $m > 1$ be a non-square integer, and suppose that
there exists a prime $p$ such that:

~~$\bullet\quad\!\!\! \pm{p}$ are both quadratic non-residues
of $m$;

~~$\bullet\quad\!\! m$ is a quadratic residue of $p$.

Then $\ZZ{m}$ is not an UFD.
\end{thm}

\begin{proof} 
%
By hypothesis we can find\, $t \in \Z$\, such that
$t^2 \congruent m$ (mod $p$)\:$\then\, p \divides
(t^2 - m) = (t + \sqrt{m})(t - \sqrt{m})$,\, but\,
$p \notdivides (t + \sqrt{m})$ and
$p \notdivides (t - \sqrt{m})$
(as one can immediately check observing that the
equality: \,$t \pm{\sqrt{m}} = p(a + b \sqrt{m})$,\,
$a, b \in \Z$\: would imply\, $pb = \pm{1} \then
p \divides 1$,\, a contradiction).
So, if $p$ is irreducible in $\ZZ{m}$, this domain
can't be an UFD.

Argue by contradiction, assuming that\,
$p = \alpha \beta$\, with\, $\alpha,\beta \in \ZZ{m}$
not units, \ie such that $\Norm{\alpha} > 1,\;
\Norm{\beta} > 1$.\, If\, $\alpha = a + b\sqrt{m}$\, with\,
$a,b \in \Z$,\, we get then, for an appropriate choice of
sign:
$$
p^2 = \Norm{\alpha}\Norm{\beta} \then \Norm{\alpha} =
\Norm{\beta} = p \then \pm{p} = \pm{\,\Norm{\alpha}} = 
a^2 - mb^2 \congruent a^2\ (\mathrm{mod}\ m),
$$
a contradiction, as \,$\pm p$\, are both quadratic non-residues
of \,$m$.
%
\end{proof}

%
From previous result and from the equality $\hh{m} = \ZZ{m}$, holding
for $m \congruent 2$ or $3 \pmod 4$, we immediately get:

\begin{cor}\label{basic_negative_result_for_uniqueness_1'} 
Let $m > 1$ be a squarefree integer such that\, $m \congruent 2$ or\,
$3 \pmod 4$, and suppose that there exists a prime $p$ such that:

~~$\bullet\quad\!\! \pm{p}$ are both quadratic non-residues
of $m$;

~~$\bullet\quad\! m$ is a quadratic residue of $p$.

Then $\hh{m}$ is not an UFD.
\end{cor}
%

\medskip
A similar result holds also for $m \congruent 1$ (mod $4$):
%
\begin{thm}\label{basic_negative_result_for_uniqueness_2}
Let $m > 1$ be a squarefree integer  such that $m \congruent 1 \pmod 4$,
and suppose that there exists a prime $p > 2$ such that:

~~$\bullet\quad\!\! \pm{p}$ are both quadratic non-residues
of $m$;

~~$\bullet\quad\! m$ is a quadratic residue of $p$.

Then $\hh{m}$ is not an UFD.
\end{thm}

\begin{proof}
%
By hypothesis we can find\, $t \in \Z$\, such that:
$$
t^2 \congruent m\ (\mathrm{mod}\ p)\then p \divides
(t^2 - m) = (t + \sqrt{m})(t - \sqrt{m}),
$$
but\, $p \notdivides (t + \sqrt{m})$ and
$p \notdivides (t - \sqrt{m})$,\, since the assumption\,
$p \divides \left(t \pm{\sqrt{m}}\right)$\, would
imply that for suitable $a, b \in \Z$ is\footnote{%
in fact, it must be also $a \congruent b \pmod 2$, but this is
of no importance here.}:
$$
t \pm{\sqrt{m}} = p\left(\frac{a + b\sqrt{m}}{2}\right)
\then \pm{2} = pb \then p \divides 2
$$
which contradicts our assumption that $p > 2$.

So, in order to prove that $\hh{m}$ is not an UFD, we
have only to show that $p$ is irreducible in $\hh{m}$.

Argue by contradiction, assuming that\,
$p = \alpha \beta$\, with\, $\alpha,\beta \in \hh{m}$
not units, \ie such that\,
$\Norm{\alpha} > 1,\;\: \Norm{\beta} > 1$.\,
Writing\, $\alpha = \frac{a + b\sqrt{m}}{2}$,\, with\,
$a,\,b \in \Z,\ a \congruent b \pmod 2$,\, we
get then, for an appropriate choice of sign:
\begin{displaymath} \begin{split}
p^2 = \Norm{\alpha}\Norm{\beta} & \then \Norm{\alpha} =
\Norm{\beta} = p \\
& \then \pm{4p} = \pm{4\:\Norm{\alpha}} 
= a^2 - mb^2 \congruent a^2\ (\mathrm{mod}\ m)
\end{split} \end{displaymath}
Since\, $\gcd{m}{4} = 1$\, (as $m$ is odd), from the
last congruence it immediately derives that at least one
between $+p$ and $-p$ is a quadratic residue of $m$,
which gives us the desired contradiction.
%
\end{proof}
%
Let's now see another useful general result about $\ZZ{m}$:

\begin{thm}\label{basic_negative_result_for_uniqueness_3}
If $m > 1$ is a non-square integer such that $m \congruent 1 \pmod 4$,
then $\ZZ{m}$ is not an UFD.
\end{thm}

\begin{proof}
%
Since $m$ is odd, it results:
$$
2 \divides (1 - m) = (1 + \sqrt{m})(1 - \sqrt{m})
$$
while obiouvsy\, $2 \notdivides (1 + \sqrt{m})$\,
and\, $2 \notdivides (1 - \sqrt{m})$\, in\, $\ZZ{m}$.
So, by proving that $2$ is irreducible in $\ZZ{m}$,
we'll also prove that this domain is not an UFD.

Argue by contradiction, assuming that $2 = \alpha \beta$
for $\alpha, \beta \in \ZZ{m}$ not units.
Then\, $\Norm{\alpha} > 1,\: \Norm{\beta} > 1$, so that,
writing\, $\alpha = a + b\sqrt{m}$\, for $a, b \in \Z$,
we obtain:
$$
4 = \Norm{2} = \Norm{\alpha}\Norm{\beta} \then
\Norm{\alpha} = \Norm{\beta} = 2 \then \pm{2} =
a^2 - mb^2
$$

From this equality, since $m$ is odd, follows immediately
that\, $a \congruent b$ (mod 2),\, and so that
$a^2 \congruent b^2$ (mod $4$). But then:
$$
2 \congruent \pm{2} = a^2 - mb^2 \congruent a^2 - b^2
\congruent 0\ (\mathrm{mod}\ 4)
$$
a contradiction.
%
\end{proof}

\bigskip
Now we want to apply the results seen so far to three simple
but concrete examples.

\medskip
\textsf{Example 1.}  Let $m = 10$, which is obiouvsly
squarefree; since $10 \congruent 2$ (mod $4$), is\,
$$
\hh{10} = \ZZ{10} =
\left\{a + b\sqrt{10} : a,b \in \Z\right\}.
$$
Moreover, $10 \congruent 0^2$ (mod $2$),
while $\pm{2}$ are both quadratic non-residues of 10
(as one can immediatly check by direct calculations).

Since 2 is prime, by
theorem~(\ref{basic_negative_result_for_uniqueness_1}),\,
$\hh{10} = \ZZ{10}$\, isn't an UFD.

\medskip
\textsf{Example 2.}  Let $m = 85$, which is obiouvsly
squarefree; since $85 \congruent 1$ (mod $4$), is\,
$$
\hh{85} = \left\{\frac{a + b\sqrt{85}}{2}:\:
a,b \in \Z\ \,\textrm{and}\,\ a \congruent
b\ \textrm{(mod\ 4)}\right\}.
$$
Moreover, $85 \congruent 1^2$ (mod $3$),
while $\pm{3}$ are both quadratic non-residues of
$85$ (as they are both quadratic non-residue of
$5$). Since $3$ is a prime $> 2$, by
theorem~(\ref{basic_negative_result_for_uniqueness_2}),
$\hh{85}$ isn't an UFD.
 
\medskip
\textsf{Example 3.}  Let $m = 41$; since
$41 \congruent 1$ (mod $4$), the domain
$\ZZ{41}$ is not an UFD by
thorem~(\ref{basic_negative_result_for_uniqueness_3})

\bigskip

%-------------------------------------------------
% THIRD SECTION:
% we obtain some deeper result using theorems
% found in the previous section and the technical
% lemma of  Chapter (1), Section (4)
%-------------------------------------------------
\section{Deeper results for real domains}

Using the
\mbox{results
(\ref{basic_negative_result_for_uniqueness_1'})%
\,--\,%
(\ref{basic_negative_result_for_uniqueness_2})}
seen in the previos section, we can now see some general
and quite strong results for the real domains of the form
$\hh{m}$ and $\ZZ{m}$.

In our proofs we will heavily use the
lemma~(\ref{basic_technical_lem})
seen in chapter~\ref{preliminary}.

\begin{thm}\label{deeper_1}
Let \,$q_1, q_2, \ldots, q_n$\, (\/$n \geq 2$\/) be pairwise distinct
odd primes such that $q_1 \congruent 1 \pmod 4$,\, and let\,
$m = q_1 q_2 \cdots q_n$.
Then $\hh{m}$ and \,$\ZZ{m}$ are not UFDs.
\end{thm}

\begin{proof}
%
Since the primes $q_i$ are pairwise distinct and all odd, we
easily deduce from lemma~(\ref{basic_technical_lem}) that there
exists a prime $p$ such that:

\begin{itemize}

\item $p \congruent 1$ (mod $4$)

\item $p$\, is a quadratic non-residue
(mod $q_1$) and (mod $q_2$)

\item $p$\, is a proper quadratic residue (mod $q_i$),
$\forall\: i \in \{3, \ldots, n\}$

\end{itemize}

So, as\, $q_1 \congruent 1$ (mod $4$),\, we obtain:
$$
\Leg{\pm{p}}{q_1} = \Leg{\pm{1}}{q_1}\Leg{p}{q_1}
= (+1) \cdot (-1) = -1
$$
But \,$q_1 \divides m$,\, so that,
\emph{a fortiori},\, $+p$ and $-p$ are both
quadratic non-residues of $m$;\, moreover, since\,
$p \congruent 1$ (mod $4$),\, using quadratic
reciprocity law, we get:
%
{\setlength\arraycolsep{2pt}
\begin{eqnarray*}
\Leg{m}{p} & = &\Leg{q_1 q_2 q_3 \cdots q_n}{p} =
\Leg{q_1}{p} \Leg{q_2}{p} \Leg{q_3}{p} \cdots
\Leg{q_n}{p} = \\[7pt]
& = & \Leg{p}{q_1} \Leg{p}{q_2} \Leg{p}{q_3} \cdots
\Leg{p}{q_n} = (-1)(-1)(+1)\cdots(+1) = +1
\end{eqnarray*}}
{\flushleft
so that $m$ is a quadratic residue of $p$.}

Since\, $p > 2$\, (as $p$ is an odd prime), from
theorems~(\ref{basic_negative_result_for_uniqueness_1})
and~(\ref{basic_negative_result_for_uniqueness_2})
and corollary~(\ref{basic_negative_result_for_uniqueness_1'}),
we conclude that $\hh{m}$ and $\ZZ{m}$ are not UFDs.
%
\end{proof}


\begin{thm}\label{deeper_2}
Let \,$q_1, q_2, \ldots, q_n$\, ($n \geq 1$) be
pairwise distinct odd primes such that\, $q_1 \congruent 1 \pmod 4$,\,
and let\, $m = 2 q_1 q_2 \cdots q_n$.
Then $\hh{m}$ and \,$\ZZ{m}$ are not UFDs.
\end{thm}

\begin{proof}
%
Since the primes $q_i$ are pairwise distinct and all odd, we easily
deduce from lemma~(\ref{basic_technical_lem}) that there exists
a prime $p$ such that:

\begin{itemize}

\item $p \congruent 5$ (mod 8)

\item $p$\, is a quadratic non-residue (mod $q_1$)

\item $p$\, is a proper quadratic residue (mod $q_i$),
$\forall\: i \in \{2, \ldots, n\}$

\end{itemize}

So, as\, $q_1 \congruent 1$ (mod $4$),\, we obtain:
$$
\Leg{\pm{p}}{q_1} = \Leg{\pm{1}}{q_1}\Leg{p}{q_1}
= (+1) \cdot (-1) = -1
$$
But $q_1 \divides m$, so that, \emph{a fortiori}, $+p$ and $-p$
are both quadratic non-residues of $m$; moreover, since
$p \congruent 1$ (mod $4$), using quadratic reciprocity law, we get:
\medskip
\setlength\arraycolsep{2pt}
\begin{eqnarray*}
\Leg{m}{p} & = &\Leg{2 q_1 q_2 \cdots q_n}{p} =
\Leg{2}{p}\Leg{q_1}{p} \Leg{q_2}{p} \cdots
\Leg{q_n}{p} = \\[7pt]
& = & \Leg{2}{p}\Leg{p}{q_1} \Leg{p}{q_2} \cdots
\Leg{p}{q_n} = (-1)(-1)(+1)\cdots(+1) = +1
\end{eqnarray*}
{\flushleft
so that $m$ is a quadratic residue of $p$.}

Since\, $p > 2$\, (as $p$ is an odd prime), from
theorems~(\ref{basic_negative_result_for_uniqueness_1})
and~(\ref{basic_negative_result_for_uniqueness_2})
and corollary~(\ref{basic_negative_result_for_uniqueness_1'}),
we conclude that $\hh{m}$ and $\ZZ{m}$ are not UFDs.
%
\end{proof}

The two previous results can immediately be unified
in the following theorem:

\begin{thm}\label{deeper_3} 
Let $m > 1$ be a squarefree composite integer that has
a prime factor of the form $4k + 1$, $k \in \N$.
Then $\hh{m}$ and \,$\ZZ{m}$ are not UFDs.
\end{thm}

\mbox{Theorems (\ref{deeper_1})\,--\,(\ref{deeper_2})}
can easily be extended to analogous cases.
This is what we'll do in the two following propositions.

\begin{thm}\label{deeper_4}
Let \,$q_1, q_2, \ldots, q_n$\, ($n \geq 3$)
be pairwise distinct odd primes such that\, $q_1 \congruent 3 \pmod 4$,\,
and let\, $m = q_1 q_2 \cdots q_n$.
Then $\hh{m}$ and \,$\ZZ{m}$ are not UFDs.
\end{thm}

\begin{proof}
%
Since the primes $q_i$ are pairwise distinct and all odd,
we easily deduce from lemma~(\ref{basic_technical_lem})
that there exists a prime $p$ such that:

\begin{itemize}

\item $p \congruent 1$ (mod $4$)

\item $p$\, is a quadratic non-residue
(mod $q_2$) and (mod $q_3$)

\item $p$\, is a proper quadratic residue (mod $q_i$),
$\forall\: i \in \{1, 4, \ldots, n\}$

\end{itemize}

So, as\, $q_1 \congruent 3$ (mod $4$),\, we obtain:
$$
\Leg{p}{q_2} = -1 \quad \textrm{and} \quad
\Leg{-p}{q_1} = \Leg{-1}{q_1} \Leg{p}{q_1}
= (-1) \cdot (+1) = -1
$$
But \,$q_1 \divides m,\; q_2 \divides m$,\, so
that, \emph{a fortiori},\, $+p$ and $-p$ are
both quadratic non-residues of $m$;\, moreover,
since\, $p \congruent 1$ (mod $4$),\, using
quadratic reciprocity law, we get:
%
{\setlength\arraycolsep{2pt}
\begin{eqnarray*}
\Leg{m}{p} & = &\Leg{q_1 q_2 q_3 q_4 \cdots q_n}{p} =
\Leg{q_2}{p} \Leg{q_3}{p} \Leg{q_1}{p} \Leg{q_4}{p}
\cdots \Leg{q_n}{p} = \\[7pt]
& = & \Leg{p}{q_2} \Leg{p}{q_3} \Leg{p}{q_1} \Leg{p}{q_4}
\cdots \Leg{p}{q_n} = (-1)(-1)(+1)(+1)\cdots(+1) = +1
\end{eqnarray*}}
{\flushleft
so that $m$ is a quadratic residue of $p$.}

Since\, $p > 2$\, (as $p$ is an odd prime), from
theorems~(\ref{basic_negative_result_for_uniqueness_1})
and~(\ref{basic_negative_result_for_uniqueness_2})
and corollary~(\ref{basic_negative_result_for_uniqueness_1'}),
we conclude that $\hh{m}$ and $\ZZ{m}$
are not UFDs.
%
\end{proof}


\begin{thm}\label{deeper_5}
Let \,$q_1, q_2, \ldots, q_n$\, ($n \geq 2$)
be pairwise distinct odd primes such that\, $q_1 \congruent 3 \pmod 4$,\,
and let\, $m = 2 q_1 q_2 \cdots q_n$. 
Then $\hh{m}$ and \,$\ZZ{m}$ are not UFDs.
\end{thm}

\begin{proof}
%
Since the primes $q_i$ are pairwise distinct and all odd, we
easily deduce from lemma~(\ref{basic_technical_lem}) that there
exists a prime $p$ such that:

\begin{itemize}

\item $p \congruent 5$ (mod 8)

\item $p$\, is a quadratic non-residue (mod $q_2$)

\item $p$\, is a proper quadratic residue (mod $q_i$),
$\forall\: i \in \{1, 3, \ldots, n\}$

\end{itemize}

So, as\, $q_1 \congruent 3$ (mod $4$),\, we obtain:
$$
\Leg{p}{q_2} = -1 \quad \textrm{and} \quad
\Leg{-p}{q_1} = \Leg{-1}{q_1} \Leg{p}{q_1}
= (-1) \cdot (+1) = -1
$$
But \,$q_1 \divides m,\; q_2 \divides m$,\,
so that, \emph{a fortiori},\, $+p$ and $-p$
are both quadratic non-residues of $m$;\,
moreover, since\, $p \congruent 1$ (mod $4$),\,
using quadratic reciprocity law, we get:
\medskip
\setlength\arraycolsep{2pt}
\begin{eqnarray*}
\Leg{m}{p} & = &\Leg{2 q_1 q_2 \cdots q_n}{p} =
\Leg{2}{p}\Leg{q_2}{p} \Leg{q_1}{p} \Leg{q_3}{p}
\cdots \Leg{q_n}{p} = \\[7pt]
& = & \Leg{2}{p}\Leg{p}{q_2} \Leg{p}{q_1} \Leg{p}{q_3}
\cdots \Leg{p}{q_n} = (-1)(-1)(+1)(+1)\cdots(+1) = +1
\end{eqnarray*}
{\flushleft
so that $m$ is a quadratic residue of $p$.}

Since\, $p > 2$\, (as $p$ is an odd prime), from
theorems~(\ref{basic_negative_result_for_uniqueness_1})
and~(\ref{basic_negative_result_for_uniqueness_2})
and corollary~(\ref{basic_negative_result_for_uniqueness_1'}),
we conclude that $\hh{m}$ and $\ZZ{m}$ are not UFDs.
%
\end{proof}

%
\bigskip
\mbox{Theorems (\ref{deeper_3})\,--\,(\ref{deeper_4})\,%
--\,(\ref{deeper_5})} can be now unified, leading to the
following result:

\begin{thm}\label{DEEPER}
Let\, $m > 1$\, be a squarefree integer such that at least one between
$\hh{m}$ and $\ZZ{m}$ is an UFD; then there exist two distinct primes
$p,q$ such that $m = p$, or $m = 2p$ and
$p \congruent 3\ \mathrm{(mod\ 4)}$, or $m = pq$ and
$p \congruent q \congruent 3\ \mathrm{(mod\ 4)}$.
\end{thm}

\begin{proof}
%
If $m$ is prime, there is nothing to prove. So let's
suppose $m$ composite.

From theorem~(\ref{deeper_3}) it derives
that $m$ has no factors of the form $4k + 1$.
From this statement we immediately conclude that one of
the two following possibilities holds:
\begin{itemize}

\item\: $m$ has only two prime factors, no one of the
form $4k + 1$, \ie\,$m = 2p$ for $p$ prime such that
$p \congruent 3$ (mod $4$),\, or\, $m = pq$ for $p,q$
primes such that $p \congruent q \congruent 3$ (mod $4$).

\item\: $m$ has one of the form indicated in
theorems~(\ref{deeper_4})~and~(\ref{deeper_5}).

\end{itemize}

But if the second case hold, we could deduce from
theorems~(\ref{deeper_4}) and~(\ref{deeper_5})
themselves that both $\hh{m}$ and $\ZZ{m}$ aren't UFD,
which contradicts our hypothesis.
So, if $m$ is composite, the first case must hold,
and this proves the thoerem.
%
\end{proof}

%------------------------------------------
% FOURTH SECTION:
% we obtain some deeper results concerning
% Z[sqrt(m)] using theorems found in
% section 2 and the technical lemma of
% Chapter (1), Section (4)
%-------------------------------------------------
% NOTE: do *not* use \ZZ here!
\section{More general results for $\Z[\sqrt{m}]$} 

At this stage, we can reinforce theorem~(\ref{DEEPER})
if we limit our attention to the domains
$\ZZ{m}$, for $m > 1$ squarefree.

This fact is expressed in the following result:

\begin{thm}\label{Z_deeper_1}
Let $m > 1$ be a squarefree integer such that
$\ZZ{m}$ is an UFD; then there exists
a prime $p \congruent 3\ \mathrm{(mod\ 4)}$ such that\,
$m = 2$\, or\, $m = p$ \,or\, $m = 2p$.
\end{thm}

\begin{proof}
%
From theorem~(\ref{DEEPER}) we know that, if $\ZZ{m}$ is an UFD,
then either $m$ has one of forms listed in our statement,
or\, $m = p \congruent 1 \pmod 4$\, ($p$ prime)\, or\,
$m = pq$\, for\, $p,q$ primes and $p \congruent q \congruent 3 \pmod 4$.

But in both these latter situations is $m \congruent 1$\!
(mod $4$), so that $\ZZ{m}$ can't be an UFD by
thorem~(\ref{basic_negative_result_for_uniqueness_3}).
%
\end{proof}

\smallskip
Now, it would be a good thing to generalize further
theorem~(\ref{Z_deeper_1}), and see what we can say
about the domains $\ZZ{m}$, for $m > 1$ non-square but
not necessarly also squarefree.

As a first step, we can easily adjust \mbox{%
theorems (\ref{deeper_1})\,--\,(\ref{deeper_2})%
} and obtain the two following propositions~(\ref{Z_deeper_2})
and~(\ref{Z_deeper_3}):

\begin{thm}\label{Z_deeper_2}
Let $A$ be a positive integer,\, $q_1, q_2, \ldots, q_n$\,
($n \geq 2$) be pairwise distinct odd primes such that\,
$q_1 \congruent 1 \pmod 4$, and let\,
$m = q_1 q_2 \cdots q_n A^2$.
Then $\ZZ{m}$ is not an UFD.
\end{thm}

\begin{proof}
%
We easily deduce from lemma~(\ref{basic_technical_lem})
that there exists a prime $p$ such that:

\begin{itemize}

\item $p \congruent 1$ (mod $4$)

\item $p > A\, \then A \not\congruent 0$ (mod $p$)

\item $p$\, is a quadratic non-residue
(mod $q_1$) and (mod $q_2$)

\item $p$\, is a proper quadratic residue (mod $q_i$),
$\forall\: i \in \{3, \ldots, n\}$

\end{itemize}

So, as\, $q_1 \congruent 1$ (mod $4$),\, we obtain:
$$
  \Leg{\pm{p}}{q_1} = \Leg{\pm{1}}{q_1}\Leg{p}{q_1}
  = (+1) \cdot (-1) = -1
$$
But \,$q_1 \divides m$,\, so that,
\emph{a fortiori},\, $+p$ and $-p$ are both
quadratic non-residues of $m$;\, moreover, since\,
$p \congruent 1$ (mod $4$),\, using quadratic
reciprocity law, we get:
%
{\setlength\arraycolsep{2pt}
\begin{eqnarray*}
\Leg{m}{p} & = &\Leg{q_1 q_2 q_3 \cdots q_n A^2}{p}
= \Leg{q_1}{p} \Leg{q_2}{p} \Leg{q_3}{p} \cdots
\Leg{q_n}{p} \Leg{A^2}{p} = \\[7pt]
& = & \Leg{p}{q_1} \Leg{p}{q_2} \Leg{p}{q_3} \cdots
\Leg{p}{q_n} \Leg{A^2}{p} = 
(-1) (-1) (+1) \cdots (+1) (+1) = +1
\end{eqnarray*}}
\!\!so that $m$ is a quadratic residue of $p$.

From theorem~(\ref{basic_negative_result_for_uniqueness_1}),
we conclude that $\ZZ{m}$ is not an UFD.
%
\end{proof}


\begin{thm}\label{Z_deeper_3}
Let $A$ be a positive integer,\, $q_1, q_2, \ldots,
q_n$\, ($n \geq 1$) be pairwise distinct odd primes
such that\, $q_1 \congruent 1 \pmod 4$, and let\,
$m = 2 q_1 q_2 \cdots q_n A^2$.

Then $\hh{m}$ and \,$\ZZ{m}$ are not UFD.
\end{thm}

\begin{proof}
%
The proof is essentially identical to the proofs of
\mbox{%
theorems (\ref{deeper_2})\,--\,(\ref{Z_deeper_1}),%
} with the difference that the conditions on the prime
$p$ have to be substituted with the following ones:
\begin{itemize}

\item $p \congruent 5$ (mod $8$)

\item $p > A$\: (\,so that $A \not\congruent 0$
(mod $p$)\,)

\item $p$\, is a quadratic non-residue (mod $q_1$)

\item $p$\, is a proper quadratic residue (mod $q_i$),
$\forall\: i \in \{2, \ldots, n\}$

\end{itemize}

The argumentation proceeds in the usual way and
quickly lead to the proof of our claim.
%
\end{proof}

\mbox{Theorems (\ref{deeper_4})\,--\,(\ref{deeper_5})}
can be adjusted similarly, leading to the two following
propositions, whose proofs are now quite obvious:

\begin{thm}\label{Z_deeper_4}
Let $A$ be a positive integer, $q_1, q_2, \ldots, q_n$\,
($n \geq 3$) be pairwise distinct odd primes
such that\, $q_1 \congruent 3 \pmod 4$, and let\,
$m = q_1 q_2 \cdots q_n A^2$.
Then $\ZZ{m}$ is not an UFD.
\end{thm}

\begin{thm}\label{Z_deeper_5}
Let $A$ be a positive integer, $q_1, q_2, \ldots,
q_n$\, ($n \geq 2$) be pairwise distinct odd primes
such that\, $q_1 \congruent 1 \pmod 4$,
and let\, $m = 2 q_1 q_2 \cdots q_n A^2$.
Then $\ZZ{m}$ is not an UFD.
\end{thm}

\smallskip
Another step towards generalization is to improve
theorem~(\ref{basic_negative_result_for_uniqueness_3}):

\begin{thm}\label{Z_deeper_6}
Let\, $m = 4^k m_1$,\, with $k \in \N$,\, $m_1 \in \Nzero$
not a perfect square, and\, $m_1 \congruent 1 \pmod 4$.
Then $\ZZ{m}$ is not an UFD.
\end{thm}

\begin{proof}
%
In virtue of theorem~(\ref{basic_negative_result_for_uniqueness_3}),
we have only to analize the case $k \geq 1$.

In $\ZZ{m}$ we have then:
$$
  2 \divides (4 - 4^k m_1) = (4 - m) =
  (2+\sqrt{m})(2 - \sqrt{m})
$$
but\, $2 \notdivides (2 + \sqrt{m})$\, and\,
$2 \notdivides (2 - \sqrt{m})$,\, so that if we
prove the irreducibility of $2$ in $\ZZ{m}$
we can immediately conclude that this domanin is
not an UFD.

But obiouvsly it results $\ZZ{m} \subseteq \ZZ{m_1}$,
and $2$ is irreducible in $\ZZ{m_1}$
(\,as shown in the proof of
theorem~(\ref{basic_negative_result_for_uniqueness_3})\,),
so that it is \textit{a fortiori} irreducible in
$\ZZ{m}$.
%
\end{proof}

\medskip
We can finally summarize
\mbox{%
theorems\, (\ref{Z_deeper_2})\,--\,(\ref{Z_deeper_3})\,--\,%
(\ref{Z_deeper_4})\,--\,(\ref{Z_deeper_5})\,--\,%
(\ref{Z_deeper_6})\,%
} into the following result,
which generalizes theorem~(\ref{Z_deeper_1}):

\begin{thm}\label{Z_DEEPER}
Let $m = m_1 A^2$, with $A$ positive integer, $m_1 > 1$
squarefree integer, and assume that $\ZZ{m}$ is an
UFD.
Then there exists a prime\,
$p \congruent 3 \pmod 4$ such that\:
$m_1 = 2$\, or\, $m_1 = p$\, or\, $m_1 = 2p$.
\end{thm}

\begin{proof}
%
If $m_1 = 2$, there is nothing to prove.
Suppose now $m_1 > 2$ prime. If\,
$m_1 \congruent1 \pmod 4$, \,writing\,
$A = 2^k A_1$\, for $k \in \N$, $A_1 \in \N$ odd, and
putting\, $m_2:= {m_1}{A_1}^2$, we get:
$$ m = m_1 A^2 = 4^k m_2 $$
where $m_2$ is not a perfect square and
$m_2 = {m_1}{A_1}^2 \congruent 1\cdot1 \congruent 1 \pmod 4$.
But then we deduce from theorem~(\ref{Z_deeper_6}) that
$\ZZ{m}$ is not an UFD, a contradiction.

So we can conclude that if $m_1 > 2$ is prime, it must
be $m_1 \congruent 3$ (mod $4$).

Suppose now $m_1$ composite. From \mbox{%
theorems (\ref{Z_deeper_2})\,--\,(\ref{Z_deeper_3}),}
we immediately deduce that $m_1$ has no prime factors of
the form $4h + 1$. Then, of the following two
possibilities one must hold: either $m_1$ has only two
prime factors (and no one of this of the form $4h + 1$),
or $m$ has one of the forms indicated in \mbox{%
theorems (\ref{Z_deeper_2})\,--\,(\ref{Z_deeper_3}).}
But in the latter case, these theorems themselves say us
that $\ZZ{m}$ is not an UFD, against our
hypothesis.

So, if $m_1$ is composite, it results\, $m_1 = 2p$ or
$m_1 = pq$,\, for suitable distinct primes $p,q$ such
that \mbox{$p \congruent q \congruent 3$ (mod $4$).}

But if the second equality holds, writing\,
$A = 2^k A_1$,\, we deduce:
$$ m = m_1 A^2 = 4^k {A_1}^2 pq = 4^k m_2 $$
where
$$
  m_2 = pq {A_1}^2 \congruent 3 \cdot 3 \cdot 1 = 9
  \congruent 1\ \mathrm{(mod\ 4)}
$$
so that, since $m_2$ is not a perfect square,
$\ZZ{m}$ can't be an UFD in virtue of
theorem~(\ref{Z_deeper_6}), a contradiction.

Thus, if $m_1$ is prime, it must be\, $m_1 = 2$ or
$m_1 = p \congruent 3 \pmod 4$, while if $m$
is composite it must be\, $m = 2p$ for $p$ prime
with $p \congruent 3 \pmod 4$, and the
theorem is proved.
%
\end{proof}

