\chapter{Limits of our methods}\label{limits}

\section{Equivalence between %
property~(\ref{property_sharp}) and %
property~(\ref{property_sharp_weaker})%
}\label{eucilidean_and_quasi-euclidean}

%----------------------------
% A BIT VERBOSE INTRODUCTION
%----------------------------
While reading chapter~\ref{UFD}, you have probably
noted that, in order to show that property~(\ref{property_sharp})
(defined at page~\pageref{property_sharp}) holds for the rings
$\hh{m}$ of our interest, we have shown, as a matter of fact, that
a (apparently) stronger property holds, \ie
property~(\ref{property_sharp_weaker})
(defined at page~\pageref{property_sharp_weaker}).

Thus, a question that arises naturally is:

``Is property~(\ref{property_sharp_weaker}) truly stronger
than property~(\ref{property_sharp}), \ie does there exist a
squarefree integer $m$ such that $\hh{m}$ satisfies
property~(\ref{property_sharp}) but not
property~(\ref{property_sharp_weaker})?''

\smallskip
In this section we'll prove that the answer to this
question is ``no''\footnote{as a matter of fact,
we will prove a more general result.}.

\bigskip

Let's begin with the following:

%------------------------------
% DEFINITION OF QUASI-EUCLIDEAN
%------------------------------
\begin{defn}\label{quasi-euclidean}
Let $D$ a domain, and let\, $\Psi: D\minz
\rightarrow \N$\, a function such that:
$$
\forall\: \alpha, \beta \in D\minz \text{ with \,}
\Psi(\alpha) \geq \Psi(\beta),~\: \exists\:
\theta, \rho \in D:\: \alpha = \beta \theta + \rho,
\: \rho = 0 \textrm{\ or\ } \Psi(\rho) < \Psi(\alpha)
$$
\medskip
Then $\Psi$ is said to be a \emph{\/quasi-euclidean function}
and $D$ a \emph{\/quasi-euclidean domain}.
\end{defn}

%-----------------------------------------------------
% property (#-weaker) is related to norm-euclidean and
% property (#) is related to norm-quasi-euclidean
%-----------------------------------------------------
By theorems (\ref{sharp2_almost_equivalent_sharp})
and (\ref{sharp-weaker_equivalent_norm-euclidean}), we have that for
every normed domain $D \subseteq \QQ{m}$ such that $D \not\subseteq \Q$,
subsistence of property (\ref{property_sharp_weaker}) is equivalent to
the fact that $D$ is norm-euclidean, while subsistence of property
(\ref{property_sharp}) is equivalent to the fact that $D$ is a
norm-quasi-euclidean, \ie that its norm is a quasi-euclidean function.

\smallskip
Our main result can now be enunciated as follows:

%------------------------------------------------
% EQUIVALENCE BETWEEN EUCLIDEAN E QUASI-EUCLIDEAN
%------------------------------------------------
\begin{thm}\label{euclidean_equivalent_quasi-euclidean}
Let $D$ a domain and let \,$\Psi: D\minz \rightarrow \N$.\,
Then $\Psi$ is quasi-euclidean if and only if it is euclidean.
\end{thm}

\begin{proof}
%
We will prove separately the two implications.

\smallskip
\textbf{Implication 1~:}\:
$\Psi\,\text{ euclidean } \then\, \Psi\,\text{ quasi-euclidean}$

Suppose given $\alpha, \beta \in D\minz$ such
that $\Psi(\alpha) \geq \Psi(\beta)$. Since $\beta \ne 0$
and since $\Psi$ is an euclidean function, there exist
$\theta, \rho \in D$ with $\alpha = \beta \theta + \rho,\ \:
\rho = 0 \textrm{\ or\ } \Psi(\rho) < \Psi(\beta)$.
Since $\Psi(\beta) \leq \Psi(\alpha)$, we have then:
$$
\alpha = \beta\theta + \rho, \quad\! \rho = 0
\ \,\textrm{or}\ \, \Psi(\rho) < \Psi(\alpha)
$$
so that $\Psi$ is quasi-euclidean.

\medskip
\textbf{Implication 2~:}\:
$\Psi\,\text{ quasi-euclidean }\then\, \Psi\,\text{ euclidean}$

Suppose given $\alpha, \beta \in D$ such that $\beta \ne 0$.
We want to prove that there exist\, $\theta, \rho \in D$\, such that\,
$\alpha = \beta \theta + \rho$, for $\rho = 0$ or
$\Psi(\rho) < \Psi(\beta)$.

Obiouvsly, if $\beta \divides \alpha$, we can choose
\,$\theta:= \frac{\alpha}{\beta}$\, and\, $\rho:= 0$.
So we can suppose $\beta \notdivides \alpha$.

From this point on, we will argue by contradiction,
assuming that:
\begin{equation}\label{violation_of_norm-euclidean}
\forall\, \theta \in D:\:
\Psi(\alpha - \beta\theta) \geq \Psi(\beta)
\end{equation}
(obiouvsly we have $\alpha - \beta\theta \ne 0$, since
$\beta \notdivides \alpha$, so that the expression
$\Psi(\alpha - \beta\theta)$ makes sense).

Defining:
$$
\mathcal{A}:= \left\{\alpha - \beta\theta:\: \theta \in D
\right\}  \subseteq D\minz,
$$
we have\,
$\Psi\left(\mathcal{A}\right) =
\left\{\Psi(\alpha - \beta\theta):
\:\theta \in D \right\}
\subseteq \N \then \exists\: \theta_0 \in D:\:
\Psi(\alpha - \beta\theta_0) =
\min \Psi\left(\mathcal{A}\right)$,\,
%
and so:
%
\begin{equation}\label{theta0_minimality}
\forall\, \theta \in D:\:
\Psi(\alpha - \beta\theta_0) \leq 
\Psi(\alpha - \beta\theta)
\end{equation}
%

But from (\ref{violation_of_norm-euclidean})
we get\, $\Psi(\alpha - \beta\theta_0) \geq
\Psi(\beta)$, \,and so, since $\Psi$ is
quasi-euclidean, we have that there exist
$\theta_1, \rho_1$ such that:
$$
(\alpha - \beta\theta_0) = \beta\theta_1 + \rho_1,
\quad\! \rho_1 = 0\ \,\textrm{or}\, \ \Psi(\rho_1)
< \Psi(\alpha - \theta_0\beta).
$$
Since $\beta \notdivides \alpha$, we have also
$\beta \notdivides (\alpha - \beta\theta_0)$, so
that $\rho_1 \ne 0$. Thus we get:
$$
\Psi(\alpha - \theta_0\beta) > \Psi(\rho_1) =
\Psi\left(\alpha - \beta(\theta_0 + \theta_1)\right)
= \Psi(\alpha - \beta\overline{\theta}),
\quad\!\! \overline{\theta} \in D
$$
which, according with (\ref{theta0_minimality}), gives
us the desired contradiction.
%
\end{proof}

%------------------------------------------------------
% SECTION 2: COMPLETE CLASSIFICATION OF COMPLEX DOMAINS
% THAT ARE NORM-EUCLIDEAN
%------------------------------------------------------
\section{Complex domains that are not norm-euclidean}%
\label{not_norm-euclidean_complex_domains}

The basic result of this section is the following:

\begin{thm}
If $m \geq 15$ is a squarefree integer, then $\hhm{m}$ is
not norm-euclidean
\end{thm}

\begin{proof}
%
First, it's obvious that $\forall m \in \Z$ squarefree is:
$$
\hhm{m} \subseteq T(m):=
\left\{ \frac{x + y\sqrt{-m}}{2}\: :\ x, y \in \Z \right\}
$$
Suppose now by contradiction that $\hhm{m}$ is norm-euclidean;
then for every $\psi \in \QQM{m}$ we can find a\, $\theta \in \hhm{m}$
(and so \textit{\/a fortiori} a\, $\theta \in T(m)$)
such that\, $\Norm{\psi - \theta} < 1$.

So, letting\, $\psi := \frac{1}{4}+\frac{1}{4}\sqrt{-m}$
\,and\, $\theta = \frac{x}{2} + \frac{y}{2}\sqrt{-m}$\,
($x, y \in \Z$), we get: \smallskip
$$
(2x - 1),\: (2y - 1)\, \in \Zzero \then
\left\{
\begin{array}{l}
(2x - 1)^2 \geq 1 \\[6pt]
(2y - 1)^2 \geq 1
\end{array}
\right.
$$
and:
%
\begin{eqnarray*}
1 > \Norm{\psi - \theta} & = &
\left(\frac{x}{2} - \frac{1}{4}\right)^2 +
m\left(\frac{y}{2} - \frac{1}{4}\right)^2 =
\frac{1}{16}\left( (2x - 1)^2 + m(2y - 1)^2 \right)
\geq \\[6pt]
& \geq & \frac{1}{16}(1 + m) = \frac{1 + m}{16}
\geq \frac{1 + 15}{16} = 1 \;\then\; 1 > 1
\end{eqnarray*}
%
which give us the desired contradiction.
%
\end{proof}

Using the previous result, it's easy to classify
completely the complex norm-euclidean domains:

\begin{thm} 
\label{complete_classification_of_complex_normed_domains}
If $m$ is a squarefree integer, then $\hhm{m}$ is
norm-euclidean if and only if
$$ m = 1,\, 2,\, 3,\, 7\ \textrm{or}\ 11. $$
\end{thm}

\begin{proof}
%
First, we know by theorem~(\ref{normed_UFDs_1})
that $\hhm{m}$ is norm-euclidean if\,
$m = 1,\, 2,\, 3,\, 7\ \textrm{or}\ 11$.
On the other hand, from
theorem~(\ref{definitive_result_on_uniqueness_in_Z[sqrt(-m)]})
and from the fact that $\ZZM{m} = \hhm{m}$ if
\mbox{$m \congruent 1$ or $2$ (mod 4)}, we know that for
every\, $m > 2$\, with\, \mbox{$m \congruent 1$ or $2$
(mod 4)},\, $\hhm{m}$ is not an UFD, and so \textit{\/a fortiori}
it is not norm-euclidean.
Moreover, from the previous theroem, we know that if $m \geq 15$
is a squarefree integer, then $\hhm{m}$ is not norm-euclidean.
Hence, apart from $1$ and $2$, the only values of $m$
that can make $\hhm{m}$ norm-euclidean are\, $3,\,7$
and $11$, and the result follows.
%
\end{proof}

%------------------------------------------------------
% SECTION 2: PARTIAL CLASSIFICATION OF COMPLEX DOMAINS
% THAT ARE NORM-EUCLIDEAN -- BASICALLY LIMITED TO THE
% CASES  m = 2 or 3 (mod 4)
%------------------------------------------------------
\section{Real domains that are not norm-euclidean}%
\label{not_norm-euclidean_real_domains}

\begin{subsection}{Introduction}

In this section we will give a partial classification
of those real normed domains $\hh{m}$ that are or are
not norm-euclidean.

Unfortunately, we can't give a theorem such simple as
theorem~(\ref{complete_classification_of_complex_normed_domains}),
but only more limited results whose proofs will be more
elaboreted and complex.

\smallskip
As a first step, we see a simple lemma that will be useful later:
\begin{lem}\label{intseplemma}
If $\alpha, \beta \in \left[0, +\infty\right[$
and $\beta > \alpha + 1$, then there exists $n \in \N$
such that $\alpha < n < \beta$.
\end{lem}
\begin{proof}
Define $n := \lfloor\alpha\rfloor + 1$; then it results:
$$
\alpha < \lfloor\alpha\rfloor + 1 \leq \alpha + 1 < \beta
\then \alpha < n < \beta
$$
as desired.
\end{proof}

We will separate the study of the cases
$m \congruent 2 \text{ or } 3 \pmod 4$ and
$m \congruent 1 \pmod 4$
in two distinct subsections.
\end{subsection}

\smallskip
%------------------------------------------------------
% non norm-euclidean h(sqrt(m)) for m = 2 or 3 (mod 4)
%------------------------------------------------------
\begin{subsection}{Not norm-euclidean $\hh{m}$ for %
                   $m \congruent 2 \text{ or } 3 \pmod 4$}

We begin analyzing three particular cases, namely $m = 23$,
$m = 14$ and $m = 35$.

%---------------------------------------------
% Proof that h(sqrt(23) is not norm-euclidean
%---------------------------------------------
\begin{thm}\label{h(sqrt(23)) not norm-euclidean}
The domain \,$\hh{23} = \ZZ{23}$ isn't norm-euclidean.%
\footnote{this example is an adaptation of one that
is presented chapter~XIV of~\cite{H&W}.}
\end{thm}

\begin{proof}
%
We're going to prove our claim by showing that
$\ZZ{23}$ can't satisfy property~(\ref{property_sharp_1}).

Define $\tau := 1 + \frac{7}{23}\sqrt{23} \in \QQ{23}$,
so that $\Norm{\tau} = \frac{26}{23} > 1$, and let
$\theta = x + y \sqrt{23},\ x,y \in \Z$.

If\, $\Norm{\tau - \theta} < \Norm{\tau}$,\, then:
$$
\abs{(1 - x)^2 - 23 \cdot \left(\frac{7}{23} - y\right)^2} <
\abs{1^2 - 23\cdot\left(\frac{7}{23}\right)^2}$$
so that
$$\abs{23(x - 1)^2 - (7 - 23y)^2} < 26$$
Let now\, $N:= 23(x - 1)^2 - (7 - 23y)^2$;\, then
$N \congruent -3 \pmod 23$) and $\abs{N} < 26$, so that
$N$ must then be $-3$\, or\, $20$.

Thus, letting\, $X:= x - 1,\: Y:= 7 - 23y$,\, we have the
possibilities:
%
\begin{enumerate}
%
\item
$N = -3$ \,\ie\,$23X^2 - Y^2 = -3$.

Clearly neither $X$ nor $Y$ can be divisible by $3$, so that
$X^2 \congruent Y^2 \congruent 1 \pmod 3
\then N \congruent 22 \congruent 1 \pmod 3$,
\,a contradiction.
%
\item
$N = 20$ \,\ie\,$23X^2 - Y^2 = 20$.

Clearly neither $X$ nor $Y$ can be divisible by $5$, so that
$X^2 \congruent \pm{1},\,\ Y^2 \congruent \pm{1} \pmod 5$;\,
moreover\, $Y^2 = 23X^2 - 20 \congruent 3X^2 \pmod 5$, and so:
$$
  3 \congruent Y^2\left(X^2\right)^{-1} \congruent
  (\pm{1}) \cdot (\pm{1})^{-1} \congruent \pm{1}\!\!\! \pmod 5,
$$
a contradiction.
%
\end{enumerate}

Thus $\hh{23} = \ZZ{23}$ doesn't satisfy
property~(\ref{property_sharp_1}), so it doesn't satisfy
property~(\ref{property_sharp_weaker}), \ie it isn't
norm-euclidean.
%
\end{proof}

%-----------------------------------------
% Proof thath h(sqrt(14) is not euclidean
%-----------------------------------------
\begin{thm}\label{h(sqrt(14)) not norm-euclidean}
The domain\, $\hh{14} = \ZZ{14}$ isn't norm-euclidean
\end{thm}

\begin{proof}
%
Take
$$ \psi:= \frac{1}{2} + \frac{1}{2} \sqrt{14} \:\in\: \QQ{14}, $$
and suppose that there exists\,
$\theta = x + y\sqrt{14} \in \ZZ{14}$, with $x, y\in\Z$,
\,such that\, $\Norm{\psi - \theta} < 1$.
We get then:
%
\begin{eqnarray*}
1 > \Norm{\psi - \theta} & = &
\Norm{\left(x - \frac{1}{2}\right) +
\left(y - \frac{1}{2}\right)\sqrt{14}} = \\[7pt]
& = & \abs{\left(x - \frac{1}{2}\right)^2
- 14\left(y - \frac{1}{2}\right)^2} =
\abs{\frac{1}{4}\left((2x - 1)^2 - 14(2y - 1)^2\right)}
\end{eqnarray*}
%
so that, letting\, $X:= 2x - 1$,\, $Y:= 2y - 1$\,
and\, $U:= X^2 - 14Y^2$, we have:
\begin{itemize}

\item $\abs{U} < 4$

\item $U \congruent 1 - 14 \congruent -13 \congruent 3 \pmod 8$\\
(since $X,Y$ are odd and\, $z^2 \congruent 1$
(mod $8$)\, $\forall\: z \in \Z$ odd)

\end{itemize}

Then:
$$
U = 3 \then X^2 - 14Y^2 = 3 \then
X^2 \congruent 3\ (\mathrm{mod}\ 7)
$$
%
which, as can be easily seen by direct calculation, is a contradiction.
\end{proof}

By reasoning in an almost identical way, we obtain:

%-----------------------------------------
% Proof thath h(sqrt(35) is not euclidean
%-----------------------------------------
\begin{thm}\label{h(sqrt(35)) not norm-euclidean}
The domain\, $\hh{35} = \ZZ{35}$ isn't norm-euclidean
\end{thm}

\begin{proof}

Take
$$ \psi:= \frac{1}{2} + \frac{1}{2} \sqrt{35} \:\in\: \QQ{35}, $$
and suppose that there exists\,
$\theta = x + y\sqrt{35} \in \ZZ{35}$, with $x, y\in\Z$,
\,such that\, $\Norm{\psi - \theta} < 1$.
We get then:
%
\begin{eqnarray*}
1 > \Norm{\psi - \theta} & = &
\Norm{\left(x - \frac{1}{2}\right) +
\left(y - \frac{1}{2}\right)\sqrt{35}} = \\[7pt]
& = & \abs{\left(x - \frac{1}{2}\right)^2
- 35\left(y - \frac{1}{2}\right)^2} =
\abs{\frac{1}{4}\left((2x - 1)^2 - 35(2y - 1)^2\right)}
\end{eqnarray*}
%
so that, letting\, $X:= 2x - 1$,\, $Y:= 2y - 1$\,
and\, $U:= X^2 - 35Y^2$, we have:
\begin{itemize}

\item $\abs{U} < 4$

\item $U \congruent 1 - 35 \congruent -2 \pmod 8$\\
(since $X,Y$ are odd and\, $z^2 \congruent 1$
(mod $8$)\, $\forall\: z \in \Z$ odd)

\end{itemize}

Then:
$$
U = -2 \then X^2 - 35Y^2 = -2 \then
X^2 \congruent 5\ (\mathrm{mod}\ 7)
$$
%
which, as can be easily seen by direct calculation, is a contradiction.
\end{proof}

\smallskip
Note that the main interest of the
theorems~(\ref{h(sqrt(23)) not norm-euclidean})
and~(\ref{h(sqrt(14)) not norm-euclidean})
depends on the fact that both $\hh{14}$ and $\hh{23}$ are UFDs%
\footnote{this statement will not be proved here.}
(as one can deduce from table~(10.4), chapter~10\linebreak of~\cite{S&T}).

\bigskip
Now we are going to prove a more general and systematic result.
But first we need three technical yet simple lemmas.

%-----------------------------%
% TWO RATHER TECHNICAL LEMMAS %
%-----------------------------%

%---------%
%  TECH1  %
%---------%
\begin{lem}\label{tech1}
If\, $t, m, \mu \in \Z$\, satisfy\, $t \congruent 1 \pmod 2$
\,and\, $m \congruent 2 \pmod 4$ \,and\, $\mu = 2$ \,or\, $3$,\,
then\, $\nexists\: X, Y \in \Z$\, such that
\begin{equation}\label{tech1_eq}
X^2 - mY^2 \congruent t^2 - \mu m\ \mathrm{(mod\ 8)}
\end{equation}
\end{lem}

\begin{proof}
%
If (\ref{tech1_eq}) holds, we have:
\begin{eqnarray*}
& & X^2 - t^2 \congruent m(Y^2 - \mu)\ \mathrm{(mod\ 8)}
\then X^2 - t^2 \congruent 0\ \mathrm{(mod\ 2)}
\ \text{(as $m$ is even)} \then
\\[4pt]
& & \Longrightarrow\; X,\,t\ \textrm{odd}\, \then
X^2 - t^2 \congruent 1 - 1 \congruent 0\ \mathrm{(mod\ 8)},
\end{eqnarray*}
so that, writing $m = 2M$ (with $M$ odd integer), we get:
$$
0 \congruent 2M(Y^2 - \mu)\ \mathrm{(mod\ 8)} \then
(Y^2  - \mu)M \congruent 0\ \mathrm{(mod\ 4)} \then
Y^2 \congruent \mu\ \mathrm{(mod\ 4)},
$$
which is a contradiction since $\mu = 2$ or $3$.
%
\end{proof}

%---------%
%  TECH2  %
%---------%
\begin{lem}\label{tech2}
If\, $t, m, \mu \in \Z$\, satisfy, $t \congruent 1 \pmod 2$
\,and\, $m \congruent 3 \pmod 4$ \,and\, $\mu = 5$ \,or\, $6$,
then\, $\nexists\: X, Y \in \Z$\, such that
\begin{equation}\label{tech2_eq}
X^2 - mY^2 \congruent t^2 - \mu m\ \mathrm{(mod\ 8)}
\end{equation}
\end{lem}

\begin{proof}
%
If (\ref{tech2_eq}) holds, we have that\,
$m(Y^2 - \mu) \congruent X^2 - t^2\ \mathrm{(mod\ 8)}$,\,
and so, being\, $t^2 \congruent m^2 \congruent 1 \pmod 8$
(as $m$ and $t$ are odd):
$$
Y^2 - \mu \congruent m^2(Y^2 - \mu) \congruent
m(X^2 - t^2) \congruent m(X^2 - 1)\ \mathrm{(mod\ 8)}
$$
\ie
$$
Y^2 \congruent \mu + m(X^2 - 1)\ \mathrm{(mod\ 8)}.
$$
But, since\, $m \congruent 3$ or $7$ (mod $8$)\, and\,
$X^2 \congruent 0,\, 1$ or $4$ (mod $8$),\, it must be\,
$m(X^2 - 1) \congruent 0,\, 1$ or $5$ (mod $8$),\, so
that, since\, $\mu = 5$ or $6$:
$$
Y^2 \congruent \mu + m(X^2 - 1) \congruent
2,\, 3,\, 5,\, 6\ \textrm{or}\ 7\ \mathrm{(mod\ 8)},
$$
a contradiction.
%
\end{proof}

%---------%
%  TECH3  %
%---------%
\begin{lem}\label{tech3}
Let\, $m$, $A$, $U$, $t$\, be integers with\, $m > 1$,\, $\abs{U} < m$,\,
$U \congruent t \pmod m$\, and\, $Am < t < (A + 1) m$;\, then\,
$U = t - Am$ \,or\, $U = t - (A+1)m$.
\end{lem}

\begin{proof}
%
Define:
$$U_1:= t - Am\quad\ \text{and}\quad\ U_2:= t - (A+1)m$$
From our hypotheses, we immediately get:
$$
  -m < U_2 < 0 < U_1 < m \quad\ \text{and} \quad
  \ U_1 \congruent U_2 \congruent t\!\!\!\pmod m
$$
so that, being\, $U \congruent t \pmod m$\, and \,$\abs{U} < m$,\, it's easy
to deduce that\, $U = U_1$\, or \,$U = U_2$,\, \ie\, $U = t - Am$\,
or \,$U = t - (A + 1)m$.
%
\end{proof}

%
Now we can state and prove our theorem, which is a generalization of
theorem~(249) in chapter~XIV of~\cite{H&W} (\,as will become clear
after the statement of theorem~(\ref{limits_for_norm-euclidean})\,).

%---------------------------------------------------
% A GENERAL THEOREM ASSURING THAT h(sqrt(m)) IS NOT
% EUCLIDEAN UNDER SUITABLE CONDITIONS, IN THE CASES
% m = 2 or 3 (mod 4)
%---------------------------------------------------
\begin{thm}\label{BIG}
Suppose that $m > 1$ is a squarefree integer such that:
\begin{itemize}
%
\item[{\rm\textsf{(a)\:}}]
$m \congruent 2\ \mathrm{(mod\ 4)}$\, and\,
$\exists\: t \in \N\ \textrm{odd},\ \exists\: k \in \N$\,
such that\, $(4k + 2)m < t^2 < (4k + 3)m$,
%
\end{itemize}
or:
\begin{itemize}
%
\item[{\rm\textsf{(b)\:}}]
$m \congruent 3\ \mathrm{(mod\ 4)}$\, and\,
$\exists\: t \in \N\ \textrm{odd},\ \exists\: k \in \N$\,
such that\, $(8k + 5)m < t^2 < (8k + 6)m$.
%
\end{itemize}
%
Then $\hh{m} = \ZZ{m}$ can't be norm-euclidean.
\end{thm}

\begin{proof}
%
Argue by contradiction, assuming that $\hh{m} = \ZZ{m}$ is
norm-euclidean. Then, defining\,
\mbox{$\psi:=\frac{t}{m}\sqrt{m} \in \QQ{m}$,\,}
we can say that\, $\exists\; \theta \in
\hh{m}$\, such that\, $\Norm{\psi - \theta} < 1$,\, \ie
writing\, $\theta = x + y\sqrt{m}$\, for suitable\,
$x, y \in \Z$:
%
\begin{equation*}
\abs{ (t - my)^2 - mx^2 } < m
\end{equation*}
%
Write now\, $U:= (t - my)^2 - mx^2$;\, we immediately
deduce that:
\begin{equation*}
\abs{U} < m \quad\ \text{and} \quad\ U \congruent
t^2 \!\!\!\pmod m
\end{equation*}
Assuming now that there exists $A \in \N$ such that\,
$Am < t^2 < (A + 1)m$,\, from lemma~(\ref{tech3}) we get:
\begin{equation}\label{BIGeq}
U = t^2 - Am \quad\ \text{or} \quad\ U = t^2 - (A + 1)m
\end{equation}
Now we must distinguish two cases.

\medskip
\textbf{Case 1\::}\: $m \congruent 2 \pmod 4$.

Then we have from hypothesis {\rm\textsf{(a)}} that
$A = 4k + 2$, so that from (\ref{BIGeq}) the two following
possibilities follow\footnote{\,recall that $m$
is even, so that $4km \congruent 0$ (mod $8$).}:

\begin{itemize}

\item $(t - my)^2 - mx^2 = U = t^2 - 2m - 4km \congruent
t^2 - 2m$ (mod $8$)

\item $(t - my)^2 - mx^2 = U = t^2 - 3m - 4km \congruent
t^2 - 3m$ (mod $8$)

\end{itemize}
which, since $t$ is odd, give us a contradiction in
virtue of lemma~(\ref{tech1}).

\medskip
\textbf{Case 2\::}\: $m \congruent 3 \pmod 4$.

Then we have from hypothesis {\rm\textsf{(b)}} that
$A = 8k + 5$, so that from (\ref{BIGeq}) the
two following possibilities follow:

\begin{itemize}

\item $(t - my)^2 - mx^2 = U = t^2 - 5m - 8km \congruent
t^2 - 5m$ (mod $8$)

\item $(t - my)^2 - mx^2 = U = t^2 - 6m - 8km \congruent
t^2 - 6m$ (mod $8$)

\end{itemize}
which, since $t$ is odd, give us a contradiction in
virtue of lemma~(\ref{tech2}).
%
\end{proof}


\medskip
Theorem~(\ref{BIG}) can immediately be used to prove the following
strong and interesting result:

%-----------------------------------------------------
% SEVERE LIMITATIONS TO THE POSSIBILITY FOR h(sqrt(m))
% TO BE NORM-EUCLIDEAN; MORE PRECISLY:
%
% @ if  m = 2 (mod 4) and m >= 42,  then h(sqrt(m))
%   isn't  norm-euclidean
%
% @ if  m = 3 (mod 4) and m >= 91,  then h(sqrt(m))
%   isn't  norm-euclidean
%-----------------------------------------------------
\begin{thm}\label{limits_for_norm-euclidean}
If $m > 1$ is a squarefree integer such that:

~~$\bullet$
$m \congruent 2\ \mathrm{(mod\ 4)}$\, and\, $m \geq 42$, or:

~~$\bullet$
$m \congruent 3\ \mathrm{(mod\ 4)}$\, and\, $m \geq 91$,\\[1.5pt]
%
then\, $\hh{m}$ isn't norm-euclidean.
\end{thm}

\begin{proof}
%
In virtue of theorem~(\ref{BIG}), to prove our theorem
it's enough to prove that:

\begin{itemize}

\item
$\forall\: m \geq 42:\ m \congruent 2\ (\mathrm{mod}\ 4),
\,\ \exists\: t \in \Z$ odd\, such that\, $2m < t^2 <3m$

\item
$\forall\: m \geq 91:\ m \congruent 3\ (\mathrm{mod}\ 4),
\,\ \exists\: t \in \Z$ odd\, such that\, $5m < t^2 < 6m$

\end{itemize}
%
We can distingush two cases.

%------------------------------------
% FIRST CASE IN THE PROOF OF THEOREM
% \ref{limits_for_norm-euclidean}
%------------------------------------
\medskip
\textbf{Case 1. } $m \congruent 2$ (mod $4$)\, and\,
$m \geq 42$.

\smallskip
It results:
$$
m \geq 42 > 40 = 4 \cdot (5 + 5) > 4 \cdot
\left(5 + 2\sqrt{6}\right) = 4 \cdot \left(\sqrt{3} +
\sqrt{2}\right)^2
$$
so that\, $\sqrt{m} > 2(\sqrt{3} + \sqrt{2}) \then
\sqrt{m}(\sqrt{3} - \sqrt{2}) > 2$,\, and then:
$$
\left(\frac{\sqrt{3m} - 1}{2}\right) -
\left(\frac{\sqrt{2m} - 1}{2}\right) =
\frac{\sqrt{3m} - \sqrt{2m}}{2} = 
\frac{1}{2}\sqrt{m} \left(\sqrt{3}-\sqrt{2}\right) > 1.
$$
%
Thus, putting
$\beta:= \frac{1}{2} \left(\sqrt{3m} - 1\right)$
and
$\alpha:= \frac{1}{2} \left(\sqrt{2m} - 1\right)$,
and using lemma~(\ref{intseplemma}), we obtain:
\begin{eqnarray*}
& \beta - \alpha > 1 & \then \beta > \alpha + 1 \then
\exists\: u \in \N\ \textrm{such that}\ \alpha < u <
\beta
\\[5pt]
& & \then \sqrt{2m} = 2 \alpha + 1 < 2u + 1 <
2 \beta + 1 = \sqrt{3m}.
\end{eqnarray*}
Finally, writing\, $t:= 2u + 1$,\, we have $t \in \N$ odd and
\,$2m < t^2 < 3m$,\, as desired.


%------------------------------------
% SECOND CASE IN THE PROOF OF THEOREM
% \ref{limits_for_norm-euclidean}
%------------------------------------
\medskip
\textbf{Case 2. } $m \congruent 3$ (mod $4$)\, and\,
$m \geq 91$.

\smallskip
It results:
$$
m \geq 91 > 88 = 4 \cdot (11 + 11) > 4 \cdot
\left(11 + 2\sqrt{30}\right) = 4 \cdot \left(\sqrt{6} +
\sqrt{5}\right)^2
$$
so that\, $\sqrt{m} > 2(\sqrt{6} + \sqrt{5}) \then
\sqrt{m}(\sqrt{6} - \sqrt{5}) > 2$,\, and then:
$$
\left(\frac{\sqrt{6m} - 1}{2}\right) -
\left(\frac{\sqrt{5m} - 1}{2}\right) =
\frac{\sqrt{6m} - \sqrt{5m}}{2} = 
\frac{1}{2}\sqrt{m} \left(\sqrt{6}-\sqrt{5}\right) > 1.
$$
%
Thus, putting
$\beta:= \frac{1}{2} \left(\sqrt{6m} - 1\right)$
and
$\alpha:= \frac{1}{2} \left(\sqrt{5m} - 1\right)$,
and using lemma~(\ref{intseplemma}), we obtain:
\begin{eqnarray*}
& \beta - \alpha > 1 & \then \beta > \alpha + 1 \then
\exists\: u \in \N\ \textrm{such that}\ \alpha < u <
\beta 
\\[5pt]
& & \then \sqrt{5m} = 2 \alpha + 1 < 2u + 1 <
2 \beta + 1 = \sqrt{6m}.
\end{eqnarray*}
Finally, writing\, $t:= 2u + 1$,\, we have $t \in \N$ odd
and\, $5m < t^2 < 6m$,\, as desired.
%
\end{proof}

\smallskip
The previous theorem gives a severe limitation to the
possibility for a given domain $\hh{m}$\,
(with $m \congruent 2$ or $3$ (mod $4$)\,) of being
norm-euclidean. Moreover, the result can be reinforced
again, leading to the following theorem:
%----------------------------------------------------------------
% NECESSARY CONDITION THAT h(sqrt(m)) ( WITH m = 2 or 3 (mod 4) )
% MUST SATISFIES TO BE NORM-EUCLIDEAN
%----------------------------------------------------------------
\begin{thm}%
\label{first_partial_classification_of_norm-euclidean_domains}
If $m \congruent 2 \text{ or } 3 \pmod 4$ is a squarefree
positive integer such that the domain $\hh{m}$ is norm-euclidean,
then it must be:
$$ m\, = \,2,\, 3,\, 6,\, 7,\, 11\ \text{or}\ 19 $$
\end{thm}

\begin{proof}
Of course, if\, $m \congruent 2$ (mod 4) and $m \geq 42$,\,
or if\, $m \congruent 3$ (mod 4) and $m \geq 91$,\, then
$\hh{m}$ isn't norm-euclidean in virtue of
the previous theorem~(\ref{limits_for_norm-euclidean}).
Thus we have only to show that $\hh{m}$ is not norm-euclidean
in the finitely many cases:

~~\textbf{1)}\:
$m \congruent 2 \pmod 4$,\:
$2 \leq m < 42$,\:
$m$ squarefree,\:
$m \neq 2$\, and\, $m \neq 6$

~~\textbf{2)}\:
$m \congruent 3 \pmod 4$,\:
$3 \leq m < 91$,\:
$m$ squarefree,\:
$m \neq 3$,\, $m \neq 7$,\,
$m \neq 11$,\, and \,$m \neq 19$

For sake of simplicity, we'll break down our analysis
into two parts, separating the cases
$m \congruent 2 \pmod 4$ and $m \congruent 3 \pmod 4$.

\medskip
\textbf{Part 1.\:} $m \congruent 2 \pmod 4$.

\smallskip

Then $m$ has one of the following values:
\begin{displaymath}
m = 10,\ 14,\ 22,\ 26,\ 30,\ 34 \text{ or } 38
\end{displaymath}

For each such $m$, we must show that $\hh{m}$ is not norm-euclidean.

\medskip
This is done in the following calculations.

\input{calcgen1.tex}

\medskip
\textbf{Part 2.\:} $m \congruent 3 \pmod 4$.

\smallskip

Since by hypotesis Then $m$ has one of the following values:
\begin{displaymath}
m = 15,\ 19,\ 23,\ 31,\ 35,\ 39,\ 43,\ 47,\ 51,\ 55,\ 59,\ 67,\ %
    71,\ 79,\ 83, \text{ or } 87
\end{displaymath}

For each such $m$, we must show that $\hh{m}$ is not norm-euclidean.

\medskip
This is done in the following calculations.

\input{calcgen2.tex}

All the possible cases have been analyzed, and the
theorem is proved.
%
\end{proof}

\end{subsection}

\smallskip
%------------------------------------------------------
% non norm-euclidean h(sqrt(m)) for m = 1 (mod 4)
%------------------------------------------------------
\begin{subsection}{Non norm-euclidean $\hh{m}$ for $m\congruent 1 \pmod 4$}

Let's start seeing a domain of the form $\hh{m}$, with
$m \congruent 1 \pmod 4$, which is not norm-euclidean.

%----------------------------------------------
% A PROOF THAT h(sqrt(53)) ISN'T NORM-EUCLIDEAN
%----------------------------------------------
\begin{thm}\label{h(sqrt(53))_not_norm-euclidean}
The domain $\hh{53}$ isn't norm-euclidean.
\end{thm}

\begin{proof}
%
Argue by contradiction, assuming that $
\hh{53} = \left\{ \frac{2x + y}{2} +
\frac{y}{2}\sqrt{53}:\ x,y \in \Z \right\}
$ is norm-euclidean, \ie that:
\begin{equation}\label{h(sqrt(53))_1}
\forall\, r,s \in \Q\quad \exists\: x,y \in \Z :\quad
\ \abs{ \left(\frac{2x + y}{2} -r \right)^2
- 53\left(\frac{y}{2} - s\right)^2 } < 1
\end{equation}
Let now $r:= 0$ and $s:= \frac{12}{53}$. Then
(\ref{h(sqrt(53))_1}) becomes:
\begin{equation*}
\abs{ \frac{\left(2x + y\right)^2}{4}
- \frac{53}{4}\left(y - \frac{24}{53}\right)^2 } < 1
\end{equation*}
\ie
\begin{equation}\label{h(sqrt(53))_2}
\abs {
\left(53 y - 24\right)^2 - 53 \left(2x + y\right)^2
} < 212
\end{equation}
%
Write now \,$X:= 53y - 24$,\, $Y:=2x + y$,\,
$U:=X^2 - 53Y^2$.\, We have then\,
$X^2 \congruent 24^2 \congruent$ \mbox{$46$ (mod $53$)}
$\then U \congruent 46$ (mod $53$),\, and\,
$X \congruent Y \congruent$ \mbox{$y$ (mod $2$)} $\then
U = X^2 - 53Y^2 \congruent X^2 - Y^2 \congruent 0$ (mod 4).
Thus, using also (\ref{h(sqrt(53))_2}), we get:

\begin{itemize}

\item $\;U\congruent 46$ (mod 53)\, and\, $U\congruent 0$
(mod 4) $\then U \congruent 152$ (mod 212).

\item $\;\abs{U} < 212\,\then\,U = 152$\, or\, $U = -60$.

\end{itemize}
%
We now will show that both this cases are impossible.
%
\begin{itemize}

\item[\textbf{1. }] $U = -60 = -4 \cdot 3 \cdot 5$.\\
Then\, $9 \notdivides U = X^2 - 53Y^2 \then 
3 \notdivides X$\, and\, $3 \notdivides Y \then X^2
\congruent Y^2 \congruent 1$ (mod $3$) $\then 0
\congruent -60 = U = X^2 - 53Y^2 \congruent 1 - 53 \cdot
1 \congruent -52 \congruent 2$ (mod $3$),\, a
contradiction.

\item[\textbf{2. }] $U = 152 = 8 \cdot 19$.\\
Then $X^2 - 53 Y^2 = U \congruent 0$ (mod $8$). Obiouvsly
is\, $X \congruent Y$ (mod $2$);\, if it were\,
$X \congruent Y \congruent 1$ (mod $2$), we would get:
$$
0 \congruent X^2 - 53 Y^2 \congruent 1 - 53 = -52
\congruent 4\ \mathrm{(mod\ 8)},
$$
a contradiction.
Thus it must be\, $X = 2a,\ Y = 2b$\, for suitable $a, b
\in \Z \then a^2 - 53b^2 = \frac{U}{4} = 38 \then a
\congruent b$ (mod $2$).\, But then:
$$
a^2 \congruent b^2\ \mathrm{(mod\ 4)} \then
38 = a^2 - 53b^2 \congruent a^2 - b^2 \congruent
0\ \mathrm{(mod\ 4)},
$$
\end{itemize}
%
which gives us the final contradiction.
%
\end{proof}

Now were going to prove a generalization of the previous result, which
will enable us to build an infinite class of domains $\hh{m}$ with
$m \congruent 1 \pmod 4$ which are all not norm-euclidean.

Here is our result\footnote{note that this result is unfortunately much
weaker than those obtained for $m \congruent 2$ or $3 \pmod 4$.}:

\begin{thm}\label{SMALL}
Let \,$m > 1$\, be a squarefree integer with \,$m \congruent 5 \pmod{24}$,
and assume that there exist $A,\, t \in \N$ such that\,
$A \congruent 6,~11,~23~\text{or}~30 \pmod{36}$, $t \congruent 3 \pmod 6$
\,and\, $Am < t^2 < (A+1)m$.
Then $\hh{m}$ isn't norm-euclidean.
\end{thm}

\begin{proof}
Argue by contradiction, assuming that 
$$\hh{m} = 
  \left\{
    \frac{2x + y}{2} + \frac{y}{2}\sqrt{m}:\:x,\,y \in \Z
  \right\}$$
is norm-euclidean.
Then, defining \,$\psi := \frac{t}{m}\sqrt{m} \in \QQ{m}$, we can say that
\,$\exists\;\theta \in \hh{m}$\, such that \,$\Norm{\psi - \theta} < 1$,\,
\ie writing \,$\theta = \frac{2x + y}{2} + \frac{y}{2}\sqrt{m}$\,
for suitable $x ,\, y \in \Z$:
$$ \abs{\frac{(2x+y)^2}{4} - \frac{(my-2t)^2}{4m^2} \cdot m} < 1 $$
or better:
\begin{equation}\label{t-ineq}
\abs{ {(2t - my)}^2 - m {(2x + y)}^2 } < 4m.
\end{equation}
Writing now:
\begin{equation}\label{U-def}
U := {(2t - my)}^2 - m {(2x + y)}^2
\end{equation}
we immediately deduce that \,$U \congruent 4t^2 \pmod m$\, and
\,$U \congruent m^2 y^2 - m y^2 = m (m - 1) y^2 \congruent 0
\congruent 4t^2 \pmod 4$\, (\,as $4 \divides (m - 1)$\,), so that,
using also inequality~(\ref{t-ineq}) and our hypothesis on $A$\,
we get:
$$
  U \congruent 4t^2\ (\mathrm{mod}\ 4m)%
  ~~~\text{and}~~~%
  \abs{U} < 4m%
  ~~~\text{and}~~~%
  A \cdot (4m) < 4t^2 < (A+1) \cdot (4m).
$$
At this point we can apply lemma~(\ref{tech3}) to deduce that\,
$U = 4t^2 - A\cdot(4m)$\, or\, $U = 4t^2 - (A+1)\cdot(4m)$,\, \ie
that:
\begin{equation}\label{U-values}
U = 4\left(t^2 - A m\right)
~~\text{or}~~\,
U = 4\left(t^2 - (A+1) m\right)
\end{equation}
Now, since $A \congruent 6,~11,~23~\text{or}~30 \pmod{36}$,
from~(\ref{U-values}) easily derives%
\footnote{it's enough to set $B = A$ and $C = A + 1$ if
$A \congruent 11 \text{ or } 23 \pmod{36}$,\, $B = A + 1$ and
$C = A$ if $A \congruent 6 \text{ or } 30 \pmod{36}$.}
that must exist $B, C \in \N$ with $B \congruent 3 \pmod 4$ and
$C \congruent 3 \text{ or } 6 \pmod 9$ such that $U = 4(t^2 - B m)$
or $U = 4(t^2 - C m)$;\,
moreover, from~(\ref{U-def}) easily derives that\,
$U = X^2 - m Y^2$\, with\, $X,\,Y\in\Z$,\, $X \congruent Y \pmod 2$.

Thus, summarizing what has been obtained so far, we can state that
there exist\, $X$, $Y$, $B$, $C$, $t$, $m$ $\in \Z$ such that:
\begin{equation}\label{XYBCtm-congruences}
\left\{
  \begin{array}{rcl}
    X & \congruent & Y~(\mathrm{mod}\ 2)  \\
    m & \congruent & 5~\;(\mathrm{mod}\ 24) \\
    t & \congruent & 3~\;(\mathrm{mod}\ 6)  \\
    B & \congruent & 3~\;(\mathrm{mod}\ 4)  \\
    C & \congruent & 3\text{ or }6~\:(\mathrm{mod}\ 9)
  \end{array}
\right.
\end{equation}
and:
\begin{equation}\label{XYBCtm-equalities}
X^2 - m Y^2 = 4(t^2 - B m)
~~\,\text{or}\,~~
X^2 - m Y^2 = 4(t^2 - C m)
\end{equation}
Now we are going to show that, with the~(\ref{XYBCtm-congruences})
holding, both the~(\ref{XYBCtm-equalities}) are impossible.

To this purpose, we'll divide the rest of our proof into two cases.

\medskip
\textbf{Case 1.\:} Assume first that:
\begin{equation}\label{eq-B}
X^2 - m Y^2 = 4(t^2 - B m).
\end{equation}
\smallskip
If\, $X \congruent Y \congruent 1 \pmod2$, it follows that\,
\begin{math}
  X^2 - m Y^2 \congruent 1 - m \congruent 4\ (\mathrm{mod}\ 8) \then
  8 \notdivides (X^2 - m Y^2) = 4 (t^2 - B m) \then
  2 \notdivides (t^2 - B m)
\end{math}, which is a contradiction since $t$, $B$ and $m$ are all odd.

Thus it must be\, $X \congruent Y \congruent 0 \pmod2$, and writing\,
$X = 2W$, $Y = 2Z$\, for suitable \,$W,\:Z\in \Z$\, we obtain:
$$
   4\left(W^2 - m Z^2\right) = 4\left(t^2 - B m\right) \then
   W^2 - m Z^2 = t^2 - B m.
$$
Now\, $t^2 - Bm \congruent 1 - 3 \cdot 5 \congruent 2 \pmod 4$,\,
so that $2 \divides \left(W^2 -m Z^2\right)$ \,and thus\,
$W \congruent Z \pmod 2$;\, but then:
$$
   2 \congruent t^2 - B m \congruent W^2 - Z^2 \congruent
   0\ (\mathrm{mod}\ 4)
$$
again a contradiction.

\medskip
\textbf{Case 2.\:} Assume now that
\begin{equation}\label{eq-C}
X^2 - m Y^2 = 4(t^2 - C m).
\end{equation}

\smallskip
Since\, $3 \divides t$\, and\, $C \congruent 3$ or $6 \pmod 9$,\,
equality~(\ref{eq-C}) can be rewritten as:
\begin{equation}\label{eq-C-1}
X^2 - m Y^2 = 4(9r - 3sm)
\end{equation}
for suitable $r \in \Z$ \,and $s \in \left\{1, 2\right\}$.

As \,$3 \notdivides s$\, and \,$3 \notdivides m$,\,
from~(\ref{eq-C-1}) we get $X,\, Y \notcongruent 0 \pmod 3$,
thus\, $X^2 \congruent Y^2 \congruent 1 \pmod 3$.
But then, since $m \congruent 2 \pmod 3$:
$$
  0 \congruent 4\left(9r - 3ms\right) = X^2 - m Y^2 \congruent 1 - 2 
  \congruent -1\ (\mathrm{mod}\ 3),
$$
which is plainly impossible.

\medskip
In both cases we reached a contradiction, and the result follows.
\end{proof}

As stated previously, the result just proven can be used to build an
infinite class of domains $\hh{m}$ with $m \congruent 1 \pmod 4$ which
are all not norm-euclidean.

\begin{cor}\label{big m = 5 mod 24 not norm-euclidean}
If\, $m \congruent 5 \pmod{24}$ and $m \geq 941\ (= 39 \cdot 24 + 5)$,
then $\hh{m}$ isn't norm-euclidean.
\end{cor}
%
\begin{proof}
We will show that for every $m \geq 941$ there exists an $u \in \N$ such
that\, $6m < {(6u + 3)}^2 < 7m$, from which our claim will follow
immediately in virtue of the previous theorem~(\ref{SMALL}).

First, it's easy to see that $941 > 36(13 + 2 \sqrt{42})$, so that
if $m \geq 941$ it's also
$$
  m > 36\left(13 + 2 \sqrt{42}\,\right) =
  6^2 {\left(\sqrt{7} + \sqrt{6}\,\right)}^2
$$
and then:
$$
    \sqrt{m} > 6\left(\sqrt{7} + \sqrt{6}\,\right) =
    \frac{6}{\sqrt{7} - \sqrt{6}\phantom{\,}}
  \ \then \ %
    \sqrt{7m} - \sqrt{6m} > 6
$$
or equivalently:
$$ \frac{\sqrt{6m} - 3}{6} + 1 < \frac{\sqrt{7m} - 3}{6} $$
From this last inequality and from lemma~(\ref{intseplemma}) we deduce
that $\exists\: u \in \N$ such that:
$$\frac{\sqrt{6m} - 3}{6} < u < \frac{\sqrt{7m} - 3}{6}$$
\ie such that:
$$6m < {(6u + 3)}^2 < 7m$$
and our claim follows.
\end{proof}

We conclude by an elegant result which reinforce the previous
corollary~(\ref{big m = 5 mod 24 not norm-euclidean}):

\begin{thm}%
\label{second_partial_classification_of_norm-euclidean_domains}
Let $m > 1$ squarefree such that\, $m \congruent 5 \pmod{24}$;
then the domain $\hh{m}$ is norm-euclidean if and only if\,
$m = 5$\, or\, $m = 29$.
\end{thm}

\begin{proof}
It was already shown in theorem~(\ref{normed_UFDs_3}) that $\hh{5}$
and $\hh{29}$ are norm-euclidean, and in the previous
corollary~(\ref{big m = 5 mod 24 not norm-euclidean}) that $\hh{m}$
isn't norm-euclidean for any $m$ squarefree such that $m \geq 941$
and $m \congruent 5 \pmod{24}$; thus, to prove our claim, we only
need to show that $\hh{m}$ isn't norm-euclidean for all $m$ squarefree
with $m \congruent 5 \pmod{24}$ and $53 \leq m < 941$,\, \ie for:
\begin{displaymath}
\begin{array}{lll}
m & = &
53,\phantom{1}\ 77,\phantom{1}\ 101,\ 149,\ 173,\ 197,\ 221,\ %
269,\ 293,\ 317,\ 341,\ 365,\ 389,\ 413,\ 437,\ \\[1pt]
& &
461,\ 485,\ 509,\ 533,\ 557,\ 581,\ 629,\ 653,\ 677,\ 701,\ 749,\ %
773,\ 797,\ 821,\ 869,\ \\[1pt]
& & 893,\text{ or \:}917.
\end{array}
\end{displaymath}

This easily follows from theorem~(\ref{SMALL}) and the following
calculations.

%% DIRTY HACK TO HAVE A RIGHT SPACING AND PAGE DIVISION
\vspace{-3pt}
\input{calcgen3.tex}

All the possible cases have been analyzed, and the
theorem is proved.

\end{proof}

\end{subsection}
