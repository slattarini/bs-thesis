% -*- LaTeX -*-

%==========================================================================

%%--------------------------------%%
%% uncomment this when debugging  %%
%\def\bmdebug{1}                  %%
%%--------------------------------%%

\ifdefined\bmdebug
  \documentclass[italian,draft]{beamer}
\else
  \documentclass[italian]{beamer}
\fi

%==========================================================================

\usepackage[italian]{babel}
\mode<presentation>{\usetheme{Warsaw}}

\IfFileExists{hownotes.tex}{\input{hownotes.tex}}{}

\ifdefined\onlynotes % produce notes' sheets
  \usepackage{pgfpages}
  \pgfpagesuselayout{4 on 1}[a4paper,landscape,border shrink=5mm]
  \setbeameroption{show only notes}
  \newcommand{\bmlabel}[1]{}
  \newcommand{\bmref}[1]{\textbf{??}}
\else % produce slides
  % workaround for a bug (maybe) in my beamer package
  \catcode`\@ = 11
  \providecommand{\beamer@notes}[0]{}
  \providecommand{\beamer@noteitems}[0]{}
  \catcode`\@ = 12
  \setbeameroption{hide notes}
  \ifdefined\bmdebug
    % labels of slides to be compiled when debugging
    \includeonlyframes{d0,d1,d2,d3,d4,d5,d6,d7,d8,d9} % change at pleasure
    \newcommand{\bmlabel}[1]{}
    \newcommand{\bmref}[1]{\textbf{??}}
  \else
    \newcommand{\bmlabel}[1]{\label{#1}}
    \newcommand{\bmref}[1]{\ref{#1}}
  \fi
\fi

\AtBeginSubsection[]
{
  \begin{frame}<beamer>{Sommario: sezione \thesection.\thesubsection}
    \tableofcontents[currentsection,currentsubsection]
  \end{frame}
}

%==========================================================================

\usepackage{amsmath,amsfonts,amsthm,amssymb}
\usepackage[latin1]{inputenc}
\usepackage{times}

%==========================================================================

% Extend theorem-like environments.
\IfFileExists{translator.sty}{
  \usepackage{translator}
  \usedictionary{translator-theorem-dictionary}
}{}
\providecommand\translate[2][]{#2}

\newcommand{\G}[0]{\mathbb{G}}
\newcommand{\eps}[0]{\epsilon}
\newcommand{\pp}[1]{\text{propriet\`a~[#1]}}
\newcommand{\PP}[1]{\text{Propriet\`a~[#1]}}
\theoremstyle{example}
\newtheorem{facts}[theorem]{\translate{Facts}{}}
\theoremstyle{plain}

% Personal LaTeX commands, shared with tesi.tex
% "minus zero"
\newcommand{\minz}{\setminus\left\{0\right\}}

% natural numbers
\newcommand{\N}{\mathbb{N}}
\newcommand{\Nzero}{\N\minz}
% integers
\newcommand{\Z}{\mathbb{Z}}
\newcommand{\Zzero}{\Z\minz}
% rational numbers
\newcommand{\Q}{\mathbb{Q}}
\newcommand{\Qzero}{\Q\minz}
% real numbers
\newcommand{\R}{\mathbb{R}}
\newcommand{\Rzero}{\R\minz}
% complex numbers
\newcommand{\C}{\mathbb{C}}
\newcommand{\Czero}{\C\minz}
% set of non-square integers
\newcommand{\NotSquare}{\Z\setminus\{h^2: h\in\Z\}}

%%
%% shortands for recurring normed domains
%%

% A "dirty hack" to give some parenthes the rigth size.
% NOTE: phnatom is a fragile command, so protect it.
\newcommand{\hh}[1]{h\big(\sqrt{\protect\phantom{m}\!\!\!\!\!#1}\,\big)}
\newcommand{\hhm}[1]{h\big(\sqrt{-\protect\phantom{m}\!\!\!\!\!#1}\,\big)}
\newcommand{\QQ}[1]{\Q\big(\sqrt{\protect\phantom{m}\!\!\!\!\!#1}\,\big)}
\newcommand{\QQM}[1]{\Q\big(\sqrt{-\protect\phantom{m}\!\!\!\!\!#1}\,\big)}
\newcommand{\ZZ}[1]{\Z\big[\sqrt{\protect\phantom{m}\!\!\!\!\!#1}\,\big]}
\newcommand{\ZZM}[1]{\Z\big[\sqrt{-\protect\phantom{m}\!\!\!\!\!#1}\,\big]}

% symbol for conjugate (alias of "\overline")
\newcommand{\cg}[1]{\overline{#1}}

% norm operator
\newcommand{\Normop}[0]{\mathcal{N}}
% norm
\newcommand{\Norm}[1]{\Normop\left(#1\right)}
% modulus of a real or complex number
\newcommand{\abs}[1]{\left|#1\right|}

% Legendre symbol
\newcommand{\Leg}[2]{\left(\frac{#1}{#2}\right)}

% congruence relation between integers
\newcommand{\congruent}{\equiv}
\newcommand{\notcongruent}{\not\equiv}

% relation of division in domains
\newcommand{\divides}{\mid}
\newcommand{\notdivides}{\nmid}

% great common divisor and least common multiple
\renewcommand{\gcd}[2]{\mathrm{g.c.d.}\left({#1},\,{#2}\right)}
\newcommand{\mgcd}[1]{\mathrm{g.c.d.}\left({#1}\right)}
\newcommand{\lcm}[2]{\mathrm{l.c.m.}\left({#1},\,{#2}\right)}
\newcommand{\mlcm}[1]{\mathrm{l.c.m.}\left({#1}\right)}

% "id est" with right spacing
\newcommand{\ie}{\text{i.e.\ }}

% "elegant" implication/bimplication symbols
\newcommand{\then}{\:\Longrightarrow\:}
\renewcommand{\iff}{\:\Longleftrightarrow\:}


\usepackage{color}
\definecolor{myblue}{rgb}{0, 0, .45}
\definecolor{mygreen}{rgb}{0, .5, 0}
\colorlet{darkred}{red!60!black}
\colorlet{darkbrown}{brown!60!black}
\xdefinecolor{darkgreen}{rgb}{0,0.35,0}
\xdefinecolor{pale}{rgb}{0.75,0.75,0.75}
\setbeamercolor{dullcolors}{fg=black,bg=pale}

\newenvironment<>{myblock}[3]
    {\setbeamercolor{block title}{bg=#1,fg=#2}\begin{block}#4{#3}}
    {\end{block}}

\newenvironment<>{property}[1]
    {
     \begin{myblock}%
     {darkbrown}{white}{Propriet\`a~[{#1}]}
    }
    {\end{myblock}}

%==========================================================================

\title[Problemi di fattorizzazione per interi algebrici quadratici]%
      {Problemi di fattorizzazione per interi algebrici quadratici}
\subtitle[]{esplorati con metodi elementari}

\author[Lattarini Stefano]{Lattarini Stefano}

\institute{Universit\`a Cattolica del Sacro Cuore -- Sede di Brescia}

\date{29 Aprile 2008}

%##########################################################################

\begin{document}

%##########################################################################

%%-------------------%%
%%  FRAME TITLEPAGE  %%
%%-------------------%%
\begin{frame}
  \titlepage
\end{frame}

%##########################################################################

\section{Introduzione}

%==========================================================================

\subsection{Introduzione al problema}

%--------------------------------------------------------------------------

%%------------%%
%%  FRAME 01  %%
%%------------%%

\begin{frame}[t]{Due Parole Sul Problema}

%** overlay 1 **%
  \begin{itemize}

    \item<1->
      Nel lavoro di tesi abbiamo affrontato, usando metodi elementari, il
      problema di determinare \alert<1>{quando valga\\ l'unicit\`a della
      fattorizzazione} in irriducibili in alcuni\\ \alert<1>{sottoanelli}
      del campo complesso $\C$, aventi per elementi\\
      \alert<1>{interi algebrici quadratici}.

  \note<1>{
      La definizione e la caratterizzazione esatte dei sottoanelli
      da noi studiati richiedono alcuni prerequisiti e definizioni
      preliminari, e saranno quindi date a breve.
  }

%** overlay 2 **%
    \vspace{5pt}
    \item<2->
      Nei sottoanelli da noi studiati, che pure sono tutti delle\\
      estensioni di $\Z$,
      \alert<2>{non \`e affatto ovvio che continui a valere\\
      l'unicit\`a della fattorizzazione} %
      in irriducibili (e in molti casi\\
      questa in effetti non vale pi\`u).
    
  \note<2>{
    \begin{itemize}
      \item
        Si pone quindi il problema interessante di determinari quali
        sottoanelli posseggono un teorema di fattorizzazione unica
        e quali no.
      \item
        Nella sua forma embrionale, questo problema risale a Gauss ed
        Euler, ed ha contribuito molto a dare origine alla teoria
        algebrica dei numeri.
        Esso \`e stato poi studiato estensivamente, portando a risultati
        profondi e generali.
      \item
        I risultati in questo campo sono interessanti sia dal punto di
        vista teorico che nelle loro applicazioni ad altri notevoli
        problemi di teoria dei numeri (ad esempio, la dimostrazione oggi
        standard del caso $n = 3$ dell'Ultimo Teorema di Fermat si basa
        su semplici risultati di teoria algebrica dei numeri).
    \end{itemize}
  } % end \note

%** overlay 3 **%
    \vspace{5pt}
    \item<3->
      Mostreremo ora alcuni esempi che possono dare un'idea\\
      delle problematiche da affrontarsi.

  \end{itemize}

\end{frame}

%--------------------------------------------------------------------------

%%------------%%
%%  FRAME 02  %%
%%------------%%

\begin{frame}[t]{Alcuni Esempi Introduttivi}

 {
  \small
  \vspace{-3pt}
  
  \begin{itemize}
    
%** overlay 1 **%
    \item<1->
      Indichiamo con $\G$ l'anello degli interi di Gauss, ossia quei
      numeri complessi aventi parte reale e immaginaria intere.
    
%** overlay 2 **%
    \item<2->
      Le unit\`a negli interi di Gauss non sono pi\`u solo $+1$ e $-1$,
      come in $\Z$, ma comprendono anche $+i$ e $-i$.

%** overlay 3 **%
    \item<3-> 
      Negli interi di Gauss, $13$ non \`e pi\`u irriducibile:\\[2pt]
      \begin{center}
        $13 = (2 + 3i)\cdot(2 - 3i) = (3 - 2i)\cdot(3 + 2i)$
      \end{center}

%** overlay 4 **%
    \item<4-> 
      Le due precedenti fattorizzazioni di $13$ non sono realmente
      distinte come potrebbe sembrare a prima vista; infatti i fattori
      corrispondenti nelle due fattorizzazioni sono associati:\\[2pt]
      \begin{center}
        $2 + 3i = (3 - 2i)\cdot i$%
          \quad~e~\quad~%
        $2 - 3i = (3 + 2i)\cdot (-i)$
      \end{center}

%** overlay 5 **%
  \item<5-> 
    L'anello gli interi di Gauss possiede, al pari di $\Z$, un teorema di
    fattorizzazione unica, come dimostreremo pi\`u avanti.
  
  \end{itemize}

 }

\end{frame}

%==========================================================================

\subsection{Definizioni e risultati preliminari}

%--------------------------------------------------------------------------

%%------------%%
%%  FRAME 03  %%
%%------------%%

\begin{frame}[t]{UFD: Dominii A Fattorizzazione Unica}

%** overlay 1 **%
  \vspace{2pt}
  \begin{definition}[UFD \,ovvero\, dominii a fattorizzazione unica]
    Con ``\alert<1>{dominio a fattorizzazione unica}'',
    o pi\`u brevemente ``dominio fattoriale'' o ``UFD''
    (\textit{Unique Factorization Domain\/})
    indichiamo un dominio di integrit\`a in cui
    \alert<1>{ogni elemento non nullo pu\`o scriversi \emph{in maniera
    essenzialmente unica\/} come prodotto di un numero finito di
    irriducibili}.
  \end{definition}

%** overlay 2 **%
\pause%
  \vspace{10pt}
  \begin{example}
    Ad esempio, per il teorema fondamentale dell'aritmetica, l'anello
    degli interi \alert<2>{$\Z$ \`e un dominio a fattorizzazione unica}.
  \end{example}

  \note<2>{
    Una precisazione importante: quando parliamo di fattorizzazione in
    irriducibili \emph{essenzialmente unica\/}, intendiamo che essa
    \`e unica solo \emph{a meno di permutazioni dei fattori e di
    moltiplicazione dei fattori stessi per opportuni elementi unitari}.
    \\[7pt]
    %
    Ad esempio, in $\Z$, \,$15$\, ammette le due fattorizzazioni in
    irriducibili:
      $$ 15 = 3 \cdot 5 \text{~~~e~~~} 15 = 5 \cdot 3 $$
    ma queste due fattorizzazioni chiaramente \emph{non\/} sono
    essenzialmente distinte, in quanto ciascuna si ottiene dall'altra
    con una permutazione dei fattori.    

    D'altronde, \,$15$\, ammette anche le due fattorizzazioni:
      $$ 15 = 3 \cdot 5 \text{~~~e~~~} 15 = (-3) \cdot (-5)$$
    che pure \emph{non\/} sono essenzialmente distinte, in quanto
    $-3$ \`e associato a $3$ e $-5$ \`e associato a $5$.
  }

\end{frame}

%--------------------------------------------------------------------------

%%------------%%
%%  FRAME 04  %%
%%------------%%

\begin{frame}{I Campi $\QQ{m}$: Definizione}

%** overlay 1 **%

  \note<1>[item]{
    Dire letteralmente:
    \begin{quote}
      Tutti i dominii che abbiamo considerate nel nostro lavoro sono
      sottoanelli di opportuni campi $\QQ{m}$, che ora definiremo.
    \end{quote}
  }

  \vspace{-6pt}
  \begin{definition}[campi $\QQ{m}$]
    \vspace{3pt}
    Sia dato un intero $m$\, (positivo o negativo) diverso da un
    quadrato perfetto.\, Allora definiamo:
      \vspace{-2pt}
      $$\alert<1>{\QQ{m}:=\left\{ a+b\sqrt{m}:\; a,b\in\Q \right\}}$$
      \vspace*{-12pt}
  \end{definition}

  \note<1>[item]{
    Osservare che elementi come $-1$ e $-4$ non sono quadrati perfetti.
  }

\end{frame}

%--------------------------------------------------------------------------

%%------------%%
%%  FRAME 05  %%
%%------------%%

\begin{frame}{I Campi $\QQ{m}$: Propriet\`a Principali}
  
%** overlay 1 **%
  \begin{facts}
    
    \vspace*{4pt}
    \begin{itemize}
      \item<1->
        $\QQ{m}$ \`e un sottocampo di $\C$, ed in effetti il minimo\\
        sottocampo di $\C$ che contiene $\sqrt{m}$.

%** overlay 2 **%
      \vspace*{2pt}
      \item<2->
        Se $m > 0$, $\QQ{m}$ \`e un sottocampo di $\R$.

%** overlay 3 **%
      \vspace*{2pt}
      \item<3->
        Dato $\xi \in \QQ{m}$ si pu\`o scrivere \,$\xi = a + b\sqrt{m}$\,
          per\\
        $a,b \in \Q$ \emph{unicamente determinati\/}.

  \note<3>{
    L'unicit\`a della rappresentazione di $\xi \in \QQ{m}$ nella forma
    \,$\xi = a + b\sqrt{m}$ ($a,b \in \Q$) deriva dall'irrazionalit\`a
    di $\sqrt{m}$.
  }

%** overlay 4 **%
      \vspace*{2pt}
      \item<4->
        Se $m_1 = m_2A^2$ \,($A \in \Z$),\, allora\, $\QQ{m_1} = \QQ{m_2}$.
  
  \note<4>{
    Quindi, studiando gli anelli $\QQ{m}$, si pu\`o in genere supporre
    senza perdita di generalit\`a che $m$\, sia libero da quadrati.
  }    
    
    \vspace*{3pt}
    \end{itemize}
  
  \end{facts}

\end{frame}

%--------------------------------------------------------------------------

%%------------%%
%%  FRAME 06  %%
%%------------%%

\begin{frame}{I Campi $\QQ{m}$: Norma}

%** overlay 1 **%

  \note<1>[item]{
    (In continuazione dalla slide precedente)
    \begin{quote}
      Grazie all'unicit\`a della rappresentazione degli elementi di
      $\QQ{m}$ nella forma $a + b\sqrt{m}$, possiamo dare la seguente
      definizione di norma di un elemento $\QQ{m}$:\:$\ldots$
    \end{quote}
  }

  \begin{definition}[norma in $\QQ{m}$]
    Per ogni $\xi \in \QQ{m}$ con $\alert<1>{\xi = a + b\sqrt m}$\,
      ($a,b \in \Q$), definiamo\\[1pt]
    la \alert<1>{norma \,$\Norm\xi$ di $\xi$}
      come\, $\alert<1>{\Norm\xi := \abs{a^2 - mb^2}}$
  \end{definition}

  \note<1>[item]{
    Ribadire che il concetto di norma ha senso poich\`e la scrittura di
    $\xi \in \QQ{m}$ come $a + b\sqrt{m}$ \`e unica.
  }
  
  \note<1>[item]{
    Osservare che nel caso di campi $\QQ{m}$ complessi, la norma di un
    elmento $\xi\in\QQ{m}$ coincide con l'usuale norma di $\xi$ in senso
    complesso, \ie con il quadrato del modulo di $\xi$.
  }

%** overlay 2 **%
\pause%
  
  \begin{theorem}[propriet\`a basilari della norma]
    Per ogni $\xi,\xi_1,\xi_2 \in \QQ{m}$ si ha:
    \begin{enumerate}
      \item
        $\Norm{\xi} \in \Q^{+}_0$
      \item
        $\Norm{\xi_1 \xi_2} = \Norm{\xi_1}\Norm{\xi_2}$
      \item
        $\Norm{\xi} = 0 \iff \xi = 0$
      \item
        $\Norm{{\xi}^{-1}} = {\left(\Norm{\xi}\right)}^{-1}$
        se\, $\xi \neq 0$
    \end{enumerate}
  \end{theorem}

\end{frame}

%--------------------------------------------------------------------------

\note{Possiamo ora dare l'esatta definizione dei dominii da noi studiati%
      $\ldots$}

%--------------------------------------------------------------------------

%%------------%%
%%  FRAME 07  %%
%%------------%%

\begin{frame}{I Dominii $\hh{m}$: Definizione}
  
%** overlay 1 **%
  \begin{definition}[\,dominii $\hh{m}$\,]
    \vspace*{3pt}
    Sia \alert<1>{$m$} un intero \alert<1>{libero da quadrati}; allora:
    
    \vspace{2pt}
    \begin{itemize}

%** overlay 2 **%
      \item<2->
        Se\, \alert<2>{$m \congruent 2$ \,o\, $3 \pmod 4$}:\\[6pt]
%** overlay 3 **%
        \uncover<3->{
          ~~~%
          $\alert<3>{\hh{m} := \left\{ x + y\sqrt{m}:\; x,y\in\Z \right\}}$
        }

%** overlay 4 **%
      \vspace{8pt}
      \item<4->
        Se\, \alert<4>{$m \congruent 1 \!\pmod 4$}:\\[6pt]
%** overlay 5 **%
        \uncover<5->{%
          \alert<5>{%
            \begin{math}
              ~~~%
              \hh{m} :=
                \left\{ 
                  \frac{x}{2} + \frac{y}{2}\sqrt{m} :\;  x,y \in \Z
                  \text{ \small \:e\: } x \congruent y \!\pmod 2
                \right\}
            \end{math}
           }
         }
    
    \end{itemize}
  
  \note<5>{
    \begin{itemize}
      \item
        Osservare che il caso $m \congruent 0 \pmod 4$ non si pone,
        poich\`e $m$ deve essere libero da quadrati.
      \item
        Osservare che, come si dimostra facilmente, gli $\hh{m}$ sono
        tutti sottoanelli di $\QQ{m}$.
    \end{itemize}
  }

  \vspace*{3pt}
  \end{definition}

\end{frame}

%%--------------------------------------------------------------------------

%%------------%%
%%  FRAME 08  %%
%%------------%%

\begin{frame}{I Dominii $\hh{m}$: propriet\`a basilari}
 
% ** overlay 1 **
  \vspace{-8pt}
  \begin{facts} 
    Si dimostrano i seguenti fatti circa i dominii $\hh{m}$.
    
    \vspace{-1pt}
    \begin{itemize}

% ** overlay 2 **
      \item<2->
        La norma di un elemento di $\hh{m}$ \`e sempre intera:
          \vspace{-7pt}
          $$ \hspace{-25pt}\alpha \in \hh{m} \then \Norm\alpha \in \N $$

% ** overlay 3 **
      \vspace{-7pt}
      \item<3->
        Le unit\`a di $\hh{m}$ sono tutti e soli gli elementi di
        norma $1$:
          \vspace{-6pt}
          $$ \hspace{-25pt}\alpha \in \hh{m}^\ast \iff \Norm\alpha = 1 $$

% ** overlay 4 **
      \vspace{-7pt}
      \item<4->
        Ogni $\alpha \in \hh{m} \minz$ pu\`o essere scritto come prodotto\\
        di un numero finito di elementi irriducibili di $\hh{m}$.
  
  \note<5>{
    Il teorema dell'esistenza della fattorizzazione in $\hh{m}$ si
    dimostra facilmente ragionando per induzione sulla norma di $\alpha$
    (ricordando che ogni unit\`a \`e vista come prodotto di zero
    irriducibili).
  }

% ** overlay 5 **
      \vspace{2pt}
      \item<5->
        Il dominio $\hh{m}$ \`e l'anello degli interi algebrici del\\[2pt]
        campo $\QQ{m}$.

  \note<5>{
    \begin{itemize}
      \item
        Ricordare brevemente la definizione di \emph{intero algebrico\/}:
        un numero complesso che sia radice di un polinomio monico
        (non-costante) a coefficienti interi.
      \item
        Citare che la dimostrazione standard del caso $n = 3$ dell'Ultimo
        Teorema di Fermat si fa usando il fatto $\hhm{3}$ \`e un UFD.
    \end{itemize}
  }
    
      \vspace*{2pt}
    \end{itemize}

  \end{facts} 

\end{frame}

%##########################################################################

\section{Dominii che sono UFD}

%==========================================================================

\subsection{Sommario del lavoro svolto nella prima parte}

%--------------------------------------------------------------------------

%%------------%%
%%  FRAME 09  %%
%%------------%%

\begin{frame}{Sommario Del Lavoro Svolto (Prima Parte)}

%** overlay 1 **%

\begin{itemize}
    
    \vspace*{-8pt}
    \item<1->
      Nella prima parte del nostro lavoro ci siamo concentrati
      sul problema di trovare
      \alert<1>{condizioni \emph{sufficienti\/} per la fattorialit\`a di
                un dominio $\hh{m}$}.
      \vspace*{5pt}

%** overlay 2 **%

    \item<2->
      Questo ci ha portati a \alert<2>{ritrovare la nota classi dei
      dominii euclidei rispetto alla norma}, seguendo per\`o una via
      diversa da quella usuale.
      \vspace*{5pt}
    
%** overlay 3 **%

    \item<3->
      Il \alert<3>{punto di partenza} \`e stato una \alert<3>{nuova
      dimostrazione della fattorialit\`a degli interi di Gauss},
      che ora presenteremo.
      \vspace*{5pt}
  
  \end{itemize}

\end{frame}

%==========================================================================

\subsection{Fattorizzazione unica negli interi di Gauss}

%--------------------------------------------------------------------------

%%------------%%
%%  FRAME 10  %%
%%------------%%

\begin{frame}{Interi Di Gauss}

%** overlay 1 **%
  
  \vspace*{-2pt}
  \begin{definition}[interi di Gauss]
    ~~Scriviamo
      \alert<1>{$\G:=\hh{-1} = \left\{\, x+y\sqrt{-1}:\; x,y\in\Z \,\right\}$},\,
      e\\[1pt]
    ~~chiamiamo $\G$ l'anello degli \alert<1>{interi di Gauss}.
  \end{definition}

  \note<1>[item]{
    $\G$ pu\`o anche caratterizzarsi come l'insieme di tutti i numeri
    complessi aventi la parte reale e la parte immaginaria intere.
  }
  
  \note<1>[item]{
    Storicamente, $\G$ \`e stato uno dei primi anelli di interi algebrici
    ad essere studiato, e per cui \`e stata dimostrata l'unicit\`a della
    fattorizzazione in irriducibili.\\
    Esso fu introdotto ed utilizzato da Gauss intorno al 1830 nelle sue
    ricerche sulla legge di reciprocit\`a biquadratica (da cui il nome
    ``interi di Gauss'').
  }

%** overlay 2 **%
\pause
  
  \vspace{2pt}
  \textbf{Osservazioni.}
  \vspace*{-1pt}
  
  \begin{itemize}
    
%** overlay 2 **%
    
    \item <2->
        La dimostrazione standard del fatto che $\G$ \`e un UFD si\\
        fa sulla falsariga di quella per $\Z$.
  
  \note<2>{
    \begin{itemize}
      \item
        La dimostrazione standard del fatto che $\G$ \`e un UFD si fa
        sulla falsariga di quella per $\Z$.
      \item
        Ossia, si dimostra che la norma in $\G$ \`e una funzione euclidea,
        e quindi $\G$ ha un algoritmo Euclideo per trovare il M.C.D.,
        eccetera.
      \item
        Oppure piu` brevemente si osserva che
          \begin{center}
            $\G$ euclideo $\then$ $\G$ a ideali principali 
            $\then$ $\G$ fattoriale.
          \end{center}
         \end{itemize}
  } % end \note<2>

%** overlay 3 **%
    
    \item<3->
      \alert<3>{%
        Noi procederemo per\`o in modo diverso, cio\`e adattando\\
        al caso di $\G$ una dimostrazione della fattorialit\`a di $\Z$\\
        diversa da quella canonica.
      }
      
%** overlay 4 **%
    
    \item<4>
        La nostra dimostrazione sar\`a basata sul principio del\\
        minimo intero, e su di un semplice lemma-chiave che\\
        ora presenteremo.
  
  \end{itemize}
  
\end{frame}

%--------------------------------------------------------------------------

%%------------%%
%%  FRAME 11  %%
%%------------%%

\begin{frame}{Un Utile Lemma}
  
 %** overlay 1 **%
  
  \note<1>[item]{
    Vediamo ora un lemma che, seppur semplice, \`e la chiave
    per la nostra dimostrazione della fattorialit\`a di $\G$.
  }
  
  \vspace{-10pt}
  \begin{lemma}<1->[lemma chiave per $\G$]
    Sia $\tau \in \C\minz$ tale che \,$\Norm{\tau} \leq 1$. Allora esiste
    $\epsilon \in \G$ unitario tale che \,$\Norm{\tau - \epsilon} < 1$.
  \end{lemma}

  \note<1>[item]{
    Osserviamo che qui \,$\Normop$\, indica la norma canonica dei numeri
    complessi; questo non \`e tuttavia un problema, poich\`e abbiamo gi\`a
    osservato che, quando $m < 0$, la norma in $\hh{m}$ coincide con
    l'usuale norma complessa.
  }

%** overlay 2 **%
  \vspace{7pt}
  \begin{myblock}<2->{brown}{white}{Interpretazione geometrica del lemma}
    Il disegno a pagina seguente offre un'interpretazione \\
    geometrica del lemma, mostrando visivamente che
      l'unione \\
    dei quattro cerchi unitari \emph{aperti\/} di centri 
      rispettivamente \\
    $(-1,0)$, \,$(0,-1)$, \,$(+1,0)$, \,$(0,+1)$\, ricopre
      completamente il \\
    cerchio unitario \emph{chiuso\/} centrato nell'origine
      \,$(0, 0)$,\, \emph{esclusa\/}\\
    \emph{l'origine stessa\/}.
  \end{myblock}

\end{frame}

%--------------------------------------------------------------------------

%%------------%%
%%  FRAME 12  %%
%%------------%%

\begin{frame}{Interpretazione Geometrica Del Lemma}

 %** overlay 1 **%

  \begin{picture}(340,140)(-30, +0)
    
    % cerchio centrale
    \put(100, 80){\color{myblue}\circle{40}}
    \put(100, 80){\color{myblue}\circle{40}}
    
    \put(100, 60){\color{mygreen}\circle{40}}
    \put(100, 60){\color{mygreen}\circle{40}}
    
    \put(100, 100){\color{mygreen}\circle{40}}
    \put(100, 100){\color{mygreen}\circle{40}}
    
    \put(80, 80){\color{mygreen}\circle{40}}
    \put(80, 80){\color{mygreen}\circle{40}}
    
    \put(120, 80){\color{mygreen}\circle{40}}
    \put(120, 80){\color{mygreen}\circle{40}}
    
    \put(20, 80){\vector(1,0){165}}
    \put(20, 80){\vector(1,0){165}}
    
    \put(100, 20){\vector(0,1){120}}
    \put(100, 20){\vector(0,1){120}}
    
    % origine
    \put(100, 80){\color{red}\circle*{2}}
    \put(100, 80){\color{red}\circle*{2}}
    
    \put(178,70){\mbox{\textit{x}}}
    
    \put(105,135){\mbox{\textit{y}}}
  
  \end{picture}

  \vspace{-10pt}
  {
    \small
    Tutti i punti del piano complesso che distano \emph{al pi\`u 1\/}
    dall'origine distano \emph{meno di 1\/} da almeno un elemento unitario
    di $\C$ (fatta eccezione per l'origine stessa).
  }

\end{frame}

%--------------------------------------------------------------------------

%%------------%%
%%  FRAME 13  %%
%%------------%%

\begin{frame}{Gli Interi Di Gauss sono un UFD}
\framesubtitle{Enunciato e dimostrazione}

%** overlay 1 **%
  
  \begin{theorem}
    L'anello $\G$ degli interi gaussiani \`e un dominio fattoriale.
  \end{theorem}
  
  \note<1>{
    Vedremo la dimostrazione di questo teorema alquanto dettagliatamente,
    poich\`e essa costituisce il punto di partenza ed il ``cuore'' della
    prima parte del nostro lavoro.
  }

%** overlay 2 **%
\pause
  
  \vspace{4pt}
  \textbf{\textsc{Dimostrazione.}}
  
  \vspace{2pt}
  Siano:
  
  \note<2>{
    Iniziamo suddividendo gli elementi non nulli di $\G$ in due
    sottoinsiemi $K$ e $H$, in base alle loro propriet\`a di
    fattorizzazione.
  }

  \begin{itemize}
%** overlay 3 **%
    \item<3->
      \alert<3>{$K$}\, l'insieme degli \alert<3>{elementi di $\G$}
      aventi \alert<3>{fattorizzazione} in irriducibili essenzialmente
      \alert<3>{unica};
%** overlay 4 **%
    \item<4->
      \alert<4>{$H$}\, l'insieme degli \alert<5>{elementi di $\G$} aventi
      \alert<4>{almeno due fattorizzazioni} in irriducibili essenzialmente
      \alert<4>{distinte}.
  \end{itemize}
  
%** overlay 5 **%
  \uncover<5>{
    \begin{center}
      \fbox{\alert{Vogliamo dimostrare che $H = \emptyset$}}
    \end{center}
  }

\end{frame}
  
%--------------------------------------------------------------------------

%%------------%%
%%  FRAME 14  %%
%%------------%%

\begin{frame}[t]{Gli Interi Di Gauss sono un UFD}
\framesubtitle{Dimostrazione (continua)}
  
%** overlay 1 **%

  \begin{itemize}
    
    \vspace{5pt}
    \item<1->
      \alert<1>{Procediamo per assurdo},
      assumendo che $H \neq \emptyset$

%** overlay 2 **%
   \vspace{3pt}
   \item<2->
      Possiamo scegliere allora
      \alert<2>{$\xi_0 \in H$ avente norma minima}

  %** overlay 3 **%
    \vspace{3pt}
    \item<3->
      Quindi 
      \alert<3>{$\xi_0$ ha due fattorizzazioni essenzialmente distinte
      ciascuna delle quali ha almeno due fattori}, ossia:
      \vspace{-6pt}
      \begin{equation}\bmlabel{two_factorizations_1}
          \xi_0 = \pi_1\pi_2\cdots\pi_r = \sigma_1\sigma_2\cdots\sigma_s
      \end{equation}
      \vspace{-16pt}\\ % do *not* remove this line break
      con $r,s \geq 2$, per opportuni $\pi_k,\sigma_h$ irriducibili
  
  \note<3>{
    \begin{itemize}
      \item
        Dire che non sappiamo se $r = s$, e questo in effetti non ci
        interessa.
      \item
        Spiegare un attimo perch\`e $r,s \geq 2$.
    \end{itemize}
  }

    
%** overlay 4 **%
    \vspace{5pt}
    \item<4->
      Dalla minimalit\`a di $\Norm{\xi_0}$ deriva facilmente che
      \alert<4>{nessun $\pi_i$ pu\`o essere associato ad alcun $\sigma_j$}
  
  \end{itemize}

\end{frame}

%--------------------------------------------------------------------------

%%------------%%
%%  FRAME 15  %%
%%------------%%

\begin{frame}{Gli Interi Di Gauss sono un UFD}
\framesubtitle{Dimostrazione (continua)}

%** overlay 1 **%
  \vspace{-4pt} 
  \begin{itemize}
    
    \item<1->
      Per simmetria possiamo supporre che\,
      \alert<1>{$ \Norm{\pi_1} \leq \Norm{\sigma_1}$}
    
%** overlay 2 **%
    \item<2->
      Poniamo $\alert<2>{\tau := \frac{\pi_1}{\sigma_1}}\in \C$
    
%** overlay 3 **%
    \item<3-> 
      Allora si ha \alert<3>{$\tau \neq 0$}\, e\,
      \alert<3>{$\Norm\tau \leq 1$}

%** overlay 4 **%
    \item<4->
      Per il lemma chiave su $\G$, esiste
      \alert<4>{$\eps \in \G^\star$} tale che:\\
      \begin{center}
        \alert<4>{$\Norm{\tau - \eps} < 1$}
      \end{center}

%** overlay 5 **%
    \item<5->
      Ci\`o si pu\`o equivalentemente riscrivere come:\\
      \begin{center}
        \alert<5>{$\Norm{-\eps\sigma_1 + \pi_1} < \Norm{\sigma_1}$}
      \end{center}

%** overlay 6 **%
    \item<6->
      Definiamo ora~\;\alert<6>{%
          \begin{math}
           \psi :=
              -\eps\,\xi_0 + \pi_1\sigma_2\cdots\sigma_s
           \end{math}
        }

  \end{itemize}

%%%** overlay 7 **%
  \vspace{-8pt} 
  \uncover<7->{
    \begin{center}
      \fbox{
        \begin{minipage}{3.2in}
          \begin{center}
            \alert<7>{Dobbiamo ora arrivare ad una contraddizione\\
                      analizzando le propriet\`a di $\psi$}
          \end{center}
        \end{minipage}
      }
    \end{center}
  }

\end{frame}

%--------------------------------------------------------------------------

%%------------%%
%%  FRAME 16  %%
%%------------%%

\begin{frame}{Gli Interi Di Gauss sono un UFD}
\framesubtitle{Dimostrazione (continua)}
  
%** overlay 1 **%
  \begin{itemize}
    
    \vspace{-1pt}
    \item<1->
      Poich\`e $\pi_1$ e $\sigma_1$ non sono associati, si ha\,
      \alert<1>{$\psi \neq 0$}

%** overlay 2 **%
    \item<2->
      Dalla disuguaglianza
      \,$\Norm{-\eps\sigma_1 + \pi_1} < \Norm{\sigma_1}$\,
      si ricava:\\[7pt]
      \begin{center}
        $ \alert<2>{\Norm{\psi} < \Norm{\xi_0}} $
      \end{center}
        \vspace*{-2pt}
        \small
        Infatti:
        \\[-15pt]
        \begin{eqnarray*}
            \Norm{\psi} &
          \!\!\! = \!\! &
             \Norm{-\eps\xi_0 + \pi_1\sigma_2\cdots\sigma_s}
          =\,
             \Norm{%
               -\eps\sigma_1\sigma_2\cdots\sigma_s +%
               \pi_1\sigma_2\cdots\sigma_s%
             }
          \\[2pt]
          & \!\!\! = \!\! &
            {
             \Norm{-\eps\sigma_1 + \pi_1}
             \Norm{\sigma_2\cdots\sigma_s}
            }
          {~<~\,}
            \Norm{\sigma_1}\Norm{\sigma_2\cdots\sigma_s}
          \\[2pt]
          & \!\!\! = \!\! & 
            \Norm{\sigma_1\cdots\sigma_s} 
          {~=~\,}
            \Norm{\xi_0}
        \end{eqnarray*}
        \\[-17pt]

  \note<2>{
    Eccoci ora al punto focale della dimostrazione: $$ \psi \in K $$
  }

%** overlay 3 **%
    \item<3->
      Vista la propriet\`a di minimo di $\xi_0$, se ne deduce che\\
      \alert<3>{$\psi \in K$,\, ossia \,$\psi$\,\ ha una
                fattorizzazione in irriducibili essenzialmente unica}
  
  \end{itemize}

\end{frame}

%--------------------------------------------------------------------------

%%------------%%
%%  FRAME 17  %%
%%------------%%

\begin{frame}{Gli Interi Di Gauss sono un UFD}
\framesubtitle{Dimostrazione (continua)}
  
%** overlay 1 **%
  \begin{itemize}
      
    \item<1->
      Riscriviamo per chiarezza la definizione di $\psi$:
      \\[-9pt]
      \begin{equation}\bmlabel{psidefs}
        \hspace{-0.16in}\alert<1>{%
          \psi = -\eps\,\xi_0 + \pi_1\sigma_2\cdots\sigma_s = 
          \left( -\eps\sigma_1 + \pi_1 \right)
          \left( \sigma_2\cdots\sigma_s  \right)
         }
       \end{equation}
      \\[-11pt]

%** overlay 2 **%
    \item<2->
      Poich\`e $\pi_1 \divides \xi_0$, si ha evidentetemente anche\,
      \alert<2>{$\pi_1 \divides \psi$}
  
%** overlay 3 **%
    \item<3->
      Poich\`e $\pi_1$ non \`e associato ad alcun $\sigma_j$,
      dall'equazione~(\bmref{psidefs}) segue allora facilmente che:\\[2pt]
      \begin{center}
        \alert<3>{$\pi_1 \divides \left(-\eps\sigma_1 + \pi_1\right)$}
      \end{center}

%** overlay 4 **%
    \item<4->
      \alert<4>{Osservazione Fondamentale}: nel dedurre ci\`o ha giocato
      un ruolo \emph{fondamentale\/} il fatto che $\psi \in K$, i.e.~che
      $\psi$~ha una fattorizzazione in irriducibili essenzialmente
      unica

  \note<4>{
    In effetti, in generale, se $x$ \`e un elemento avente fattorizzazione
    unica in irriducibili e $p$ \`e un irriducibile che divide $x$, allora
    se $x = x_1 x_2$ si ha $p \divides x_1$ o $p \divides x_2$.
    Vedasi ``allegato.pdf'' per una dimostrazione.
  }

  \end{itemize}

\end{frame}

%--------------------------------------------------------------------------

%%------------%%
%%  FRAME 18  %%
%%------------%%

\begin{frame}{Gli Interi Di Gauss sono un UFD}
\framesubtitle{Dimostrazione (conclusione)}
  
%%%** overlay 1 **%
  \uncover<1->{
    \vspace{-3pt}
    \begin{center}
      \fbox{
        \begin{minipage}{3in}
          \begin{center}
            \alert<1>{Ma a questo punto abbiamo raggiunto\\
                      la contraddizione cercata}
          \end{center}
        \end{minipage}
      }
    \end{center}
  }

%** overlay 2 **%
\pause
  Infatti:
  
  \begin{itemize}    
%** overlay 3 **%
    \item<3->
      Poich\`e\,
        $\pi_1 \divides \left(-\eps\sigma_1 + \pi_1\right)$
      \,ed \,$\eps$\, \`e unitario, otteniamo:\\
      \vspace{-2pt}
       \begin{center}
         \alert<3>{$\pi_1 \divides \sigma_1$}
       \end{center}
%** overlay 4 **%
    \item<4->
       Quindi, essendo $\pi_1$ e $\sigma_1$ irrudicibili, si ha
       che \alert<4>{$\pi_1$ e $\sigma_1$ sono associati}
%** overlay 5 **%
    \item<5->
       \alert<5>{Ma questa \`e una contraddizione}, poich\`e sappiamo
       che nessun $\pi_i$ pu\`o essere associato ad alcun $\sigma_j$
  \end{itemize}

%** overlay 6 **%
  \uncover<6>{
    \vspace{-10pt}
    \begin{center}
      \fbox{\alert<6>{E la dimostrazione \`e conclusa}}
    \end{center}
  }

\end{frame}

%==========================================================================

\subsection{Generalizzazione ad altri dominii}

%--------------------------------------------------------------------------

%%------------%%
%%  FRAME 19  %%
%%------------%%

\begin{frame}[t]{Generalizzazione Del Nostro Metodo Ad Altri Dominii}
  
%** overlay 1 **%
  \begin{itemize}
    
    \vspace{3pt}
    \item<1->
      ~La dimostrazione appena esposta dell'unicit\`a della\\
      ~fattorizzazione in $\G$ \`e in effetti pi\`u generale di quanto\\
      ~sembri a prima vista.
  
%** overlay 2 **%
    \vspace{3pt}
    \item<2->
        ~Pi\`u precisamente, essa pu\`o essere facilmente adattata\\
        ~per dimostrare che%
        \alert<2,3>{
          ogni dominio $\hh{m}$ che soddisfa\\
          ~la seguente propriet\`a \`e fattoriale%
        }:
  \end{itemize}

%** overlay 3 **%
  \uncover<3->{
    \vspace{-8pt}
    \begin{center}
      \begin{minipage}{4in}
        \begin{beamercolorbox}[shadow=true,rounded=true,center]{dullcolors}
          $\,\forall \tau \in \QQ{m}$\, tale che
            \,$\Norm{\tau} \leq 1$\, e 
            \,$\tau \neq 0$\, e 
            \,$\tau \notin \hh{m}^\ast$,\\[2pt]
          $\exists\;\theta \in \hh{m}$\, tale che\,
            $\Norm{\theta\tau + 1} < 1$
        \end{beamercolorbox}
      \end{minipage}
    \end{center}
  }

\end{frame}

%--------------------------------------------------------------------------

%%------------%%
%%  FRAME 20  %%
%%------------%%

\begin{frame}[t]{Forme Equivalenti Per I Nostri Risultati (1)}

%** overlay 1 **%
  \begin{itemize}
    
    \item<1->
      La propriet\`a appena vista non sembra certo molto chiara n\'e
      ``amichevole''; risulta quindi naturale chiedersi se essa possa
      essere \alert<1>{riformulata in modo da risultare pi\`u maneggevole}.

  \note<1>{
    Citare il fatto che la propriet\`a in questione pu\`o essere usata
    direttamente e.g.~per dimostrare che $\hh{2}$ \`e fattoriale, ma i
    calcoli risultanti sono ingombranti, tediosi e per niente ovvi.
  }

%** overlay 2 **%
    \vspace{4pt}
    \item<2->
      La risposta \`e positiva, ed in effetti risulta che la propriet\`a
      poco fa enunciata \`e \alert<2>{equivalente alla seguente}:
      
      \begin{minipage}{3.8in}
      \begin{property}{$F$}
        \vspace*{2pt}
        ~~~~$\,\forall\, \alpha, \beta \in \hh{m}$\, tali che\,
            $\beta \neq 0$\, e\, $\Norm{\alpha} \geq \Norm{\beta}$,\\[2pt]
        ~~~~$\exists\; \theta,\rho \in \hh{m}$ con\, $\alpha =
              \beta\theta + \rho$\, e\, $\Norm{\rho} < \Norm{\alpha}$
        \end{property}
      \end{minipage}
      \vspace*{3pt}

  \end{itemize}

  \note<2>{
    La chiamiamo ``\pp{$F$}'' per ricordarci che implica la fattorialit\`a.
  }

\end{frame}

%--------------------------------------------------------------------------

%%------------%%
%%  FRAME 21  %%
%%------------%%

\begin{frame}[t]{Forme Equivalenti Per I Nostri Risultati (2)}

%** overlay 1 **%
  \begin{itemize}
    
    \vspace*{2pt} 
    \item<1->
      Ricordiamo che un dominio $\hh{m}$ si dice \alert<1>{normo-euclideo}\\
      quando soddisfa la seguente propriet\`a:

      \vspace*{-1pt} 
      \begin{minipage}{3.8in}
        \begin{property}{$E$}
        \vspace*{2pt}
          \hspace{0.16in}$\,\forall\; \alpha, \beta \in \hh{m}$\, con\,
             $\beta \neq 0$,~\;$\exists\; \theta,\rho \in \hh{m}$\, con\\[1pt]
          \hspace{0.16in}~~\,$\alpha = \beta\theta + \rho$\, e\,
               $\Norm{\rho} < \Norm{\beta}$
        \end{property}
      \end{minipage}
    
%** overlay 2 **%
    \vspace*{12pt} 
    \item<2->
      Si dimostra che la \alert<2>{\pp{$E$}} e la \alert<2>{\pp{$F$}} sono\\
      effettivamente \alert<2>{equivalenti}
  
  \note<2>{
    Il fatto che \,$(\pp{$E$}) \then (\pp{$F$})$\, \`e pressoch\`e evidente
    ed immediato a dimostrasi; il fatto veramente interessante \`e che\,
    $(\pp{$F$}) \then (\pp{$E$})$. Ci\`o pu\`o dimostrarsi facilmente
    usando il principio del minimo intero.
  }

%** overlay 3 **%
    \vspace*{4pt} 
    \item<3->
      Perci\`o, come promesso nell'introduzione, \alert<3>{abbiamo ora \\
      ritrovato per una via inusuale la nota classe dei dominii \\
      euclidei rispetto alla norma}
  
  \end{itemize}

  \note<3>{
    Sottolineare bene il fatto che, come promesso, abbiamo ritrovato per
    una via inusuale la nota classe dei dominii euclidei rispetto alla
    norma.\\
    Ripercorrere brevemente i passaggi che ci hanno portato fin qui:
    \begin{itemize}
      \item
        Nuova dimostrazione della fattorialit\`a di $\G$.
      \item
        Generalizzazione del nostro metodo alla classe dei dominii $\hh{m}$
        che soddisfano la \pp{$F$}.
      \item
        Dimostrazione del fatto che la \pp{$F$} e la \pp{$E$} sono
        equivalenti, e quindi che la classe dei dominii fattoriali da noi
        trovata coincide con l'usuale classe dei dominii euclidei rispetto
        alla norma.
    \end{itemize}
  }

\end{frame}

%--------------------------------------------------------------------------

%%------------%%
%%  FRAME 22  %%
%%------------%%

\begin{frame}[t]{Identificazione dei dominii normo-euclidei}
 
 {
 \small

%** overlay 1 **%
  \begin{itemize}
    
    \item<1->
      Vista l'equivalenza tra la \pp{$F$} e la \pp{$E$}, siamo costretti
      a concludere che \alert<1>{il nostro criterio di fattorialit\`a
      non pu\`o applicarsi a quei dominii che non sono normo-euclidei}.
    
%** overlay 2 **%
    \item<2->
      Sfortunatamente, essi sono la maggioranza, come illustrato dal
      \alert<2>{seguente classico risultato} di teoria dei numeri, che
      mostra insieme \alert<2>{le potenzialit\`a ed i limiti del nostro
      metodo}:
 
 \normalsize

      \begin{theorem}[classificazione dei dominii normo-euclidei]
        Il dominio $\hh{m}$ \`e normo-euclideo per tutti e soli i %
        seguenti\\[2pt]
        valori di\, $m$:\\[-18pt]
        \begin{eqnarray*}
          m & \!\! = \!\! &
            -11,\,
            -7,\,
            -3,\,
            -2,\,
            -1,\;
             2,\;
             3,\;
             5,\;
             6,\;
             7,\;
            11,\;
            13,\;
            17,\;
            19,\\
          & &
            21,\;
            29,\;
            33,\;
            37,\;
            41,\;
            57,\;
            73
        \end{eqnarray*}
      \end{theorem}

  \end{itemize}

 }

\end{frame}

%##########################################################################

\section{Dominii che non sono UFD}

%==========================================================================

\subsection{Sommario del lavoro svolto nella seconda parte}

%--------------------------------------------------------------------------

%%------------%%
%%  FRAME 23  %%
%%------------%%

\begin{frame}{Sommario Del Lavoro Svolto (Seconda Parte)}

%** overlay 1 **%
  \begin{itemize}
    
    \vspace{-12pt}
    \item<1->
      Nella prima parte del nostro lavoro ci eravamo concentrati\\
      sul problema di trovare condizioni sufficienti affinch\`e un\\
      certo dominio $\hh{m}$ sia \emph{fattoriale\/}.
    
%** overlay 2 **%
    \vspace{12pt}
    \item<2->
      Nella seconda parte ci siamo invece rivolti al problema\\
      opposto, ossia quello di trovare \alert<2>{condizioni sufficienti}\\
      affinch\`e un certo dominio
      \alert<2>{$\hh{m}$ sia \emph{non fattoriale\/}}.
    
  \end{itemize}

\end{frame}

%--------------------------------------------------------------------------

%%------------%%
%%  FRAME 24  %%
%%------------%%

\begin{frame}{Sommario Del Lavoro Svolto (Seconda Parte)}

%** overlay 1 **%
  \begin{itemize}

    \vspace{-6pt}
    \item<1->
      Usando risultati basilari di teoria dei numeri, siamo\\
      riusciti a ricavare condizioni abbastanza generali\\
      e interessanti
    
%** overlay 2 **%
    \vspace{6pt}
    \item<2->
      I risultati pi\`u importanti da noi utilizzati sono stati:
      \begin{itemize}
        \item il teorema cinese del resto
        \item il teorema di reciprocit\`a quadratica
        \item il teorema di Dirichlet sui primi nelle progressioni
              aritmetiche
      \end{itemize}
    
%** overlay 3 **%
    \vspace{3pt}
    \item<3->
      Per mancanza di tempo, daremo \alert{solo un sommario\\
      dei risultati ottenuti}, senza approfondimenti e senza\\
      dimostrazioni
  
  \end{itemize}

\end{frame}

%--------------------------------------------------------------------------

%%------------%%
%%  FRAME 25  %%
%%------------%%

\begin{frame}{Condizioni per la fattorialit\`a di $\hhm{m}$ e $\hh{m}$}

%** overlay 1 **%
  \vspace*{-7pt}
  \begin{theorem}[\,c.n.~fattorialit\`a di $\hhm{m}$\,]
    Sia \,$m > 1$ un intero libero da quadrati tale che \,$\hhm{m}$ sia\\
    un dominio fattoriale. Allora o \,$m = 2$, oppure $m$ \`e un numero\\
    \emph{primo\/} della forma \,$m = 4q - 1$\, per \,$q > 1$ anch'esso
    primo.
  \end{theorem}

%** overlay 1 **%
\pause
  \vspace*{2pt}
  \begin{theorem}[\,c.n. fattorialit\`a di $\hh{m}$\,]
    Sia $m > 1$ un intero libero da quadrati tale che \,$\hh{m}$ sia
    un dominio fattoriale. Allora:
    \begin{itemize}
      \item 
        $m$ \`e primo,\: oppure:
      \item 
        $m = 2p$\, per $p$ primo con \,$p \congruent 3 \!\pmod 4$,\:
        oppure:
      \item
        $m = pq$\, per $p$, $q$ primi con\,
        $p \congruent q \congruent 3 \!\pmod 4$.
    \end{itemize}
  \end{theorem}

  \note<2>{
    Osservare che questo risultato \`e abbastanza buono, considerato il
    fatto che \`e stato ottenuto con metodi assolutamente elementari.\\
    In effetti, tuttoggi ancora non si sa se i dominii reali $\hh{m}$
    fattoriali siano finiti o infiniti (si congettura che siano infiniti,
    contrariamente a quanto accade per i dominii complessi).
  }

\end{frame}

%##########################################################################

\section{Conclusioni e Sommario}

\begin{frame}<beamer>{Conclusioni e Sommario}
  \tableofcontents[currentsection,currentsubsection]
\end{frame}

%--------------------------------------------------------------------------

%%------------%%
%%  FRAME 26  %%
%%------------%%

\begin{frame}[t]{Conclusioni e Sommario}
 
 % Keep the summary *very short*.
 {
  \small
  
  \begin{itemize}

%** overlay 1 **%
  \item<1->
    Nel nostro lavoro, abbiamo dato una \alert<1>{nuova dimostrazione\\
    della fattorialit\`a di $\G$}.
%** overlay 2 **%
  \vspace{2pt}
  \item<2->
    Abbiamo poi \alert<2>{generalizzato questa tecnica dimostrativa},\\
    mostrando che tutti i dominii \alert<2>{$\hh{m}$ soddisfacenti una\\
    certa \pp{$F$} sono fattoriali}.
%** overlay 3 **%
  \vspace{2pt}
  \item<3->
    Abbiamo quindi mostrato come il fatto che $\hh{m}$ soddisfi\\
    questa \pp{$F$} equivalga all'essere $\hh{m}$ euclideo\\
    rispetto alla norma,
    \alert<3>{%
      ritrovando quindi, per una via diversa\\
      da quella usuale, la nota classe dei dominii euclidei rispetto\\
      alla norma%
    }.
%** overlay 4 **%
  \vspace{2pt}
  \item<4->
    Infine, \alert<4>{usando tecniche e risultati elementari} 
      ma non banali\\
    di teoria dei numeri, abbiamo trovato \alert<4>{interessanti
      condizioni\\
    necessarie per la fattorialit\`a dei dominii $\hh{m}$}\,
      (per $m$\, sia\\
    positivo che negativo).
  
  \end{itemize}
 
 }

\end{frame}

%##########################################################################

\end{document}

% vim: ft=tex ts=2 sw=2 et
