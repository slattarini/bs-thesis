\chapter{Preliminary results}\label{preliminary}

\section{The concepts of conjugate and norm}\label{conj_and_norm}

%--------------------------
% DEFINITION OF Q(sqrt(m))
%--------------------------
\begin{defn}
For all $m\in \Z$ let
\begin{equation}\label{quadraticField}
\QQ{m} := \left\{ a + b\sqrt m: a,b \in \Q\right\}.
\end{equation}
\end{defn}

It is straightforward to verify that $\QQ{m}$ is a subfield of $\C$
and that $\QQ{m} = \Q$ if and only if $m$ is a perfect square, \ie
$m = h^2$ for an appropriate $h\in\Z$.
So, in the rest of this section, we will denote by $m$
a fixed non-square in $\Z$.

%--------------------------------------
% DEFINTION AN PROPERTIES OF CONJUGATE
%--------------------------------------
\begin{defn}\label{conjugatedef}
For each\, $z \in \QQ{m}$, written\, $z = x + y\sqrt m$ with $x,y \in \Q$,
we define\, $\cg z:= x - y\sqrt m$,\, and call it the {\emph{conjugate}}
of $z$.
\end{defn}


The map $z\mapsto \cg z$ is an involutory automorphism of $\QQ{m}$, \ie
%
\begin{equation}\label{conjugate1}
\forall z_1,z_2 \in \QQ{m}: \qquad
\cg{z_1 + z_2}= \cg z_1 + \cg z_2, \quad
\cg{z_1z_2} = \cg z_1\cg z_2 , \quad
\cg1 = 1, \quad \cg{\cg{z_1}} = z_1.
\end{equation}
%
It follows in particular that:
\begin{equation}\label{conjugate2}
\forall z\in \QQ{m}:\qquad
\cg{z} = 0 \iff z=0,\quad\
z\not=0 \then \cg{z^{-1}}= {({\cg z})}^{-1}.
\end{equation}

Note that when $m < 0$, the conjugate of $z \in \QQ{m}$ as defined
above coincide with the usual complex conjugate of $z$ in $\C$.

%-----------------------------------
% DEFINITION AND PROPERTIES OF NORM
%-----------------------------------
\begin{defn}\label{normdef}
For $z\in \QQ{m}$ we define $$\Norm{z}:= \abs{z \cg z}$$
and call it the {\emph{norm}} of $z$.
\end{defn}
Note that for every $z\in \QQ{m}$ it is
$z\cg{z} \in \Q$, and then\, $\Norm{z} = \abs{z \cg z}
= \pm{z \cg z}$.


\begin{lem}\label{norm} For all
$z,\ z_1,\ z_2\in \QQ{m}$:

\begin{itemize}
\item[{\rm (i)}] 
$\Norm{z} = \Norm{\cg z}$

\item[{\rm (ii)}]
$\Norm{z_1z_2} = \Norm{z_1}\Norm{z_2}$

\item[{\rm (iii)}]
$\Norm{z} = 0 \iff z = 0$

\item[{\rm (iv)}]
$z \not= 0 \then \Norm{z^{-1}} = {\Norm{z}}^{-1}$
\end{itemize}
\end{lem}

\begin{proof}
%
It results:
%
\begin{itemize}

\item

$\Norm{z} = \abs{z\cg z} = \abs{\cg{\cg z}\ \cg z} =
\abs{\cg z\ \cg{\cg z}} = \Norm{\cg z}$;

\item

$\Norm{z_1z_2} = \abs{(z_1z_2) (\cg{z_1z_2})} =
\abs{z_1z_2 \cg z_1 \cg z_2} =
\abs{(z_1\cg z_1) (z_2 \cg z_2)} = \Norm{z_1}\Norm{z_2}$;


\item

since $\cg z = 0 \iff z = 0$ we have:\\
\mbox{$\Norm{z} = 0 \iff \abs{z\cg z} = 0 \iff
z\cg z = 0 \iff z = 0\,\ {\rm or}\,\ \cg z = 0\ \iff
z = 0$;}

\item

$z\not=0 \then \Norm{z} \Norm{z^{-1}} = \Norm{zz^{-1}} =
\Norm{1} = 1 \then \Norm{z^{-1}} = {\Norm{z}}^{-1}$;

\end{itemize}
and the lemma is proved.
\end{proof}
%

\section{Normed Domains}\label{normed_domains}

%------------------------------
% DEFINITION OF NORMED DOMAINS
%------------------------------
\begin{defn}\label{normedDomain}
Let $D$ be a domain. We say that D is a \emph{quadratic
normed domain\/} if there exists a fixed non-square integer
$m$ such that $D$ is a subring of $\QQ{m}$, and:
\begin{itemize}
\item $\forall \alpha\in D:\: \Norm{\alpha} \in \N$
\item $\forall \alpha\in D:\: \cg{\alpha} \in D$
\end{itemize}
\end{defn}

Equivalently, we can say that $D$ is a \emph{quadratic
normed domain\/} if there exists a fixed non-square intege
 $m$ such that $D$ is a subring of $\QQ{m}$, and:
$$ 
\forall \alpha\in D:\: \cg{\alpha} \in D
\textrm{\,\ and\,\ } \alpha \cg{\alpha} \in \Z
$$

From now on, for sake of conciseness, we will
write simply \emph{normed domain\/} instead of
\emph{quadratic normed domain\/}.

\bigskip
In the rest of this section, $m$ will be a fixed
non-square integer; moreover, we will denote by $D$
a given normed domain.

%------------------------
% CHARACTERIZATION OF D*
%------------------------
\begin{lem}
Let \,$\alpha \in D$;\, then:
\begin{equation}\label{D_ast}
\alpha \in D^\ast \iff
\Norm{\alpha} = 1 \iff
\cg\alpha \in D^\ast .
\end{equation}
\end{lem}
%
\begin{proof}
%
We start proving the first double implication of
\eqref{D_ast}.

\begin{itemize}

\item $\alpha \in D^\ast \then \exists\, \beta \in D :
\alpha\beta = 1 \then 1 = \Norm{1} = \Norm{\alpha\beta} =
\Norm{\alpha}\Norm{\beta}  \then \Norm{\alpha} =
\Norm{\beta} = 1$ (since $\Norm{\alpha}$ and $\Norm{\beta}$
are natural numbers). We conclude that $\Norm{\alpha} = 1$.

\item $\Norm{\alpha} = 1 \then {\alpha \cg\alpha} =
\pm{\,\Norm{\alpha}} = \pm 1 \then
(\alpha)(\pm{\cg\alpha}) = 1 \then \alpha \in D^\ast$\,
(since if $\alpha \in D$ then $\cg\alpha \in D$ and thus
also $\pm\cg\alpha \in D$).

\end{itemize}

Thus we have proved the first double implication of
\eqref{D_ast}. The second double implication can easily
be deduced from the first, noting that
$\forall \gamma\in D:\: \Norm{\gamma}=\Norm{\cg\gamma}$.
%
\end{proof}

%--------------------------------------------
% FIRST LEMMA ON EXISTENCE OF FACTORIZATION
% INTO IRREDUCIBLES IN A NORMED DOMAINS
%--------------------------------------------
\begin{lem} Let $\alpha, \beta\in D$ be such that
$\beta \divides \alpha$.
Then $\beta$ is associate to $\alpha$ if and only if
$\Norm{\beta}=\Norm{\alpha}$.
\end{lem}

\begin{proof}
%
Assume first that $\beta$ is associate to $\alpha$, \ie
$\beta=\epsilon \alpha$, for an appropriate
$\epsilon\in D^\ast$. Thus we get immediatly
$\Norm{\beta} = \Norm{\alpha}\Norm{\epsilon} =
\Norm{\alpha}$.

Viceversa, assume now that $\Norm{\beta} = \Norm{\alpha}$.
We can distinguish two cases.

\begin{enumerate}
%
\item[\textbf{1.}] $\beta \neq 0$,\, \ie\:$\Norm{\beta} \neq 0$.

Write $\alpha = \beta\gamma$, with $\gamma \in D$.
It follows:
$$
\Norm{\beta} = \Norm{\alpha} = \Norm{\beta}\Norm{\gamma}
\then \Norm{\gamma} = 1.
$$
We conclude $\gamma \in D^\ast$, \ie $\alpha$ and
$\beta$ are associate.
%
\item[\textbf{2.}] $\beta=0$.

The assumption $\beta \divides \alpha$ implies now
$\alpha=0$. So $\alpha = \beta = 0$ and, a fortiori,
$\alpha$ and $\beta$ are associate.
%
\end{enumerate}
%
In both cases, our claim follows.
\end{proof}

%----------------------------------------------------
% COROLLARY REGARDING THE EXISTENCE OF FACTORIZATION
% INTO IRREDUCIBLES IN A NORMED DOMAINS
%----------------------------------------------------
The previous two lemmas immediately imply the
following useful result:

\begin{cor} Let $\alpha\in D\setminus\left(D^\ast\cup\{0\}\right)$.
Then $\alpha$ is irreducible if and only if
\begin{equation*}
\forall\, \beta \in D:\enskip
1 < \Norm{\beta} < \Norm{\alpha}\then
\beta \notdivides \alpha
\end{equation*}
\end{cor}

%--------------------------------------------
% SECOND LEMMA ON EXISTENCE OF FACTORIZATION
% INTO IRREDUCIBLES IN A NORMED DOMAINS
%--------------------------------------------
\begin{lem}\label{factorizationIntoIrreducibles}
Every $\alpha \in D\minz$ can be written as a
product of a finite number of irreducibles of $D$.
\end{lem}

\begin{proof}
%
By contradiction, let $\alpha \in D \minz$ do not
satisfy our claim. Clearly we may assume that
$\Norm{\alpha}$ has the least possible value. Since every
unit and every irreducible obviously satisfy our claim,
$\alpha$ must be reducible. By the previous Corollary
there exists $\beta\in D$ such that $1 < \Norm{\beta} <
\Norm{\alpha}$ and $\beta \divides \alpha$.
Writing $\alpha = \beta\gamma$, $\gamma \in D$, we see
that \mbox{$1 < \Norm{\gamma} < \Norm{\alpha}$}. By the
minimality of $\Norm{\alpha}$ we have that $\beta$ and
$\gamma$ can be written a finite product of irreducibles.
Hence the same also holds for $\alpha = \beta\gamma$, a
contradiction.
%
\end{proof}

%--------------------------------------------------------
% SOME NOTATIONS (intoductions of the subsets I, H, K of
% a gived normed domain D)
%--------------------------------------------------------
\section{Some notations}\label{notations}

Let $D$ be a domain in which every non-zero element can
be written as a product of a finite number of irreducibles.

We denote by $I$ the set of irreducibles in $D$, by $K$
the set of those elements of $D$ whose factorization
into irreducibles is essentially unique, and by $H$ the
set of those elements of $D$ which do not have this
property, \ie $H = D\setminus \left(K \cup \{0\}\right)$.

Noting that every unit can be factorized only into $0$
irreducibles, we have $D^\ast \subseteq K$. On the other
hand, because every irreducible element has only trivial
factorizations, we easily get $I\subseteq K$. We conclude
that $D^\ast \cup I\subseteq K$.

%------------------------------
% A very useful property of K
%-----------------------------
\medskip
With this notations the following useful result holds:

\begin{lem}
Let $\alpha \in K,\; \pi \in I$ such that $\pi \divides
\alpha,\, $ and\, \mbox{$\alpha_1, \alpha_2 \in D$} such
that \mbox{$\alpha = \alpha_1 \alpha_2$.}
Then $\pi \divides \alpha_1$ or $\pi \divides \alpha_2$.
\end{lem}

\begin{proof}
%
If $\alpha_1=0$ or $\alpha_2=0$, obiouvsly $\pi \divides
\alpha_1$ or $\pi \divides \alpha_2$.
So we can suppose $\alpha_1 \neq 0$ and $\alpha_2\not=0$;
we have then by our hypothesis on $D$ that $\alpha_1,
\alpha_2$ can be written as product of finite number of
irreducibles.
So $\alpha_1 = \sigma_1 \cdots \sigma_r$, $\alpha_2 =
\rho_1 \cdots \rho_s$ for appropriate $r,s \in \N$ and
$\sigma_1, \ldots, \sigma_r, \rho_1, \ldots, \rho_s
\,\in I$ (\ie irreducibles).
Since \mbox{$\alpha = \alpha_1 \alpha_2$}, and
$\alpha \in K$ we have then that:
$$
\alpha = \alpha_1 \alpha_2 =
\sigma_1 \ldots \sigma_r \rho_1 \cdots \rho_s 
$$
is the \emph{essentially unique factorization} of $\alpha$
as product of irreducibles.
Since $\pi \divides \alpha$, we can then say that $\pi$
must be associated with one of the irreducibles:
$\sigma_1, \ldots \sigma_r, \rho_1, \ldots, \rho_s$.
If $\pi$ is associated to a $\sigma_i$, then
$\pi \divides \alpha_1$, while if $\pi$ is associated to
a $\rho_j$, then $\pi \divides \alpha_2$, as desired. 
%
\end{proof}

From previous result it immediatly derives by
induction the following:

\begin{cor}\label{basic_property_of_K}
Let $\alpha \in K$, $\pi \in I$ such that
$\pi \divides \alpha$, and
\mbox{$\alpha_1, \ldots, \alpha_n \in D$} such that
$\alpha = \alpha_1 \ldots \alpha_n$. Then exists
$i \in \left\{ 1, \ldots n\right\}$ such that
$\pi \divides \alpha_i$
\end{cor}

\section{Useful results in number theory}\label{number_theory}

In this section we will present, briefly and without proof, some
classical results in number theory, that we will use extensively
later in our work.

Proofs and background information about these results can be found
in~\cite{H&W}, \cite{Childs} and~\cite{Dav}\footnote{%
except for the proof of the Dirichlet's theorem, which is very
difficult.}.

We will also state and prove a simple technical lemma, which will be
of paramount importance in chapter~\ref{notUFD}.

\begin{thmen}[Chinese Remainder Theorem]
Let $m_1, m_2, \ldots, m_n$ be pairwaise coprime positive integers, and
let $a_1, a_2, \ldots, a_n$ be arbitrary integers; then there exists a
solution $x = x_0\in\N$ of the system:

\begin{equation*}
%
\left\{
\begin{array}{ll}
x \congruent a_1 & \pmod{m_1} \\
x \congruent a_2 & \pmod{m_2} \\
\quad \vdots & \quad\quad \vdots \\
x \congruent a_n & \pmod{m_n} \\
\end{array}
\right.
\end{equation*}

Morever, any other integer solution of this system is given by\/
$x = x_0 + k m_1 m_2 \cdots m_n$, for arbitrary $k\in\Z$.
\end{thmen}

\begin{defnen}[Quadratic residues]
Let $p$ be a prime. Then $x \in \Z$ is said to be a\/
\emph{quadratic residue of $p$\/} if there exists\/ $y\in\Z$\/ such
that\/ $a \congruent y^2 \pmod p$, and a\/ \emph{quadratic non-residue
of $p$\/} if no such $y\in\Z$ exists; if $x$ is a quadratic residue of
$p$ and\/ $x \notcongruent 0 \pmod p$, then\/ $x$\/ is said to be a\/
\emph{proper quadratic residue of $p$\/}.

More generally, given an integer $m > 1$, we say that $x\in\Z$ is
a\/ \emph{quadratic residue of $m$\/} if\/ $x \congruent y^2 \pmod m$\/
for a suitable $y\in\Z$; on the contrary, $x\in\Z$ is said to be a\/
\emph{quadratic non-residue of $m$\/} if\/ $x \notcongruent y^2 \pmod m$\,
for every\/ $y\in\Z$.
\end{defnen}

\begin{defnen}[Legendre symbol]
Let\/ $p$\/ be a prime and\/ $x$\/ an integer; the\/
\emph{Legendre symbol\/} of\/ $x$\/ over\/ $p$\/ is defined as:
\begin{equation*}
\Leg{x}{p} :=
\left\{ \begin{array}{ll}
 +1 & \textrm{if \:} x \textrm{\, is a proper quadratic residue of } p\\
 -1 & \textrm{if \:} x \textrm{\, is a quadratic non-residue of } p\\
~~0 & \textrm{if \:} x \congruent 0\!\!\! \pmod p\\
\end{array}
\right.
\end{equation*}
\end{defnen}

The most important properties of quadratic residues and of the Legendre
symbol are summarized in the following proposition:
\begin{thmen}
Let $p$, $q$ be distinct odd primes, $a$, $b$ integers. Then:
\begin{itemize}
 \item[{\rm (i)\;}]
   $\Leg{a^2}{p} = 1$\, if $p \notdivides a$
 \item[{\rm (ii)\;}]
   $\Leg{a}{p} = \Leg{b}{p}$\, if\/ $a \congruent b\! \pmod p$
 \item[{\rm (iii)\;}]
   $\Leg{ab}{p} = \Leg{a}{p} \Leg{b}{p}$
 \item[{\rm (iv)\;}]
   $\Leg{-1}{p} = {(-1)}^{\frac{p-1}{2}}$,\, or more explicitly:
   \begin{itemize}
     \item[$\bullet$]
       $\Leg{-1}{p} = +1$\/ if\/ $p \congruent 1 \pmod 4$,
     \item[$\bullet$]
       $\Leg{-1}{p} = -1$\/ if\/ $p \congruent 3 \pmod 4$.
   \end{itemize}
\item[{\rm (v)\;}]
   $\Leg{2}{p} = {(-1)}^{\frac{p^2-1}{8}}$,\, or more explicitly:
   \begin{itemize}
     \item[$\bullet$]
        $\Leg{2}{p} = +1$\/ if\/ $p \congruent 1\textrm{ or }7 \pmod 8$,
     \item[$\bullet$]
        $\Leg{2}{p} = -1$\/ if\/ $p \congruent 3\textrm{ or }5 \pmod 8$.
   \end{itemize}
\item[{\rm (vi)\;}]
   $\Leg{p}{q} = {(-1)}^{\frac{(p-1)(q-1)}{4}}\Leg{q}{p}$
   \ \:\emph{(quadratic reciprocity law)},\; or more explicitly:
  \begin{itemize}
     \item[$\bullet$]
        $\Leg{p}{q} = \Leg{q}{p}$\/ if\/ $p \congruent 1 \pmod 4$\/ or\/ 
          $q \congruent 1 \pmod 4$,
     \item[$\bullet$]
        $\Leg{p}{q} = -\Leg{q}{p}$\/ if\/
          $p \congruent q \congruent 3 \pmod 4$.
  \end{itemize}
\end{itemize}

\end{thmen}

\smallskip

Here is a simple result on quadratic residues that will be useful
in chapter~\ref{notUFD}:
\begin{lem}\label{helper_lem_on_quadratic_residues}
Let $m > 1$ integer, $p$ a prime such that $p \divides m$, and $x \in \Z$.
If $x$ is a quadratic non-residue of $p$, then it is also a quadratic
non-residue of $m$.
\end{lem}
\begin{proof}
Argue by contradiction, assuming that there exists a suitable $y \in \Z$
such that \mbox{$x \congruent y^2 \pmod m$}; then, since $p \divides m$,
it's also \mbox{$x \congruent y^2 \pmod p$}, which contradicts the fact
that $x$ is a quadratic non-residue of $p$.
\end{proof}

\medskip

We'll need also the classical Dirichlet's theorem on primes in
arithmetic progressions:

\begin{thmen}[Dirichlet's thorem on primes in arithmetic progressions]
~Let\/ $a$, $b$\, be integers with\/ $a>0$\/ and\/ $\gcd{a}{b} = 1$.
Then the arithmetic progression $\left\{a n + b : \, n\in\N\right\}$
contains infinite primes.
\end{thmen}

Let's finally see a quite simple and technical lemma that will
be very useful in chapter~\ref{notUFD}.

\begin{lem}\label{basic_technical_lem}
Suppose given\, $n \in \N$\, such that \,$n \geq 1$,\,
$ p_1,\, p_2,\, \ldots,\, p_n$\, distinct primes,\,
$t \in \Z$\, such that\, $\forall\, i \in
\{1, \ldots, n\}:\, p_i \notdivides t$,\, $a \in \Z$\, such that\,
$\gcd{a}{t} = 1$ and\, $m \in \N$\, with\, $m \leq n$.

Then there exist infinitely many primes $q$ such that:

\begin{itemize}
\item $\:q$ is a proper quadratic residue of\, $p_i$\, for\,
      $i = 1,\, \ldots\, m$
\item $\:q$ is a quadratic non-residue of\, $p_j$\, for\,
      $j = m + 1,\, \ldots\, n$
\item $\:q \congruent a\ \mathrm{(mod\ } t \mathrm{)}$
\end{itemize}

%
\end{lem}

\begin{proof}
%
As a first step, take
\,$\xi_{m+1},\, \xi_{m+2},\, \ldots,\, \xi_n \in \Z$\,
quadratic non-residue respectively of
\,$p_{m+1},\, p_{m+2},\, \ldots,\, p_n$.

Consider then the following system of linear congruences:
\medskip
\begin{equation}\label{congruences_system}
%
\left\{
\begin{array}{ll}
x \congruent 1 & \pmod{p_1} \\
x \congruent 1 & \pmod{p_2} \\
\quad \vdots & \quad\quad \vdots \\
x \congruent 1 & \pmod{p_m} \\
x \congruent \xi_{m+1} & \pmod{p_{m+1}}\\
\quad \vdots & \quad\quad \vdots \\
x \congruent \xi_n & \pmod{p_n} \\
x \congruent a & \pmod{t} \\
\end{array}
\right.
\end{equation}
%

\bigskip
Write\, $M:= p_1 p_2 \cdots p_n t$.
Since the $p_j$ are pairwise coprime and all coprime with $t$,
we know from Chinese remainder theorem that \mbox{system
(\ref{congruences_system})} is soluble, and that if
$x_0 \in \Nzero$ is its minimal positive solution then
$x_k := x_0 + kM$ is also a solution for every $k \in \Z$.

Now, $\gcd{x_0}{M} = 1$, since:
%
\begin{itemize}

\item
for every $i \in  \{ 1,\, 2,\, \ldots, \, m \}$, is\,
$x_0 \congruent 1 \pmod{p_i}$, so that\, $\gcd{x_0}{p_i} = 1$;

\item
for every $j \in \{m+1,\, m+2,\,\ldots,\,n\}$, is\,
$x_0 \congruent \xi_j \pmod{p_j}$, so that\, $\gcd{x_0}{p_j} = 1$
(since $p_j \notdivides \xi_j$ \,as\, $\xi_j$ is a
quadratic non-residue of $p_j$);

\item
$x_0 \congruent a \pmod t$\, and\, $\gcd{a}{t} = 1$,\,
so that\, $\gcd{x_0}{t} = 1$.

\end{itemize}

Thus, from Dirichlet theorem, we have that the set:
$$
\Gamma:= \left\{x_k:\, k \in \N\right\} =
\left\{x_0 + kM:\, k \in \N\right\} \subseteq \N
$$
contains infinetely many primes $q$; each of this primes
satisfies system~(\ref{congruences_system}), so that:
%
\begin{itemize}
%
\item
$q \congruent a \pmod t$;
%
\item $\forall\:i \in \{1, 2,\,\ldots\, m\}:\:q \congruent
1^2 \pmod{p_i}$,\, so that $q$ is a quadratic residue of
$p_i$;
%
\item $\forall\:j \in \{m+1, m+2,\,\ldots\, n\}:\:q \congruent
\xi_j \pmod{p_i}$\, with\, $\xi_j$ quadratic non-residue
of $p_j$, so that $q$ is a quadratic non-residue of $p_j$;
%
\end{itemize}
%
and our claim follows.
\end{proof}

%********************************************************************

%-------------------------------------------
% DEFINITION OF THE SUBRINGS OF THE COMPLEX
% FIELD C THAT WE ARE GOING TO DEAL
%-------------------------------------------
\section{Some subrings of\enskip$\C$}\label{C_subrings}

We are now going to define the subrings of $\C$ which
will be considered in this work. They can be subdivided
into two classes, namely:

\begin{enumerate}
 \item the rings of the form $\ZZ{m}$, with $m$ a
       non-square in $\Z$;
 \item the rings of algebraic integers of quadratic
       extensions of $\Q$.
\end{enumerate}

%-------------------------
% DEFINITION OF Z[sqrt(m)]
%-------------------------
\begin{defn} Given $m$ in $\Z$ we define
$\ZZ{m} := \{ a + b\sqrt m:\, a,b \in \Z \}$.
\end{defn}
%

%------------------------------------------------------------
% PRELIMINARIES FOR THE DEFINITIONS OF THE RINGS OF CLASS (2)
%------------------------------------------------------------
More attention is deserved by the rings which belong to the above class
$(2)$, whose definition is somehow more elaborate and complex.
All definitions and preliminar properties which we are now going to
explain (quite schematically and without proofs) are analyzed in a more
complete and organic form in chapter XIV of~\cite{H&W}.

We begin recalling the definitions of ``algebraic number'',
``algebraic integer'', ``algebraic field'' and ``quadratic
field''.
\begin{itemize}

%---------------------------------
% DEFINITION OF "ALGEBRAIC NUMBER"
%---------------------------------
\item A complex number $z \in \C$ is said to be an
\emph{algebraic number\/} if there exist
\mbox{$n \in \Nzero$}\, and\, $a_0, a_1, \ldots, a_n \in \Z$\,
such that\, $a_n \neq 0$\, and\,
$a_n z^n + a_{n-1} z^{n-1} + \ldots + a_1z + a_0 = 0$.
Equivalently, $z \in \C$ is algebraic if an only if
there exists a non-costant polynomial $P(x) \in \Z[x]$
such that $P(z) = 0$.

%----------------------------------
% DEFINITION OF "ALGEBRAIC INTEGER"
%----------------------------------
\item A complex number $z \in \C$ is said to be an
\emph{algebraic integer\/} if there exist
\mbox{$n \in \Nzero$}\, and\,
$a_0, a_1, \ldots, a_{n-1} \in \Z$\, such that\,
$z^n + a_{n-1}z^{n-1} + \cdots + a_1z + a_0 = 0$.
Equivalently, $z \in \C$ is an algebraic integer if an
only if there exists a non-costant monic polynomial
$P(x) \in \Z[x]$ such that $P(z) = 0$. It can be proved with
standard algebraic methods that $z \in \C$ is an algebraic
integer if and only if it is algebraic over $\Q$ and its
minimal polynomial over $\Q$ has rational integer
coefficients (see for example Theorem (236) in chapter XIV
of~\cite{H&W}\footnote{which, as a matter of fact, is given
in a slighty different but equivalent form.}).

%--------------------------------
% DEFINITION OF "ALGEBRAIC FIELD"
%--------------------------------
\item For every $\vartheta \in \C$, we define
\begin{displaymath}
\Q\left(\vartheta\right):= \left\{ \frac{P(\vartheta)}{Q(\vartheta)}:
\ P(x),\, Q(x) \in \Q[x]\ \textrm{\,and\,}\ Q(\vartheta) \neq 0 \right\}
\end{displaymath}
and call it the simple extension of $\Q$ by $\vartheta$.

This definition is consistent, since it can easily be
proved that $\Q\left(\vartheta\right)$ is the smallest subfield
of $\C$ which contains $\vartheta$. If $\vartheta$ is
algebraic, we call it the \emph{algebraic field\/} on
$\vartheta$.

%--------------------------------
% DEFINITION OF "QUADRATIC FIELD"
%--------------------------------
\item If $\xi \in \C$ is an irrational algebraic number
which satisfies an equation of degree two with coefficients in $\Z$,
then $\xi$ is said to be \emph{quadratic\/}. A \emph{quadratic field\/}
(or \emph{quadratic estension of $\,\Q$\/}) is an algebraic field of the
form $\Q\left(\xi\right)$, with $\xi \in \C$ quadratic.

\end{itemize}

%------------------------------------------------------------
% EXPLICIT CHARACTERIZATION OF h(sqrt(m)) AND CLASSIFICATION
% OF RINGS OF QUADRATIC ALGEBRAIC INTEGERS
%------------------------------------------------------------
\begin{defn}\label{h(sqrt(m))-definition}
For every squarefree $m \in \Zzero$ we define:
$$
\hh{m} :=
\left\{ \begin{array}{ll}

\left\{ x + y\sqrt{m} :\ x,y \in \Z \right\} &
    \mathrm{if}\ m \congruent 2\ \mathrm{or}\ 3
    \mathrm{\ (mod\ 4)}

\\ % this two newlines is useful for 
\\ % having a right vertical space

\left\{ \frac{x + y\sqrt{m}}{2}:\ x,y \in \Z\ \,
   \mathrm{and}\ \, x \congruent y
   \mathrm{\ (mod\ 2)}\right\} &
   \mathrm{if}\ m \congruent 1\ \mathrm{(mod\ 4)} 

\end{array} \right.
$$
\end{defn}

For each algebraic number $\xi \in \C$ we define:
$$
A_{\xi}:= \left\{ z \in \C:\ z
\textrm{ algebraic integer and } z \in \Q\left(\xi\right) \right\}.
$$

According to the terminology introduced above, it can be
proved that the collection of sets:
$$
\left\{A_{\xi}:\ \xi \in \C \textrm{ quadratic }\right\}
$$
is precisly the collection of all the rings of the form $\hh{m}$
for $m \in \Zzero$ squarefree.

\medskip
Finally we have:
\begin{thm}
For every $m \in \NotSquare$,\, $\ZZ{m}$ is a normed domain.
Moreover, for every squarefree $m \in \Zzero$,
$\hh{m}$ is a normed domain.
\end{thm}

\begin{proof}
%
First, it is easy to prove that $\ZZ{m}$ is a normed domain.

From the equality
\mbox{$\hh{m} = \ZZ{m}$}, holding for $m \congruent 2 \textrm{ or }
3 \pmod 4$, we immediately have that $\hh{m}$ is a normed domain if
$m \congruent 2 \textrm{ or } 3 \pmod 4$.

Hence we have only to deal with the case $m \congruent 1 \pmod 4$,
when it results:
$$\hh{m} = \left\{ \frac{x + y\sqrt{m}}{2}:
\ x,\,y \in \Z \textrm{ \,and\, }
x \congruent y\!\!\! \pmod 2 \right\}.$$

Let\, $x,y,X,Y \in \Z$\, such that\, $x \congruent y \pmod 2$\, and\,
$X \congruent Y \pmod 2$;\, we have:

\begin{itemize}

\item
$\left(\dfrac{x + y \sqrt{m}}{2}\right) +
\left(\dfrac{X + Y \sqrt{m}}{2}\right)
= \dfrac{(x + X) + (y + Y) \sqrt{m}}{2}\,
\in \hh{m}\,$,\: since\\[6pt]
$x + X \congruent y + Y \pmod 2$;
\vspace{4pt}
\item
$\left(\dfrac{x + y \sqrt{m}}{2}\right)
\left(\dfrac{X + Y \sqrt{m}}{2}\right)
= \dfrac{(xX + myY) + (xY + Xy) \sqrt{m}}{4}
= $\\[5pt]
$ =
\dfrac{a + b\sqrt{m}}{2} \in \hh{m}$,\: since:
\begin{itemize}
   %
   \item[(1)] $ xY +  Xy \congruent xX + xX =
   2xX \congruent 0  \pmod 2$, so that:\\
   \mbox{$b = \frac{xY + Xy}{2} \in \Z$;}
   %
   \item[(2)] $ xX +  myY \congruent xX + yY
   \congruent xX + Xx  = 2Xx \congruent 0 \pmod 2$, so that:\\
   \mbox{$a = \frac{xX + myY}{2} \in \Z$;}
   %
   \item[(3)] $ 2a - 2b = (xX + myY) - (xY + Xy) = xX + myY - xY - Xy
   \congruent xX + yY - xY -Xy \pmod 4 \then 2a - 2b \congruent
   x(X - Y) - y(X - Y) = (x - y)(X - Y) \congruent 0 \pmod 4$\,
   (since $2\divides (x - y)$ and $2 \divides (X - Y)$), so that:\\
   $2a \congruent 2b\ \pmod 4 \then a \congruent b \pmod 2$;
\end{itemize}

\item
$1 = \frac{2 + 0\sqrt{m}}{2}\, \in \hh{m}$\,
(as obiouvsly $2 \congruent 0 \pmod 2$);

\item
if $\alpha \in \hh{m}$, then $\alpha = \frac{x + y \sqrt{m}}{2}$,
for $x,y \in \Z$ such that $x \congruent y \pmod 2$; so\,
$\cg{\alpha} = \frac{x - y \sqrt{m}}{2} \in \hh{m}$\, since\,
$x \congruent y \congruent -y \pmod 2$,\, and also\,
$\alpha \cg\alpha = \left(\frac{x^2 - my^2}{4}\right)\in \Z$,\, since:

~~$x \congruent y \pmod 2 \then x^2 \congruent y^2 \pmod 4 \then
x^2 - my^2 \congruent x^2 - y^2 \congruent 0 \pmod 4$.

\end{itemize}

The first three points of the precedent list prove that $\hh{m}$ is a
subring of $\QQ{m}$, while the last point proves that $\hh{m}$ is normed.
%
\end{proof}
