\documentclass[11pt,a4paper]{article}

\usepackage{amsmath,amsfonts,amsthm,amssymb}

\begin{document}

\baselineskip4ex
\oddsidemargin 0cm
\evensidemargin 0cm
\textwidth 14.7cm
\topmargin -1.0cm
\headheight 1cm
\headsep 1.2cm
\textheight 22.5cm
\parindent 0.0em

% "minus zero"
\newcommand{\minz}{\setminus\left\{0\right\}}

% natural numbers
\newcommand{\N}{\mathbb{N}}
\newcommand{\Nzero}{\N\minz}
% integers
\newcommand{\Z}{\mathbb{Z}}
\newcommand{\Zzero}{\Z\minz}
% rational numbers
\newcommand{\Q}{\mathbb{Q}}
\newcommand{\Qzero}{\Q\minz}
% real numbers
\newcommand{\R}{\mathbb{R}}
\newcommand{\Rzero}{\R\minz}
% complex numbers
\newcommand{\C}{\mathbb{C}}
\newcommand{\Czero}{\C\minz}
% set of non-square integers
\newcommand{\NotSquare}{\Z\setminus\{h^2: h\in\Z\}}

%%
%% shortands for recurring normed domains
%%

% A "dirty hack" to give some parenthes the rigth size.
% NOTE: phnatom is a fragile command, so protect it.
\newcommand{\hh}[1]{h\big(\sqrt{\protect\phantom{m}\!\!\!\!\!#1}\,\big)}
\newcommand{\hhm}[1]{h\big(\sqrt{-\protect\phantom{m}\!\!\!\!\!#1}\,\big)}
\newcommand{\QQ}[1]{\Q\big(\sqrt{\protect\phantom{m}\!\!\!\!\!#1}\,\big)}
\newcommand{\QQM}[1]{\Q\big(\sqrt{-\protect\phantom{m}\!\!\!\!\!#1}\,\big)}
\newcommand{\ZZ}[1]{\Z\big[\sqrt{\protect\phantom{m}\!\!\!\!\!#1}\,\big]}
\newcommand{\ZZM}[1]{\Z\big[\sqrt{-\protect\phantom{m}\!\!\!\!\!#1}\,\big]}

% symbol for conjugate (alias of "\overline")
\newcommand{\cg}[1]{\overline{#1}}

% norm operator
\newcommand{\Normop}[0]{\mathcal{N}}
% norm
\newcommand{\Norm}[1]{\Normop\left(#1\right)}
% modulus of a real or complex number
\newcommand{\abs}[1]{\left|#1\right|}

% Legendre symbol
\newcommand{\Leg}[2]{\left(\frac{#1}{#2}\right)}

% congruence relation between integers
\newcommand{\congruent}{\equiv}
\newcommand{\notcongruent}{\not\equiv}

% relation of division in domains
\newcommand{\divides}{\mid}
\newcommand{\notdivides}{\nmid}

% great common divisor and least common multiple
\renewcommand{\gcd}[2]{\mathrm{g.c.d.}\left({#1},\,{#2}\right)}
\newcommand{\mgcd}[1]{\mathrm{g.c.d.}\left({#1}\right)}
\newcommand{\lcm}[2]{\mathrm{l.c.m.}\left({#1},\,{#2}\right)}
\newcommand{\mlcm}[1]{\mathrm{l.c.m.}\left({#1}\right)}

% "id est" with right spacing
\newcommand{\ie}{\text{i.e.\ }}

% "elegant" implication/bimplication symbols
\newcommand{\then}{\:\Longrightarrow\:}
\renewcommand{\iff}{\:\Longleftrightarrow\:}

\newtheorem*{xthm}{Theorem}

\begin{xthm}
  Se $x$ \`e un elemento avente fattorizzazione unica in irriducibili e
  $p$ \`e un irriducibile che divide $x$, allora se $x = x_1 x_2$ si ha
  $p \divides x_1$ o $p \divides x_2$.
\end{xthm}

\textsf{Dimostrazione.}\\[3pt]
  \hspace*{8pt}Procediamo per assurdo, assumendo che $p \notdivides x_1$
    e $p \notdivides x_2$.
  %
  \begin{itemize}
    
    \item
      Siccome \,$p \divides x$, abbiamo \,$x = pX$; fattorizzando $X$
      come \,$X = P_1 P_2 \cdots P_r$\, (non ci interessa se questa
      fattorizzazione sia unica o meno), abbiamo:
        $$ x = p P_1 P_2 \cdots P_r $$
      che \`e una fattorizzazione in irriducibili di $x$ contenente $p$.
    
    \item
      Poich\`e \,$p \notdivides x_1$\, e \,$p \notdivides x_1$, abbiamo
      che essi ammettono due fattorizzazioni del tipo:
      \begin{eqnarray*}
        x_1 & \!=\! & q_{1,1} q_{1,2} \cdots q_{1,s_1}\\
        x_2 & \!=\! & q_{2,1} q_{2,2} \cdots q_{2,s_2}
      \end{eqnarray*}
      ove nessun $q_{i,j}$ \`e associato a $p$; ma allora:
      $$
        x = x_1 x_2 =  q_{1,1} q_{1,2} \cdots q_{1,s_1}
                      q_{2,1} q_{2,2} \cdots q_{2,s_2}
      $$
      \`e una fattorizzazione in irriducibili di $x$ ove nessun fattore
      irriducibile \`e asssociato a $p$.
    
    \item
      Ma ci\`o contraddice il fatto che $x$ ammetta una scomposizione in
      fattori essenzialmente unica. \qed
  
  \end{itemize}

\end{document}
     
