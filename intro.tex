\addcontentsline{toc}{chapter}{Introduction}
\chapter*{Introduction}
\markboth{Introduction}{Introduction}

\section{Prerequisites}
In this work, we'll assume that the reader has a good knowledge of
undergraduate algebra and of elementary number theory, especially the
elementary theory of congruences and quadratic residues.

These topics are explained in great details in~\cite{H&W}, \cite{Childs}
and~\cite{Sha}.

\section{Background, scope and goals of this work}
The aim of the present work is to study certain subrings of\, $\C$\, with
respect to the property of being (or not being) UFD%
\footnote{%
by an UFD we mean a Unique Factorization Domain, \ie a domain in which
every element $\neq 0$ can be written, in an essentialy unique way, as
a product of a finite number of irreducibles.%
}. We will consider essentially the following classes:

\smallskip

~$\bullet$ the rings $\ZZ{m}$, where $m\in\Z$ is a non-square in $\Z$;

~$\bullet$ the rings $\hh{m}$, \ie the rings of algebraic integers of
           quadratic extensions of $\Q$.
\smallskip

These issues are part of a broader class of problems, interesting both
for their theoretic aspects and their applications, and which (at least in
their embrional form) date back to Gauss and Euler.

These problems have been extensively studied in the last two centuries,
and have led to many deep and interesting results (some of which are
described in~\cite{S&T} and in chapters~{\small(XII)}, {\small(XIV)}
and~{\small(XV)} of~\cite{H&W}).

In this work, we will \emph{not\/} study the mentioned problems in their
full generality; what we will do is to show how it is possible to obtain
partial yet interesting results about those problems by means of just
elementary methods. For more complete results and more powerful methods
the reader can refer to~\cite{S&T}.

\section{Overview of results}
\smallskip
\subsection*{Overview of chapter~\ref{preliminary}}
%
In chapter~\ref{preliminary}, some preliminary concepts, definitions and
results will be presented, most of which are classical and well known.

Particularly important are the definition~(\ref{normedDomain}), the
lemma~(\ref{basic_technical_lem}) and the whole section~\ref{C_subrings}.

\subsection*{Overview of chapter~\ref{UFD}}
%
In chapter~\ref{UFD}, we will give a simple positive criterion for the
uniqueness of factorization, \ie a sufficient condition that, when
satisfied, ensures that a given domain $\ZZ{m}$ or $\hh{m}$ is an UFD\@.
This criterion will be proved by means of elementary methods, and will
be used to find some examples of interesting UFDs, among which is the
set $G = \ZZM{1}$ of gaussian integers.

The most significant results of this chapter are the enunciation of
properties~(\ref{property_sharp}), (\ref{property_sharp_2})
and~(\ref{property_sharp_weaker}), the
theorem~(\ref{property_sharp_implies_UFD}), and the retrieval of the
well-known class of norm-euclidean domains through an unusual path.

\subsection*{Overview of chapter~\ref{notUFD}}
%
In chapter~\ref{notUFD}, we will give some negative criterions for the
uniqueness of factorization, \ie a set of sufficient conditions such
that, when any of them is satisfied, ensure that a given domain $\ZZ{m}$
or $\hh{m}$ is not an UFD\@.
These criterions will be used to show that a large fraction of
the $\ZZ{m}$ and $\hh{m}$ domains are not UFDs.

The proofs of important results of this chapter will use extensively
the lemma~(\ref{basic_technical_lem}) proved in chapter~\ref{preliminary}
and the law of quadratic reciprocity.

The most outstanding results of this chapter are five ``summarizing''
results:
theorem~(\ref{definitive_result_on_uniqueness_in_Z[sqrt(-m)]}),
theorem~(\ref{h(sqrt(-m))_UFD_constraint_m}),
corollary~(\ref{second_theorem_on_uniqueness_in_complex_h(sqrt(-p))}),
theorem~(\ref{DEEPER}) and
theorem~(\ref{Z_DEEPER}).

\subsection*{Overview of chapter~\ref{limits}}
%
In this chapter, we'll explore some limits of the methods presented in
chapter~\ref{UFD}.

First and foremost, in theorem~(\ref{euclidean_equivalent_quasi-euclidean})
we'll show the equivalence between the property~(\ref{property_sharp})
and the apparently stronger property~(\ref{property_sharp_weaker})
(both defined in chapter~\ref{UFD}).

Finally, we'll show (extending a result presented in
chapter~{\small (XIV)} of~\cite{H&W}) that the great majority of
the domains $\hh{m}$ are not norm-euclidean.

\section{Thanks}

The author wishes to thank Professor Maria Clara Tamburini for her
helpfulness, continuous support and good advices.

A deep thank is also due to Professor Maxim Vsemirnov, who read, discussed
and corrected all the drafts of the present work, and provided a lot of
useful suggestions, ideas and improvements.
