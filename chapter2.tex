\chapter{Some UFDs of algebraic integers}\label{UFD}

%----------------------------------
% FIRST SECTION: gaussian integers
%----------------------------------
\section{Unique factorization in the ring of
Gaussian integers}\label{gaussians}

%---------------------------------------------
% DEFINITION OF THE RING OF GAUSSIAN INTEGERS
%---------------------------------------------
\begin{defn}\label{def1}
The ring $G$ of Gaussian integers consists of
all complex numbers $x+iy$ where $x,y$ are
rational integers, \ie
% ==> it's essential not to leave a blank line here!
\begin{equation}\label{gauss}
G:= \Z[i] = \left\{x + iy :\ x,y \in \Z \right\}
\end{equation}
\end{defn}
Clearly $G$ is a normed domain in the sense of
definition~(\ref{normedDomain}), with
$G^\ast =\{1,-1,i,-i\}$.

\bigskip
First we make an important remark: in the rest
of this section we will denote by\, $\Norm{\cdot}$
\,the usual complex norm\, $\Normop:\, \C
\longrightarrow \left[\, 0, +\infty \right[$;
however this will cause no confusion since, as
$G = \Z[i] \subseteq \QQM{1}$, we easily have that the
complex norm and the norm in the sense of
definition~(\ref{normdef}) coincide for $G$.

\bigskip
The aim of this Section is to prove that $G$
is a UFD, without using the well known fact that
$G$ is an euclidean domain. To this purpose it will
be useful the following result.

%--------------------------------------------------
% VERY IMPORTANT LEMMA (ANCESTOR OF "PROPERTY(#)")
%--------------------------------------------------
\begin{lem}\label{lemmaGaussian} Let
$\tau \in \Czero$ be such that $\Norm{\tau}
\leq 1$. Then there exists $\epsilon \in G^\ast$
such that $\Norm{\tau + \epsilon} < 1$.
\end{lem}

\begin{proof}
%
First note that, for every $\epsilon_1, \epsilon_2
\in G^\ast$, it results \;$\Norm{\epsilon_1\tau} \leq 1$\,
(since $\Norm{\epsilon_1\tau} = \Norm{\tau}$\, and\,
$\Norm{\tau} \leq 1$);\; moreover, if
$\Norm{\epsilon_1\tau + \epsilon_2} < 1$, we have then:

$\Norm{\tau + \epsilon} < 1 \textrm{,\; with \,} \epsilon:=
\epsilon_1^{-1}\epsilon_2 \in G^\ast$.

So we need to prove our assert only for $\epsilon\tau$,
where $\epsilon \in G^\ast$ is an arbitrary unit of $G$.

We can now use this observation to simplify our problem.
Let $\tau = a + ib$. Multiplying $\tau$ by $i\in G^\ast$,
if necessary, we may assume $\abs{a} \geq \abs{b}$. Moreover,
with a further multiplication by $\pm 1 \in G^\ast$ (with sign
appropriatly chosen), we may assume $ a \geq 0 $.
Summarizing we can assume without loss of generality that
$a \geq \abs{b}$.

We distinguish now 2 cases:

\smallskip

\textbf{Case 1:} $a > \abs{b}$. Clearly $a \not= 0$, and
$a \leq 1$ (since $a^2 + b^2 = \Norm{\tau} \leq 1$); hence,
letting $\epsilon:= -1 \in G^\ast$, we have:\,
$\Norm{\tau + \epsilon} -1 = \Norm{a + ib - 1} - 1 =
(a - 1)^2 + b^2 - 1 = a^2 + b^2 - 2a < 2a^2 - 2a =
2a(a - 1)\leq 0$.
Thus $\Norm{\tau + \epsilon} < 1$.

\smallskip

\textbf{Case 2:} $a = \abs{b}$. We have then $a^2 = b^2$,
and so $2a^2 = a^2 + b^2 = \Norm{\tau} \textrm{ with }
\Norm{\tau} \leq 1 \textrm{ and } \Norm{\tau} \not= 0
\then 0 < a \leq \frac{1}{\sqrt 2} < 1 \then a(a - 1) < 0$.
Hence, letting $\epsilon:= -1 \in G^\ast$, we have:\,
$\Norm{\tau + \epsilon} -1 = \Norm{a + ib - 1} - 1 =
(a - 1)^2 + b^2 - 1 = a^2 + b^2 - 2a = 2a^2 - 2a =
2a(a - 1) < 0$. Thus $\Norm{\tau + \epsilon} < 1$.
% 
\end{proof}

%--------------------------------------------------
% A GEOMETRIC INTERPRETATION OF THE PREVIOUS LEMMA
% (due to Maxim Vsemirnov)
%--------------------------------------------------
\medskip
The following picture provides a geometric interpretation of the
previous lemma, as it shows visually that the union of the four
unitary \emph{open\/} circles with center respectively
\,\mbox{$(-1,0)$, $(0,-1)$, $(1,0)$, $(0,1)$}\,
covers completely the unitary \emph{closed\/} circle centered
in $(0,0)$, deprived of the center \,(\,\ie the point $(0,0)$\,).

\vspace{6pt}

\begin{picture}(360,160)(-50, +20)
\put(150, 60){\circle{40}}
\put(150, 80){\circle{40}}
\put(150, 100){\circle{40}}
\put(130, 80){\circle{40}}
\put(170, 80){\circle{40}}
\put(70, 80){\vector(1,0){170}}
\put(150, 10){\vector(0,1){150}}
\put(230,70){\mbox{\textit{x}}}
\put(155,150){\mbox{\textit{y}}}
\end{picture}

\vspace{36pt}
Now we return to our main topic.
Our central theorem may be enunciated as follows:

%-----------------------------------------
% THE RING OF GAUSSIAN INTEGERS IS AN UFD
%-----------------------------------------
\begin{thm}\label{gaussUFD}
The ring $G$ is an UFD.
\end{thm}

To prove this theorem, we need some preliminary
observations and one further lemma.

%-----------------------------------------------------
% PRELIMINARY LEMMAS AND OBSERVATIONS USEFUL TO PROVE
% THAT GAUSSIAN INTEGERS FORM AN UFD
%-----------------------------------------------------
First, in the notation of section~\ref{notations}, our aim is to show
that $H = \emptyset$, \ie the set of elements of $G$ which do not admit
an essentially unique factorization is empty.

We will argue by contradiction assuming that
$H \neq \emptyset$.

Recalling that $I \cup G^\ast \subseteq K$, we have that
any factorization into irreducible of an arbitrary element
of $H$ consists of at least two (not necessarily distinct)
factors.

As $G$ is a normed domain, we know from
lemma~(\ref{factorizationIntoIrreducibles}) that
every $\alpha \in G\minz$ can be written as a product
of a finite number of irreducible.

By the assumption $H \not= \emptyset$, we may choose
$\xi_0\in H$ such that $\Norm{\xi_0} = \min{\Norm{H}}$,
where (of course) $\Norm{H}:=\left\{\,\Norm{h}:\ h \in H\right\}$.

By definition of $H$ we have, in particular,
\begin{equation}\label{two_factorizations_1}
\xi_0 = \pi_1\pi_2\cdots\pi_r = \sigma_1\sigma_2\cdots\sigma_s
\end{equation}
with $r,s \geq 2$, for suitable $\pi_k,\sigma_h \in I$
which make the two factorizations essentially distinct.

Our proof will be based on the following two Lemmas.

\begin{lem}\label{lemmaFactorsNotAssociate}
Referring to the equation \eqref{two_factorizations_1},
we have that for every\/ $i,j \in \N$\/ with\/ $1 \leq i \leq r$\/ and\/
$1 \leq j \leq s$,\, $\pi_i$\/ and\/ $\sigma_j$\/ are not associated.
\end{lem}

\begin{proof}
%
Assume that our claim is false. Then, up to a reordering
of the factors, we may suppose that $\sigma_1$ is
associate to $\pi_1$, \ie $\sigma_1 = \epsilon \pi_1$,
with $\epsilon \in G^\ast$.
Letting $\xi := \pi_2 \cdots \pi_r$ and simplifing
\eqref{two_factorizations_1} by $\pi_1$, we get:
\begin{equation}\label{two_factorizations_1_semplified}
\xi = \pi_2\cdots\pi_r = (\epsilon\sigma_2)\cdots\sigma_s.
\end{equation}
Since the two factorizations of $\xi_0$ in
\eqref{two_factorizations_1} are essentially distinct,
and since $\epsilon$ is a unit of $G$, it can easily be
shown that also the two factorizations of $\xi$ in
\eqref{two_factorizations_1_semplified} are essentially
distinct, \ie that $\xi \in H.$
Noting that we have also $\Norm{\xi} < \Norm{\xi_0}$
(since $\xi_0 = \pi_1\xi$ and $\Norm{\pi_1} > 1$) and
$\Norm{\xi_0} = \min{\Norm{H}}$
(by definition of $\xi_0$), we get the desired
contradiction.
%
\end{proof}

It's also clear that we may suppose without loss of
generality that $\Norm{\pi_1} \le \Norm{\sigma_1}$;
at this stage we can finally give:

%---------------------------
% PROOF THAT G IS AN UFD
%---------------------------
\begin{proof}[Proof of theorem~(\ref{gaussUFD}).]

Let \,$\tau:= \frac{\pi_1}{\sigma_1}$;\, clearly
$\Norm{\tau} \leq 1$ and $\tau \neq 0$, so we have by
lemma~(\ref{lemmaGaussian}) that:
%
\begin{equation*}
\exists\ \epsilon \in G^\ast :\ \Norm{\tau + \epsilon} < 1,
\quad\!\! \textrm{\ie such that:} \quad
\Norm{\epsilon\sigma_1 + \pi_1} < \Norm{\sigma_1}.
\end{equation*}
%
Define now
$\psi:= \epsilon \xi_0 + \pi_1 \sigma_2 \cdots \sigma_s$;
from equation~(\ref{two_factorizations_1}) it derives:

~~$\bullet\quad \psi = \pi_1 \left( \epsilon\pi_2\cdots\pi_r +
\sigma_2\cdots\sigma_s \right) = \pi_1\omega,
\ \,\omega \in G\, \then\, \pi_1 \divides \psi$

~~$\bullet\quad \psi = \left( \epsilon\sigma_1 + \pi_1 \right)
\left( \sigma_2\cdots\sigma_s  \right)$

From the last equality it follows in particular that
$\psi \not= 0$, since $\epsilon \sigma_1 + \pi_1 \neq 0$
as $\pi_1, \sigma_1$ not associate.

Moreover:
\begin{equation*}
\Norm{\psi} =
   \Norm{\epsilon\sigma_1 + \pi_1}
   \Norm{\sigma_2\cdots\sigma_s} 
< \Norm{\sigma_1}\Norm{\sigma_2\cdots\sigma_s}
= \Norm{\sigma_1\cdots\sigma_s} = \Norm{\xi_0}
\end{equation*}
so that:
\begin{equation*}
\Norm{\psi} < \Norm{\xi_0} = \min{\Norm{H}} \then
\psi \notin H \then \psi \in K 
\end{equation*}

Thus $\psi$ has an essentially unique factorization into
irreducibles. Since:

~~$\bullet\quad \pi_1 \divides \psi\,,\;$ and:

~~$\bullet\quad \psi = \left( \epsilon\sigma_1 + \pi_1 \right)
\left( \sigma_2\cdots\sigma_s  \right),\:$ and:

~~$\bullet\quad \pi_1$ is not associate with any $\sigma_j\,,$

it follows from corollary~(\ref{basic_property_of_K}) that\,
$\pi_1 \divides \left( \epsilon\sigma_1 + \pi_1 \right)$,\,
and then\, $\pi_1 \divides \epsilon\sigma_1$.

Since $\pi_1,\sigma_1$ are irreducible and $\epsilon$ is a
unit of $G$, this means that $\pi_1$ and $\sigma_1$ are
associate, which, according to
lemma~(\ref{lemmaFactorsNotAssociate}),
gives us the desired contradiction.
%
\end{proof}


%-----------------------------------------------------------
% SECOND SECTION:
% EXSTENSION OF METHODS GIVEN ABOVE TO OTHER NORMED DOMAINS
%-----------------------------------------------------------
\section{Exstensions to a class of normed domains}\label{ext}

In this section we will extend and generalize the methods and
concepts seen so far, and use them to show that some well-known
real and complex normed domains are UFDs.%
\footnote{Regarding the domains we're going to analyze, we must
say that it's known that all them are, in fact,\/
\emph{euclidean domains\/}, and so, a fortiori, UFDs.
But we won't make direct use of this fact; instead, we will retrieve
this well-known class of domains in an original way, and only later
we will find, state and prove the well-known result asserting that
their norm is an euclidean function. More observation about these
facts will be given later.}

\medskip
In the rest of this section, we'll limit our attention to that normed
domains $D$ which satisfy an analogue of lemma~(\ref{lemmaGaussian});
more precisely, from now on and until the end of this section, we will
denote by $D \subseteq \QQ{m}$ a normed domain which satisfy the
following:

%-------------------------
% PROPERTY (#) ENUNCIATED
%-------------------------
\begin{property}\label{property_sharp} 
$\,\forall \tau \in \QQ{m}\minz$\,
such that\, $\Norm{\tau} \leq 1$\,
and\, $\tau \notin D^\ast,\; \exists\; \theta \in D$
\,such that\, $\Norm{\theta\tau + 1} < 1$.
\end{property}

Our aim is to prove that, in such hypothesis, $D$
is an UFD, \ie in the notation of section~(\ref{notations})
of chapter~(\ref{preliminary}), that $K = D\minz$,
or equivalently that $H = \emptyset$ (note that we have no
need to prove that every $\alpha \in D$ can be written as a
finite product of irreducibles, since, being $D$ is a normed
domain, this fact already assured by
lemma~(\ref{factorizationIntoIrreducibles})).

Here is our proposition in a precise form:

%--------------------------------------------------------------------
% PROPERTY (#) IMPLIES UNIQUENESS OF FACTORIZATION INTO IRREDUCIBLES
%--------------------------------------------------------------------
\begin{thm}\label{property_sharp_implies_UFD}
Let $D \subseteq \QQ{m}$ be a normed domain such that:
$$
\forall \tau \in \QQ{m}\minz \text{ with }\, \Norm{\tau} \leq 1
\textrm{ \,and\, }
\tau \notin D^\ast,\ \, \exists\; \theta \in D:\;
 \Norm{\theta\tau + 1} < 1
$$
(\ie $D$ satisfies property~(\ref{property_sharp})).
Then $D$ is an UFD.
\end{thm}

\begin{proof}
%
The proof is very similar to that seen in the previous
section for the theorem~(\ref{gaussUFD}).

Argue by contradiction, supposing $H \neq \emptyset$.
By this assumption we may choose $\xi_0\in H$ such that
$\Norm{\xi_0} = \min{\Norm{H}}$.

Since $\xi_0 \in H$, by definition of $H$ we have in
particular:
\begin{equation}\label{two_factorizations_2}
\xi_0 = \pi_1\pi_2\cdots\pi_r =
\sigma_1\sigma_2\cdots\sigma_s
\end{equation}
with $r,s \geq 2$, for suitable $\pi_k,\sigma_h \in I$
which make the two factorizations essentially distinct.

Exactly as seen in the proof of lemma~(\ref{lemmaFactorsNotAssociate}),
it can be proved that, in the two factorizations given by
equation~(\ref{two_factorizations_2}), no $\pi_k$ is associate
with any $\sigma_h$.
Moreover, we can clearly assume without loss of generality
that $\Norm{\pi_1} \le \Norm{\sigma_1}$.

Let now \,$\tau:= \frac{\pi_1}{\sigma_1} \in
\QQ{m}$;\, clearly $\Norm{\tau} \leq 1$,
and, since $\pi_1$ and $\sigma_1$ are not associate,
$\tau \notin D^\ast$; so we have by property~(\ref{property_sharp})
that:
\begin{equation*}
\exists\ \theta \in D
~~\textrm{such that}~~
\Norm{\tau\theta + 1} < 1,
~\:\textrm{\ie such that}~~
\Norm{\sigma_1 + \theta\pi_1} < \Norm{\sigma_1}.
\end{equation*}
Define now
$\psi:= \xi_0 + \theta\pi_1\sigma_2\cdots\sigma_s$;
from~(\ref{two_factorizations_2}) it derives:

~~$\bullet$~~\:$\psi = \pi_1 \left( \pi_2\cdots\pi_r +
\theta\sigma_2\cdots\sigma_s \right) = \pi_1\omega,\;
\omega \in G\, \then\, \pi_1 \divides \psi$

~~$\bullet$~~\:$\psi = \left( \sigma_1 + \theta\pi_1 \right)
\left( \sigma_2\cdots\sigma_s  \right)$

From the last equality it follows in particular that
$\psi \neq 0$, since $\sigma_1 + \theta\pi_1 \not= 0$
(as $\pi_1 \notdivides \sigma_1$ because $\pi_1, \sigma_1$
are irreducible and not associate).

Moreover:
\begin{equation*}
\Norm{\psi} =
   \Norm{\sigma_1 + \theta\pi_1}
   \Norm{\sigma_2\cdots\sigma_s} 
< \Norm{\sigma_1}\Norm{\sigma_2\cdots\sigma_s}
= \Norm{\sigma_1\cdots\sigma_s} = \Norm{\xi_0}
\end{equation*}
so that:
\begin{equation*}
\Norm{\psi} < \Norm{\xi_0} = \min{\Norm{H}} \then
\psi \notin H \then \psi \in K 
\end{equation*}
%
Thus $\psi$ has an essentially unique factorization into
irreducibles. Since:

~~$\bullet$~~\:$\pi_1 \divides \psi\,$,\; and:

\nopagebreak
~~$\bullet$~~\:$\psi = \left( \sigma_1 + \theta\pi_1 \right)
\left( \sigma_2\cdots\sigma_s  \right)$,\; and:

\nopagebreak
~~$\bullet$~~\:$\pi_1$ is not associate with any $\sigma_j$,

it follows from corollary~(\ref{basic_property_of_K}) that\,
$\pi_1 \divides \left( \sigma_1 + \theta\pi_1 \right),$\,
and so\, $\pi_1 \divides \sigma_1$.

Since $\pi_1,\sigma_1$ are irreducible, this means that
$\pi_1$ and $\sigma_1$ are associate, which, according to
what proved previously, gives us the desired contradiction.
%
\end{proof}
%
%------------------------------------------------------
% CONCRETE EXAMPLE OF APPLICATION OF PREVIOUS THEOREM:
% h(sqrt(2)) is UFD
%------------------------------------------------------
Now we consider a concrete exemple of normed domain which
satisfies theorem~(\ref{property_sharp_implies_UFD}).

What we want to prove here is that the real normed
domain\, \mbox{$\hh{2} = \ZZ{2}$}\, is an UFD, showing
that it satisfies property~(\ref{property_sharp}).

Suppose that $\tau = a + b\sqrt{2} \in
\QQ{2} \setminus \left( \ZZ{2}^\ast \cup \{0\} \right)
\subseteq \Rzero$
(with $a, b \in \Q$ not both zero, since $\tau \not= 0$)
and such that\, $\Norm{\tau} \leq 1$\,
(\ie $\abs{a^2 - 2b^2} \leq 1$\,).

Since $a,b \in \Q$ are not both zero, it results $a^2 - 2b^2 \neq 0$.
So we can find $x,y \in \Z$ such that:
%
\begin{equation}\label{x_and_y_limits}
\left\{ \begin{array}{l}
   \abs{x + \frac{a}{a^2 - 2b^2}} \leq \frac{1}{2}
   \\
   \\
   \abs{y - \frac{b}{a^2 - 2b^2}} \leq \frac{1}{2}
\end{array}
\right.
\end{equation}

Let now\, $\theta:= x + y \sqrt{2} \in \ZZ{2}$;\,
we want to show that:\, $\Norm{\theta\tau + 1} < 1$.

This follows from simple (yet tedious) calculations:

%----------------------
% HORRIBLE CALCULATIONS
%----------------------
$
~~~\Norm{\theta\tau + 1} = \Norm{ (x + y\sqrt{2})
  (a + b\sqrt{2}) + 1 } = \\[3pt]
~~~~\abs{(ax + 2by + 1)^2 - 2(ay + bx)^2} = \\[6pt]
%
~~~~\abs{ a^2 x^2 + 4 b^2 y^2 + 1 + 4abxy + 2ax + 4by
          - 2 a^2 y^2 - 2 b^2 x^2 - 4abxy} = \\[6pt]
~~~~\abs{ x^2 (a^2 - 2b^2) - 2y^2 (a^2 - 2b^2) + 1 +
    2(ax + 2by) } = \\[6pt]
%
~~~~\abs{a^2 - 2b^2 } \cdot
    \abs{ x^2 - 2y^2 + \dfrac{1}{a^2 - 2b^2}
          + \dfrac{2(ax + 2by)}{a^2 - 2b^2}
        } 
   \leq \\[4pt]
%
~~~~\abs{ x^2 - 2y^2 + \dfrac{1}{a^2 - 2b^2}
          + \dfrac{2(ax + 2by)}{a^2 - 2b^2}
        }
    = \\[5pt]
%
~~~~\abs{ \left( x^2 + 2 \cdot \dfrac{ax}{a^2 - 2b^2} +
          \dfrac{a^2}{\left( a^2 - 2b^2 \right)^2} \right)
          - 2\left( y^2 - 2 \cdot \dfrac{by}{a^2 - 2b^2} +
          \dfrac{b^2}{\left(a^2 - 2b^2\right)^2} \right)
        } = \\[4pt]
%
~~~~\abs{ \left(x + \dfrac{a}{a^2 - 2b^2}\right)^2
          - 2 \left(y - \dfrac{a}{a^2 - 2b^2}\right)^2
        }
    \leq \\[4pt]
%
~~~~\left(x + \dfrac{a}{a^2 - 2b^2}\right)^2
          + 2 \left(y - \dfrac{a}{a^2 - 2b^2}\right)^2
    \leq\,
    \dfrac{1}{4} + 2 \cdot \dfrac{1}{4} = \dfrac{3}{4} < 1.
$

\medskip

So property~(\ref{property_sharp}) is valid for
$\hh{2}$, which then is an UFD.


%-------------------------------------------------
% THIRD SECTION: equivalent forms for property (#)
%-------------------------------------------------
\section{Equivalent forms for our results}\label{equiv}

%-------------------------------------------
% INTRODUCTION OF NEW FORMS FOR PROPERTY(#)
%-------------------------------------------
In this section we'll give two altenative equivalent forms
for property~(\ref{property_sharp}).
Then, we'll see how it is possible to express such property
in a slighty stronger way, which is however much more manageable.
In this weaker form it will be easy to recognize the
fundamental property of those normed domains which are
``euclidean with respect to the norm'' (or, more briefly,
``norm-euclidean'').
So we see as it's possibile to retrieve by an unusual way the
well-known class of euclidean real or complex domains of quadratic
numbers.

\medskip
In what follows, $m$ is of course a non-square integer
and $D$ is a normed domain with $D \subseteq \QQ{m}$.

\medskip
%--------------------------
% PROPERTY (#1) ENUNCIATED
%--------------------------
\begin{property}\label{property_sharp_1}
$\forall\, \tau \in \QQ{m}$ such that\/ $\Norm{\tau} \geq 1$,
$\, \exists\; \theta \in D$\/ such that\/
$\Norm{\tau - \theta} < \Norm{\tau}$.
\end{property}

We have immediatly:

%----------------------------------------------
% PROPERTY (#1) IS EQUIVALENT TO PROPERTY (#) 
%----------------------------------------------
\begin{thm}\label{sharp1_equivalent_sharp}
If $D \subseteq \QQ{m}$ is a normed domain, then it satisfies
property~(\ref{property_sharp}) if and only if it satisfies
property~(\ref{property_sharp_1}).
\end{thm}

\begin{proof}
%
We'll prove the two implication separately.

\begin{enumerate}

\item[\textbf{1.}]
Let us prove that property~(\ref{property_sharp})
implies property~(\ref{property_sharp_1}).

Suppose $\tau \in \QQ{m}$ such that
$\Norm{\tau} \geq 1$ . If $\tau \in D^{\ast}$, than
\textit{a fortiori} $\tau \in D$; so, letting
$\theta:= \tau \in D$, we obtain
$\Norm{\tau - \theta} = 0 < \Norm{\tau}$.
If $\tau \notin D^{\ast}$, then is also
$\tau_1:= \left(-\tau\right)^{-1} \notin D^{\ast}$,
and moreover is $\Norm{\tau_1} =
\left(\Norm{\tau}\right)^{-1} \leq 1$,
so that we obtain from property~(\ref{property_sharp})
that \,$\exists\: \theta \in D:\,
\Norm{\theta\tau_1 + 1} < 1 \then \Norm{\tau - \theta}
= \Norm{\tau}\Norm{1 -\theta\tau^{-1}} = 
\Norm{\tau}\left(\Norm{1 +\theta\tau_1}\right)
< \Norm{\tau}$.

\item[\textbf{2.}]
Let us prove that property~(\ref{property_sharp_1})
implies property~(\ref{property_sharp}).

Suppose $\tau \in \QQ{m} \setminus \left(D^\ast \cup \{0\}\right)$
such that\, $\Norm{\tau} \leq 1$. Letting
$\tau_1:= -\left(\tau^{-1}\right)$,\, it results\,
$\Norm{\tau_1} = \Norm{-\left(\tau^{-1}\right)}
= \Norm{\tau^{-1}} = \left(\Norm{\tau}\right)^{-1}\geq 1$,
\,so that we immediately obtain from
property~(\ref{property_sharp_1}) that\,
$\exists\: \theta \in D:\, \Norm{\theta - \tau_1} <
\Norm{\tau_1} \then \Norm{\tau\theta + 1} =
\Norm{\tau} \Norm{\theta +\tau^{-1}}
= \Norm{\tau} \Norm{\theta - \tau_1} <
\Norm{\tau}\Norm{\tau_1} = \Norm{\tau \tau_1}
=\Norm{-1} = 1$.

\end{enumerate}
The two implications prove the theorem.
%
\end{proof}

Let's now see another property which is (almost)
equivalent to property~(\ref{property_sharp}):

%--------------------------
% PROPERTY (#2) ENUNCIATED
%--------------------------
\begin{property}\label{property_sharp_2}
$\,\forall\, \alpha, \beta \in D\minz$\,
such that\, $\Norm{\alpha} \geq \Norm{\beta}$,\:
$\exists\; \theta,\rho \in D$\, such that\,
$\alpha = \beta\theta + \rho$\, and\,
$\Norm{\rho} < \Norm{\alpha}$.
\end{property}


The following result holds:
%------------------------------------------------------
% PROPERTY (#2) IS (ALMOST) EQUIVALENT TO PROPERTY (#) 
%------------------------------------------------------
\begin{thm}\label{sharp2_almost_equivalent_sharp}
If $D \subseteq \QQ{m}$ is a normed
domain such that $D \not\subseteq \Q$, then it satisfies
property~(\ref{property_sharp}) if and only if it
satisfies property~(\ref{property_sharp_2}).
\end{thm}

\begin{proof}
%
First, since $D \not \subseteq \Q$, there exist\,
$r,s \in \Q:\: r + s\sqrt{m} \in D,\: s \ne 0$ .
From this, it's quite easy to deduce that the field of
quotients of $D$ is $\QQ{m}$, so that:
$$
\forall \xi \in \QQ{m}:\: \exists\: \lambda_1,
\lambda_2 \in D\ \,\textrm{such\ that}\, \ \lambda_2
\ne 0\ \textrm{and}\ \xi = \frac{\lambda_1}{\lambda_2}
$$
%
At this point, note that thanks to the
theorem~(\ref{sharp1_equivalent_sharp})
we need only to prove that:
\begin{equation*}
\textrm{(property~(\ref{property_sharp_1}) holds for $D$)}
\iff
\textrm{(property~(\ref{property_sharp_2}) holds for $D$)}.
\end{equation*}

We'll do this in two steps.

\begin{enumerate}

\item[\textbf{1.}]
Let us prove that property~(\ref{property_sharp_1})
implies property~(\ref{property_sharp_2}).

Let $\alpha,\beta \in D\minz$ with\, $\Norm\alpha \geq \Norm\beta$,
and put\, $\tau:= \frac{\alpha}{\beta} \in \QQ{m}$, so that
\,$\Norm\tau \geq 1$.
Then for property~(\ref{property_sharp_1}) we have that
$\exists\; \theta \in D$\, with\,
$\Norm{\tau - \theta} < \Norm{\tau}$, \ie
$\Norm{\alpha - \theta\beta} = \Norm{\beta}\Norm{\tau - \theta}
< \Norm{\beta}\Norm{\tau} = \Norm{\beta\tau} = \Norm{\alpha}$.
So it's enough to put $\rho:= (\alpha - \beta\theta) \in D$
to get the desired result.

\item[\textbf{2.}]
Let us prove that property~(\ref{property_sharp_2})
implies property~(\ref{property_sharp_1}).

Let $\tau \in \QQ{m}$ such that\, $\Norm{\tau} \geq 1$;\,
we know that $\tau$ can be written as
$\tau = \frac{\alpha}{\beta}$ for suitable $\alpha,\beta
\in D$ with $\beta \neq 0$. So by our hypotesis it results:
$$ \Norm{\tau} \geq 1 \then \Norm{\alpha} \geq \Norm{\beta} $$
and thus, from property~(\ref{property_sharp_2}):
$$
\exists\: \theta,\rho \in D \textrm{ such that }\,
\Norm{\rho} < \Norm{\alpha} \textrm{ and }
\alpha = \beta\theta + \rho
$$
so that:
\begin{eqnarray*}
\Norm{\tau - \theta} & = &
\Norm{\beta^{-1}} \Norm{\beta\tau - \beta\theta} =
\left(\Norm{\beta}\right)^{-1}
\Norm{\alpha - \beta\theta} = \\
& = & \frac{\Norm{\rho}}{\Norm{\beta}} <
\frac{\Norm{\alpha}}{\Norm{\beta}} =
\Norm{\frac{\alpha}{\beta}} = \Norm{\tau}.
\end{eqnarray*}
Thus\, $\Norm{\tau - \theta} < \Norm{\tau}$,\,
as required.
\end{enumerate}
The two implications prove the theorem.
%
\end{proof}


\smallskip
Let's finally see a stonger but handier form of
property~(\ref{property_sharp})\footnote{%
but we'll see in chapter~\ref{limits} that
property~(\ref{property_sharp_weaker}) is only
apparently stronger than property~(\ref{property_sharp}),
the two being essentially equivalent.}:

%----------------------------------
% PROPERTY (#-weaker) ENUNCIATED
%----------------------------------
\begin{property}\label{property_sharp_weaker}
$\,\forall\, \psi \in \QQ{m},\:
\exists\: \theta \in D$ such that\,
$\Norm{\psi - \theta} < 1$.
\end{property}

The following useful result holds:
%--------------------------------------------
% PROPERTY (#-weaker) IMPLIES PROPERTY (#)
%--------------------------------------------
\begin{thm}\label{property_sharp_weaker_implies_property_sharp}
If $D \subseteq \QQ{m}$ is a normed domain which satisfies
property~(\ref{property_sharp_weaker}), then it also satisfies
property~(\ref{property_sharp}).
In particular, every normed domain $D$ which satisfies
property~(\ref{property_sharp_weaker}) is an UFD.
\end{thm}

\begin{proof}
%
Since property~(\ref{property_sharp}) is equivalent to
property~(\ref{property_sharp_1}), it's enough to prove that
property~(\ref{property_sharp_1}) follows from
property~(\ref{property_sharp_weaker}).

Let $\tau \in \QQ{m}$\, such that\,
$\Norm{\tau} \geq 1$; then, from property~(\ref{property_sharp_weaker})
we can say that $\exists\: \theta \in D$\, such that\,
$\Norm{\tau - \theta} < 1 \then \Norm{\tau - \theta} <
\Norm{\tau}$ since $1 \leq \Norm{\tau}$.
So $D$ satisfies property~(\ref{property_sharp_1}).
%
\end{proof}

\medskip

%----------------------------------------------------------
% EQUIVALENCE BETWEEN PROPERTY(#-weaker) AND NORM-EUCLIDEAN
%----------------------------------------------------------
Finally we see that property~(\ref{property_sharp_weaker})
is equivalent to the fact that the domain $D$ is
\emph{norm-euclidean} (at least in the case
$D \not\subseteq \Q$).

Obiouvsly, we say that a normed domain $D$ is norm-euclidean
if its norm\, $\Normop: D \longrightarrow \N$ is an
euclidean function, \ie if:
$$
\forall \alpha,\beta \in D,\,\beta \ne 0:\, \exists\:
\theta,\rho \in D\,\ \textrm{such that}\ \, 
\alpha = \beta \theta + \rho\,\ \textrm{and}\,\
(\rho = 0\ \textrm{or}\ \Norm{\rho} < \Norm{\beta})
$$

\smallskip
The following result holds:

\begin{thm}\label{sharp-weaker_equivalent_norm-euclidean}
If $D \subseteq \QQ{m}$ is a normed
domain such that $D \not\subseteq \Q$, then it satisfies
property~(\ref{property_sharp_weaker}) if and only if
it is norm-euclidean.
\end{thm}

\begin{proof}
%
Again, from $D \not\subseteq \Q$ we can immediatly conclude
that the field of quotients of $D$ is $\QQ{m}$, so that:
$$
\forall \xi \in \QQ{m}:\: \exists\: \lambda_1,
\lambda_2 \in D\ \,\textrm{such\ that}\, \ \lambda_2
\ne 0\ \textrm{and}\ \xi = \frac{\lambda_1}{\lambda_2}
$$
Suppose first that $D$ is norm-euclidean.
Then, given\, $\psi \in \QQ{m}$\, and writing
\,$\psi = \frac{\alpha}{\beta}$ for suitable $\alpha,\beta \in D$
with $\beta \ne 0$,\, we have that there exist $\theta,\rho
\in D$ such that $\alpha = \beta \theta + \rho$, with
$\Norm{\rho} < \Norm{\beta}$\footnote{%
in principle, there is also the separate case $\rho = 0$, which
however still implies $\Norm{\rho} = 0 < \Norm{\beta}$, since
$\beta \ne 0$.}.
We have then:
$$
\Norm{\psi - \theta} =
\Norm{\frac{\alpha - \theta\beta}{\beta}}
= \Norm{\frac{\rho}{\beta}}
= \frac{\Norm{\rho}}{\Norm{\beta}}
< \frac{\Norm{\beta}}{\Norm{\beta}} = 1
$$
so that $D$ satisfies property~(\ref{property_sharp_weaker}).

Conversly, suppose now that $D$ satisfies
property~(\ref{property_sharp_weaker}).
Then, given\, $\alpha,\beta \in D$ with $\beta \ne 0$,\,
letting\, $\psi:= \frac{\alpha}{\beta} \in \QQ{m}$,
\,we can find $\theta \in D$ such that\,
$\Norm{\psi - \theta} < 1$,\, \ie such that
$\Norm{\alpha - \beta\theta} < \Norm{\beta}$.
Thus, defining $\rho:= \alpha - \beta\theta$, we
obtain:
$$
\alpha = \beta\theta + \rho, \quad
\Norm{\rho} < \Norm{\beta}
$$
and so \emph{a fortiori}:
$$
\alpha = \beta\theta + \rho, \quad
\Norm{\rho} < \Norm{\beta}\ \,\textrm{or}\,\ \rho = 0.
$$
so that $D$ satisfies property~(\ref{property_sharp_weaker}).
%
\end{proof}

%----------------------------------------------------
% FOURTH SECTION: CONCRETE APPLICATION OF OUR METHODS
%----------------------------------------------------
\section{Some concrete applications}\label{apps}

At this point we can retrieve the classical:

%------------------------------------------
% Proof that h(sqrt(m)) is an U.F.D for
% m = -1, +2, -2, +3, -3, +5, -7, -11, +13
%------------------------------------------
\begin{thm}\label{normed_UFDs_1}
The complex normed domains:
$$
\hhm{1} = \Z\left[\,i\,\right],\ \hhm{2} = \ZZM{2},
\ \hhm{3},\ \hhm{7},\ \hhm{11}
$$
and the real normed domains:
$$
\hh{2} = \ZZ{2},\ \hh{3} = \ZZ{3},\ \hh{5},\ \hh{13}
$$
are UFDs.
\end{thm}

\begin{proof}
%
We want to show that the domains listed in the statement
of the thorem all satisfy property~(\ref{property_sharp_weaker}),
so from theorem~(\ref{property_sharp_weaker_implies_property_sharp})
it will follow immediatly that all these domains are UFDs.

We distinguish two cases.

\smallskip

% FIRST CASE: m = 2 or 3 (mod 4)
%-------------------------------
\textbf{Case 1:} $ D = \ZZ{m}$, for $m = -1,\: -2,\: +2,\: +3$.

This case comprises all the $m$ such that $m \congruent 2$ or $3 \pmod 4$.

Given\, $\psi \in \QQ{m}$, it results
\,$\psi = a + b\sqrt{m}$\, for appropriate\, $a,b\in\Q$.
\,If we take $x,y \in \Z$ such that\,
$\abs{a - x} \leq \frac{1}{2}$\, and\,
$\abs{b - y} \leq \frac{1}{2}$,\, and define\,
$\theta := x + y\sqrt{m}$,\, then we have clearly\,
$\theta \in D$.

Moreover:
$$
\Norm{\psi - \theta} = \Norm{(a - x) + (b - y)\sqrt{m}}
= \abs{(a - x)^2 - m(b - y)^2}.
$$

If $m = -1,\; -2,\; +2$ we have then:
$$
\Norm{\psi - \theta} \leq (a - x)^2 + \abs{m}(b - y)^2
\leq \frac{1}{4} + \abs{m} \cdot \frac{1}{4}
= \frac{1 + \abs{m}}{4} \leq \frac{1 + 2}{4} =
\frac{3}{4} < 1
$$
\ie \,$\Norm{\psi - \theta} < 1.$

If $m = 3$, we must reason in a slighty different way.
There are the two following possibilities:

$$
\abs{(a - x)^2 - m (b - y)^2} = (a - x)^2 - m (b - y)^2
\leq (a - x)^2 \leq \frac{1}{4} < 1
$$

or:
$$
\abs{(a - x)^2 - m (b - y)^2} = m (b - y)^2 - (a - x)^2
\leq m(b - y)^2 \leq \frac{m}{4} = \frac{3}{4} < 1
$$

In both these cases we have:
$$
\Norm{\psi - \theta} = \abs{(a - x)^2 - m (b-y)^2} < 1
$$
as required.

\smallskip

% SECOND CASE: m = 1 (mod 4)
%---------------------------
\textbf{Case 2:} $ D = \hh{m}$,\,
                   for $m = -3,\, -7,\, -11,\, +5,\, +13$.

This case comprises all the $m$ such that $m \congruent 1 \pmod 4$.

Given\, $\psi \in \QQ{m}$, we can obviously find\, $a,\,b\in\Q$
such that $\psi = \frac{a + b\sqrt{m}}{2}$.
If we take $y \in \Z$ such that\,
$\abs{b - y} \leq \frac{1}{2}$,\, and then $x \in \Z$
such that\, $\abs{\frac{a - y}{2} - x} \leq \frac{1}{2}$\,
and define\, \mbox{$\theta:= \frac{(2x+y) + y\sqrt{m}}{2}$,}
\,then we have\, $\theta \in D$
(since $ 2x + y \congruent y $ (mod 2)).

Moreover:
$$
\Norm{\psi - \theta} =
\Norm{\frac{(a -y - 2x) + (b - y)\sqrt{m}}{2}} =
\abs{\frac{1}{4}(a -y - 2x)^2 - \frac{1}{4} m (b - y)^2}
$$

\smallskip
If\, $m = -3,\; -7,\; -11,\; +5$\, we have then:
\smallskip
%
\begin{eqnarray*}
\Norm{\psi - \theta} & \leq &
\left( \frac{a -y - 2x} {2} \right)^2 +
  \frac{1}{4} \abs{m} (b - y)^2
\leq \left(\frac{1}{2}\right)^2
+ \frac{\abs{m}}{4}\left(\frac{1}{2}\right)^2 = \\
& & \\
& = & \frac{1}{4} + \abs{m}\cdot\frac{1}{16} =
\frac{4+\abs{m}}{16}
\leq \frac{4 + 11}{16} = \frac{15}{16} < 1
\end{eqnarray*}

\smallskip
\ie \,$\Norm{\psi - \theta} < 1.$
%

\smallskip
If $m = 13$, we must reason in a slighty different way.

In this case, we can have either:
%
\begin{eqnarray*}
\abs{ \frac{1}{4}(a -y - 2x)^2 - \frac{1}{4} m (b - y)^2 }
& = & \frac{1}{4}(a - y - 2x)^2 - \frac{1}{4} m (b - y)^2 \leq \\[7pt]
& \leq & \frac{1}{4}(a - y - 2x)^2 = \left(\frac{a -y - 2x}{2}\right)^2
\leq \frac{1}{4} < 1
\end{eqnarray*}

or: \smallskip
%
\begin{eqnarray*}
\abs{ \frac{1}{4}(a -y - 2x)^2 - \frac{1}{4} m (b - y)^2 }
& = & \frac{1}{4} m (b - y)^2 - \frac{1}{4}(a -y - 2x)^2 \leq \\[6pt]
& \leq & \frac{1}{4} m (b - y)^2 \leq
\frac{m}{4} \cdot \left(\frac{1}{2}\right)^2 =
\frac{m}{16} = \frac{13}{16} < 1
\end{eqnarray*}

In both cases it results:
$$
\Norm{\psi - \theta} =
\abs{ \frac{1}{4}(a - y - 2x)^2 - \frac{1}{4} m(b - y)^2 }
< 1
$$

as required.
%
\end{proof}

%-----------------
% SOME REFERENCES
%-----------------
More information about normed domains $D$ of the form
$\hh{m}$ which satisfy (or do not satisfy)
property~(\ref{property_sharp_weaker})\, (which is
effectively equivalent to say that the domain $D$ is
euclidean with respect to the norm) can be found in
Chapter XIV of~\cite{H&W}. In particular, see
\mbox{theorems\, (246)\,--\,(247)\,--\,(248)\,--\,(249).}

%
\bigskip
Now we see two results that, taken together, give a
slighty stronger version of theorem (248) of~\cite{H&W}.
They also improve theorem~(\ref{normed_UFDs_1}), at
least for positive values of $m$.

%---------------------------------------
% Proof that h(sqrt(m)) is an U.F.D for
% m = 2, 3, 6, 7, 11
%---------------------------------------
\medskip
The first result is:

\begin{thm}\label{normed_UFDs_2}
The domain $\hh{m}$ is norm-euclidean for the following values of $m$:
$$ m\, =\, 2,\, 3,\, 6,\, 7,\, 11. $$
\end{thm}

\begin{proof}
%
Note that in what follows is $\hh{m} = \ZZ{m}$ since every $m$ here
is $\not\congruent 1\ (\mathrm{mod}\ 4)$.

Argue by contradiction, assuming that $\hh{m}$ is not norm-euclidean.
Then it's easy to see that there must exist\, $r, s \in \Q$\, such that:
\begin{equation}\label{r_and_s_basic_eq}
\forall\: x,y \in \Z: \quad
\abs{ (r - x)^2 - m(s - y)^2 } \geq 1
\end{equation}

First, note that the expression\,
$E(x, y, r, s):= \abs{ (r - x)^2 - m (s - y)^2 }$\,
is unaltered by the substitution:
%
\begin{equation}
\varrho\,:\:\left\{ \begin{array}{l}
r \mapsto \epsilon_1 r + u \\[4pt]
x \mapsto \epsilon_1 x + u \\[4pt]
s \mapsto \epsilon_2 s + v \\[4pt]
y \mapsto \epsilon_2 y + v \\[4pt]
\end{array}
\right.
\end{equation}
%
where\, $\epsilon_1, \epsilon_2 \in \{+1, -1\}$\, and
$u,v$ are arbitrary integers,
Moreover (and this is a fundamental fact), if $x,y$ run
over all the rational integers, $\varrho(x)$ and $\varrho(y)$
do the same.

At this point it's clear that we can suppose without loss of
generality that \,$0 \leq r \leq \frac{1}{2}$ and
$0 \leq s \leq \frac{1}{2}$\, (since for every\, $q \in \Q$\,
there exist\, $\epsilon \in \{+1, -1\}$ and $z \in \Z$\, such
that\, $0 \leq z + \epsilon q \leq \frac{1}{2}$\,).

Now, from~(\ref{r_and_s_basic_eq}) it derives that
for every $x, y \in \Z$ one of the following
statements must hold:
%
\begin{eqnarray}
\left[P(x,y)\right]\::\quad (r - x)^2 \geq 1 + m(s - y)^2
\\[4pt]
\left[N(x,y)\right]\::\quad m(s - y)^2 \geq 1 + (r - x)^2
\end{eqnarray}
%
Some particular inequalities which we shall use are:
\begin{equation*}
\begin{array}{llll}
\left[P(0,0)\right]\:: & r^2 \geq 1 + ms^2 &
\quad\quad
\left[N(0,0)\right]\:: & ms^2 \geq 1 + r^2
\\[7pt]
\left[P(1,0)\right]\:: & (1 - r)^2 \geq 1 + ms^2 &
\quad\quad
\left[N(1,0)\right]\:: & ms^2 \geq 1 + (1 - r)^2
\\[7pt]
\left[P(-1,0)\right]\:: & (1 + r)^2 \geq 1 + ms^2 &
\quad\quad
\left[N(-1,0)\right]\:: & ms^2 \geq 1 + (1 + r)^2
\end{array}
\end{equation*}
%
We know that one inequality at least in each of these
pair of inequalities is true; we also know that\,
$0 \leq r \leq \frac{1}{2}$\, and\,
$0 \leq s \leq \frac{1}{2}$.

First, if\, $r = s = 0$,\, then $P(0,0)$ and $N(0,0)$
are both false, so this possibility is excluded.
Moreover we have:
$$
1 \leq 1 + ms^2 \quad\: \textrm{and}
\quad\: r^2 \leq \frac{1}{4}
$$
so that $P(0,0)$ is false.
But also $P(0,1)$ is false, because from it derives:
$$
1 \geq (1 - r)^2 \geq 1 + ms^2 \geq 1 \,\then\,
\left\{
\begin{array}{lll}
1 + ms^2 & = & 1 \\
(1 - r)^2 & = & 1
\end{array}
\,\then\,
r = s = 0
\right.
$$
a possibility already excluded.

Thus it's assured that $P(0,0)$ and $P(1,0)$ are both
false, so that $N(0,0)$ and $N(1,0)$ are true.
If $P(-1, 0)$ were true, then $N(1, 0)$ and
$P(-1, 0)$ combined would give:
$$
(1 + r)^2 \geq 1 + ms^2 \geq 2 + (1 - r)^2 \then
4r \geq 2 \then r \geq \frac{1}{2}
$$
so that $r = \frac{1}{2}$, and consequentely:
$$
\frac{9}{4} = (1 + r)^2 \geq 1 + ms^2 \geq
2 + (1 - r)^2 = \frac{9}{4} \:\then\:
1 + ms^2 = \frac{9}{4} \:\then\: ms^2 = \frac{5}{4}
$$
But this is impossible, since writing
$s = \frac{p}{q},\,$ for $p, q \in \N$ coprime,
it would give:
$$
4mp^2 = 5q^2 \then p^2 \divides 5 \then p = 1
\then 5q^2 = 4m \then 5 \divides m
$$
with\, $m = 2,\, 3,\, 6,\, 7\ \textrm{or}\ 11$,\,
a contradiction.

We have thus proved that $P(-1, 0)$ is false, so that
$N(-1, 0)$ holds. This gives:
%
\begin{equation}\label{m*s^2 basic bound}
ms^2 \geq 1 + (1 + r)^2 \geq 2
\end{equation}
and so, since $s^{-2} \geq 4$:
$$ m \geq 2s^{-2} \geq 8 $$
so that the only case we have to deal with from now on
is\, $m = 11$.

Let us consider the inequalities:
$$
\left[P(2,0)\right]\::\; (2 - r)^2 \geq 1 + 11s^2
\quad\quad
\left[N(2,0)\right]\::\; 11s^2 \geq 1 + (2 - r)^2
$$
We know that at least one of them must hold.
But $N(2,0)$ is false, since it implies:
$$
11s^2 \geq 1 + (2 - r)^2 \geq 1 + \left(2 - \frac{1}{2}\right)^2
= 1 + \left(\frac{3}{2}\right)^2 = \frac{13}{4}
\;\then\; 11 \geq s^{-2}\cdot \frac{13}{4} \geq 13,
$$
a contradiction. So $P(2,0)$ is true, \ie:\:
$(2 - r)^2 \geq 1 +ms^2$.\\
From this and from (\ref{m*s^2 basic bound}) we deduce\,
$(2 - r)^2 \geq 1 + 11s^2 \geq 2 + (1 + r)^2$,\, and so:
%%%
$$
2 \leq (2 - r)^2 - (1 + r)^2 = 3 - 6r \then - 6r \geq -1
$$
\ie:
\begin{equation}\label{r_upper_bound_1}
r \leq \frac{1}{6}
\end{equation}
%
Let us now consider the inequalities:
$$
\left[P(-2,1)\right]\::\; (2 + r)^2 \geq 1 + 11(1 - s)^2
\quad\quad
\left[N(-2,1)\right]\::\; 11(1 - s)^2 \geq 1 + (2 + r)^2
$$
We know that at least one of them must hold.
If this were $N(-2, 1)$, we would have:
$$
11(1 - s)^2 \geq 1 + (2 + r)^2 \geq 5 \then
1 - s \geq \sqrt{\frac{5}{11}} \then
s \leq 1 - \sqrt{\frac{5}{11}} < \frac{1}{3}.
$$
\medskip
But then, from (\ref{m*s^2 basic bound}) would derive:
$$
1 + \frac{11}{9}= 1 + 11 \cdot \left(\frac{1}{3}\right)^2 >
1 + 11s^2 \geq 2 + (1 + r)^2 \geq 3
\then \frac{20}{9} \geq 3,
$$
a contradiction.

Thus $N(-2, 1)$ is false, so that $P(-2, 1)$ must hold, \ie:
\begin{equation}\label{N(-1,2)}
(2 + r)^2 \geq 1 + 11(1 - s)^2
\end{equation}
%
Now we must consider the two following further inequalities:
$$
\left[P(2,1)\right]\::\; (2 - r)^2 \geq 1 + 11(1 - s)^2
\quad\quad
\left[N(2,1)\right]\::\; 11(1 - s)^2 \geq 1 + (2 - r)^2
$$
Again, we know that at least one of them must hold.
If $N(2,1)$ were hold, we'd get from (\ref{N(-1,2)}):
$$
(2 + r)^2 \geq 1 + 11(1 - s)^2 \geq 2 + (2 - r)^2
\then 4 + 4r + r^2 \geq 2 + 4 - 4r + r^2 \then 8r \geq 2
$$
and so\, $r \geq \frac{1}{4}$,\, which contradicts
(\ref{r_upper_bound_1}).

Thus $P(2,1)$ holds, and so, being\, $1 - s \geq \frac{1}{2}$\,,\,
we obtain:
$$
(2 - r)^2 \geq 1 +11(1 - s)^2 \geq 1 + \frac{11}{4}
= \frac{15}{4} \then 2 - r \geq \frac{\sqrt{15}}{2} \then
r \leq \frac{4 - \sqrt{15}}{2} < \frac{1}{15}.
$$
\medskip
So we have a further upper bound for\, $r$\,:
\begin{equation}\label{r_upper_bound_2}
r < \frac{1}{15}
\end{equation}
%
From (\ref{N(-1,2)}) we deduce now:
$$
1 + 11(1 - s)^2 \leq (2 + r)^2 < (2 + \frac{1}{15})
= \left(\frac{31}{15}\right)^2 \then (1 - s)^2 <
\frac{1}{11}\left(\left(\frac{31}{15}\right)^2-1\right)
$$
so that, since\: $(0.55)^2 \geq
\left(\dfrac{\left(\frac{31}{15}\right)^2 - 1}{11}\right)$%
,\, we have the following lower bound for $s$\,:
\begin{equation}\label{s_lower_bound_1}
1 - s < 0.55 \;\then\; s > 0.45
\end{equation}
Finally, let us consider the following pairs of inequalities:
$$
\left[P(5,-1)\right]\::\; (5 - r)^2 \geq 1 + 11(1 + s)^2
\quad\quad
\left[N(5,-1)\right]\::\; 11(1 + s)^2 \geq 1 + (5 - r)^2
$$
If $P(5, -1)$ were true, we would get from
(\ref{s_lower_bound_1}):\\
$$
20.25 = \frac{81}{4} = \left(5 - \frac{1}{2}\right)^2 \geq
(5 - r)^2 \geq 1 + 11(1 + s)^2 > 1 + 11\cdot(1.45)^2 =
24.1275,
$$
a contradiction.
So $N(5, -1)$ must hold.
Using (\ref{r_upper_bound_2}), we get then:
$$
\frac{99}{4} = 11\left(1 + \frac{1}{2}\right)^2 \geq
11(1 + s)^2 \geq1 + (5 - r)^2 >
1 + \left(5 - \frac{1}{15}\right)^2 = \frac{5701}{225}
$$
so that\, $22275 = 99 \cdot 225 > 4 \cdot 5701 = 22804$,\,
a contradiction.

\medskip
Thus we can eventually conclude that also $\hh{11} = \ZZ{11}$
is norm-euclidean, and the theorem is proved.
%
\end{proof}

%---------------------------------------
% Proof that h(sqrt(m)) is an U.F.D for
% m = 5, 13, 17, 21, 29, 33, 37, 41
%---------------------------------------
Now we can see an analogue of the previous theorem for
the case $m \congruent 1$ (mod $4$). The argumentations
will be very similar.

\begin{thm}\label{normed_UFDs_3}
The domain $\hh{m}$ is norm-euclidean for the following
values of $m$:
$$ m\, =\, 5,\, 13,\, 17,\, 21,\, 29,\, 33,\, 37,\, 41. $$
\end{thm}

\begin{proof}
%
First note that here every $m$ here is $\congruent 1 \pmod 4$, so that:
$$
\hh{m} = \left\{\dfrac{(2x + y) + y\sqrt{m}}{2}:\: x, y \in \Z\right\}.
$$
We'll argue by contradiction, assuming that $\hh{m}$ is not
norm-euclidean.

Then it's easy to see that, written\, $n := \frac{1}{4}m$,
\,must exist\, $r, s \in \Q$\, such that:
\begin{equation}\label{new_r_and_s_basic_eq}
\forall\: x,y \in \Z: \quad
\abs{ \left(r - x - \frac{y}{2}\right)^2 - n(s - y)^2 }
\geq 1
\end{equation}
(to prove this, write the generic element $\tau \in \QQ{m}$
as $\tau = r + \frac{1}{2}s\sqrt{m}$,\, with $r, s \in \Q$)

First, note that the expression\, $E(x, y, r, s):=
\abs{ \left(r - x - \frac{y}{2}\right)^2 - n (s - y)^2 }$\,
is unaltered by any of the following four substitutions:
%
\begin{equation}
\varrho_1:
\left\{ \begin{array}{l}
r \mapsto \epsilon r + u  \\[4pt]
x \mapsto \epsilon x + u  \\[4pt]
s \mapsto s                 \\[4pt]
y \mapsto y                 \\[4pt]
\end{array}
\right.
\end{equation}
%
\begin{equation}
\varrho_2:
\left\{ \begin{array}{l}
r \mapsto r      \\[4pt]
x \mapsto x - v  \\[4pt]
s \mapsto s + 2v \\[4pt]
y \mapsto y + 2v \\[4pt]
\end{array}
\right.
\end{equation}
\begin{equation}
\varrho_3:
\left\{ \begin{array}{l}
r \mapsto r     \\[4pt]
x \mapsto x + y \\[4pt]
s \mapsto -s    \\[4pt]
y \mapsto -y    \\[4pt]
\end{array}
\right.
\end{equation}
\begin{equation}
\varrho_4:
\left\{ \begin{array}{l}
r \mapsto \frac{1}{2} - r \\[4pt]
x \mapsto -x              \\[4pt]
s \mapsto 1 - s           \\[4pt]
y \mapsto 1 - y           \\[4pt]
\end{array}
\right.
\end{equation}
%
where\, $\epsilon \in \{+1, -1\}$\, and $u,v$ are
arbitrary integers.
Moreover (and this is a fundamental fact), for any of the transormations
$\sigma_i$ given above, if $x,y$ run over all the rational integers,
${\varrho_i}(x)$ and ${\varrho_i}(y)$ do the same.

Now, using this observation, we're going now to prove that it
can be assumed without loss of generality that\,
$0 \leq r \leq \frac{1}{2}$\, and\,
$0 \leq s \leq \frac{1}{2}$.

We firt use $\varrho_1$ to make
$0 \leq r \leq \frac{1}{2}$, then $\varrho_2$ to make
$-1 \leq s \leq 1$; and then, if necessary, $\varrho_3$
to make $0 \leq s \leq 1$. If $0 \leq s \leq \frac{1}{2}$,
the reduction is completed. On the other hand, if
$\frac{1}{2}\leq s\leq 1$, we end by using $\varrho_4$,
as we can do because $\frac{1}{2} - r$ lies between $0$
and $\frac{1}{2}$ if $r$ does so.

Now, from (\ref{new_r_and_s_basic_eq}) it derives that
for every $x, y \in \Z$ one of the following
statements must hold:
%
\begin{eqnarray}
\left[P(x,y)\right]\::\quad \left(r-x-\frac{y}{2}\right)^2
\geq 1 + n(s - y)^2 \\[4pt]
\left[N(x,y)\right]\::\quad n(s - y)^2 \geq
1 + \left(r - x - \frac{y}{2}\right)^2
\end{eqnarray}
%
Some particular inequalities which we shall use are:
\begin{equation*}
\begin{array}{llll}
\left[P(0,0)\right]\:: & r^2 \geq 1 + ns^2 &
\quad\quad
\left[N(0,0)\right]\:: & ns^2 \geq 1 + r^2
\\[7pt]
\left[P(1,0)\right]\:: & (1 - r)^2 \geq 1 + ns^2 &
\quad\quad
\left[N(1,0)\right]\:: & ns^2 \geq 1 + (1 - r)^2
\\[7pt]
\left[P(-1,0)\right]\:: & (1 + r)^2 \geq 1 + ns^2 &
\quad\quad
\left[N(-1,0)\right]\:: & ns^2 \geq 1 + (1 + r)^2
\end{array}
\end{equation*}
%
We know that one inequality at least in each of these
pair of inequalities is true; we also know that\,
$0 \leq r \leq \frac{1}{2}$\, and\,
$0 \leq s \leq \frac{1}{2}$.

First, if\, $r = s = 0$,\, then $P(0,0)$ and $N(0,0)$
are both false, so this possibility is excluded.
Moreover we have:
$$
1 \leq 1 + ns^2 \quad\: \textrm{and}
\quad\: r^2 \leq \frac{1}{4}
$$
so that $P(0,0)$ is false.
But also $P(1,0)$ is false, because from it derives:
$$
1 \geq (1 - r)^2 \geq 1 + ns^2 \geq 1 \,\then\,
\left\{
\begin{array}{lll}
1 + ns^2 & = & 1 \\
(1 - r)^2 & = & 1
\end{array}
\,\then\,
r = s = 0
\right.
$$
a possibility already excluded.

Thus it's assured that $P(0,0)$ and $P(1,0)$ are both
false, so that $N(0,0)$ and $N(1,0)$ are true.
If $P(-1, 0)$ were true, then $N(1, 0)$ and
$P(-1, 0)$ combined would give:
$$
(1 + r)^2 \geq 1 + ns^2 \geq 2 + (1 - r)^2 \then
4r \geq 2 \then r \geq \frac{1}{2}
$$
so that $r = \frac{1}{2}$, and consequentely:
$$
\frac{9}{4} = (1 + r)^2 \geq 1 + ns^2 \geq
2 + (1 - r)^2 = \frac{9}{4} \:\then\:
1 + ns^2 = \frac{9}{4} \:\then\: ns^2 = \frac{5}{4}
$$
But this is impossible, since writing
$s = \frac{p}{q},\,$ for $p, q \in \N$ coprime,
it would give:
$$
mp^2 = 4np^2 = 5q^2 \then p^2 \divides 5 \then p=1
\then 5q^2 = m \then 5 \divides m
$$
with\, $m = 5,\, 13,\, 17,\, 21,\, 29,\, 33,\, 37
\ \textrm{or} \ 41$,\, so that:
$$
m = 5 \then q^2 = 1 \then p = q = 1 \then
s = \frac{p}{q} = 1,
$$
a contradiction.

We have thus proved that $P(-1, 0)$ is false, so that
$N(-1, 0)$ holds. This gives:
%
\begin{equation}\label{shared}
ns^2 \geq 1 + (1 + r)^2 \geq 2
\end{equation}
and so, since $s^{-2} \geq 4$:
$$
\frac{1}{4}m = n \geq 2s^{-2} \geq 8 \then
m \geq 4 \cdot 8 = 32,
$$
%
Thus, from now on, we have only to deal with the cases
$m = 33$ or $m = 37$ or $m = 41$, so that it is surely:
\begin{equation}\label{m_and_n_bounds}
33 \leq m \leq 41 \quad~ \ie \quad~\!
\frac{33}{4} \leq n \leq \frac{41}{4}
\end{equation}
%
From the inequality~(\ref{shared}) we immediately get:
$$
\frac{41}{4}s^2 \geq ns^2 \geq 1 + (1 + r)^2 \geq 2 \then
s^2 \geq \frac{8}{41} \then s \geq \sqrt{\frac{8}{41}} >
0.4417
$$
so that:
\begin{equation}\label{s_lower_bound}
s > 0.4417
\end{equation}
and:
$$
(1 + r)^2 \leq ns^2 - 1 \leq \frac{41}{4} \cdot
\left(\frac{1}{2}\right)^2 - 1 = \frac{25}{16} \then
1 + r \leq \frac{5}{4}
$$
so that:
\begin{equation}\label{r_upper_bound}
r \leq \frac{1}{4} = 0.25
\end{equation}
%
Consider now the inequality:
$$
\left[P(-1, 1)\right]\::\;
\left(1 - \frac{1}{2} - r\right)^2 \geq 1 + n(1 - s)^2
$$
If it were true, we would get:
$$
\left(\frac{1}{2} + r\right)^2 \geq 1 + n(1 - s)^2 \geq
1 + n\left(\frac{1}{2}\right)^2 = 1 + \frac{1}{4}n
\geq 1 + \frac{1}{4} \cdot \frac{33}{4} = \frac{49}{16}
= \left(\frac{7}{4}\right)^2
$$
so that:
$$
1 = \frac{1}{2} + \frac{1}{2}  \geq \frac{1}{2} + r \geq \frac{7}{4}
$$
a contradiction.

Then $N(-1, 1)$ must hold, \ie:
\begin{equation}\label{N(-1,1)}
n(1 - s)^2 \geq 1 + \left(\frac{1}{2} + r\right)^2
\end{equation}

Consider then the following inequality:
$$
\left[P(1, 1)\right]\::\;
\left(1 + \frac{1}{2} - r\right)^2 \geq 1 + n(1 - s)^2
$$
If it were true as a strict inequality, we would get
from (\ref{N(-1,1)}):
$$
\left(\frac{3}{2} - r\right)^2 > 1 + n(1 - s)^2 \geq
2 + \left(\frac{1}{2} + r\right)^2 \then
\frac{9}{4} -3r + r^2 > 2 + \frac{1}{4} +r + r^2
$$
so that:
$$
- 4r > 2 + \frac{1}{4} - \frac{9}{4} = 0
\then r < 0
$$
a contradiction.
Then, if $P(1,1)$ holds, it must be\,
$\left(\frac{3}{2} - r\right)^2 = 1 + n(1 - s)^2$.
Moreover, if (\ref{N(-1,1)}) were true as a strict
inequality, we would have:
$$
\left(\frac{3}{2} - r\right)^2 = 1 + n(1 - s)^2 >
2 + \left(\frac{1}{2} + r\right)^2
$$
from which can be deduced exactly as above that $r < 0$,
a contradiction.
So, if $P(1,1)$ holds, it must be:
$$
\left(\frac{3}{2} - r\right)^2 = 1 + n(1 - s)^2
\quad\ \textrm{and}\ \quad
n(1 - s)^2 = 1 + \left(\frac{1}{2} + r\right)^2
$$
so that:
$$
\left(\frac{3}{2} - r\right)^2 =
2 + \left(\frac{1}{2} + r\right)^2 \then r = 0 \then
n (1 - s)^2 = 1 + \left(\frac{1}{2} + r\right)^2
= \frac{5}{4}
$$
from which we can finally deduce, using also
(\ref{m_and_n_bounds}):
$$
(1 - s)^2 = \frac{5}{4n} = \frac{5}{m} \leq \frac{5}{33}
\then 1 - s = \sqrt{\frac{5}{33}} < 0.4 \then s > 1 - 0.4
= 0.6 > \frac{1}{2}
$$
again a contradiction.

Thus is now proved that $P(1, 1)$ can't hold, so that
$N(1, 1)$ certainly holds, \ie:
\begin{equation}\label{N(1,1)}
n(1 - s)^2 \geq 1 + \left(\frac{3}{2} - r\right)^2
\end{equation}

Finally, consider the inequality:
$$
\left[P(-2, 1)\right]\::\;
\left(2 - \frac{1}{2} - r\right)^2 \geq 1 + n(1 - s)^2
$$
If it were true, we would get from (\ref{N(1,1)}):
$$
\left(\frac{3}{2} + r\right)^2 =
\left(2 - \frac{1}{2} - r\right)^2 \geq 1 + n(1 - s)^2
\geq 2 + \left(\frac{3}{2} - r\right)^2
$$
so that:
$$
\frac{9}{4} + 3r + r^2 \geq 2 + \frac{9}{4} - 3r + r^2
\then 6r \geq 2 \then r \geq \frac{1}{3} > \frac{1}{4}
$$
which contradicts (\ref{r_upper_bound}).
Thus $N(-2, 1)$ must hold, \ie:
\begin{equation}\label{N(-2,1)}
n(1 - s)^2 \geq 1 + \left(\frac{3}{2} + r\right)^2
\end{equation}
From this inequality we finally deduce:
$$
\frac{m}{4}(1 - s)^2 = n(1 - s)^2 \geq
1 + \left(\frac{3}{2} + r\right)^2 \geq
1 + \frac{9}{4} = \frac{13}{4}
$$
which, togheter with (\ref{s_lower_bound}), lead us to the
conclusive desired contradiction:
$$
13 \leq m(1 - s)^2 \leq 41(1 - s)^2
< 41(1 - 0.4417)^2 = 12.77965449
$$
%
\end{proof}
